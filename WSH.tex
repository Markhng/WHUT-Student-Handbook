\documentclass[UTF8,12pt,a4paper]{report}
\usepackage{ctex}
\usepackage{geometry}
\usepackage{hyperref}

\geometry{left=2cm,right=2cm,top=2.5cm,bottom=2cm}

\title{\textbf{武汉理工大学} \\ \textbf{学生手册}}
\author{武汉理工大学}
\date{2017}

\begin{document}
	
	\maketitle
该学生手册电子版版由热心网友编辑制作,供广大学生线(tong)下(guo)学(kao)习(shi)\\
该手册\LaTeX 源码见 \href{https://github.com/Markhng/WHUT-Student-Handbook/}{GitHub}\\
该书版权归武汉理工大学所有
	\tableofcontents
\part{总则}
\chapter{高等学校学生行为准则}
  一、志存高远,坚定信念。努力学习马克思列宁主义、毛泽东思想、邓小平理论和“三个代表”重要思想,面向世界,了解国情,确立在中国共产党领导下走社会主义道路、实现中华民族伟大复兴的共同理想和坚定信念,努力成为有理想、有道德、有文化、有纪律的社会主义新人。

  二、热爱祖国,服务人民。弘扬民族精神,维护国家利益和民族团结。不参与违反四项基本原则、影响国家统一和社会稳定的活动。培养同人民群众的深厚感情,正确处理国家、集体和个人三者利益关系,增强社会责任感,甘愿为祖国为人民奉献。

  三、勤奋学习,自强不息。追求真理,崇尚科学;刻苦钻研,严谨求实;积极实践,勇于创新;珍惜时间,学业有成。

  四、遵纪守法,弘扬正气。遵守宪法、法律法规,遵守校纪校规;正确行使权利,依法履行义务;敬廉崇洁,公道正派;敢于并善于同各种违法违纪行为作斗争。

  五、诚实守信,严于律己。履约践诺,知行统一;遵从学术规范,恪守学术道德,不作弊,不剽窃;自尊自爱,自省自律;文明使用互联网;自觉抵制黄、赌、毒等不良诱惑。

  六、明礼修身,团结友爱。弘扬传统美德,遵守社会公德,男女交往文明;关心集体,爱护公物,热心公益;尊敬师长,友爱同学,团结合作;仪表整洁,待人礼貌;豁达宽容,积极向上。

  七、勤俭节约,艰苦奋斗。热爱劳动,珍惜他人和社会劳动成果;生活俭朴,杜绝浪费;不追求超越自身和家庭实际的物质享受。

  八、热爱生活,强健体魄。积极参加文体活动,提高身体素质,保持心理健康;磨砺意志,不怕挫折,提高适应能力;增强安全意识,防止意外事故;关爱自然,爱护环境,珍惜资源。	
		\chapter{武汉理工大学学生管理规定}
第一章    总 则

第一条 为规范我校学生管理行为,维护学校正常的教育教学秩序和生活秩序,保障学生合法权益,培养德、智、体、美等方面全面发展的社会主义建设者和接班人,依据教育部《普通高等学校学生管理规定》(教育部令第41号)等有关法律、法规及《武汉理工大学章程》制定本规定。

第二条 本规定适用于我校对接受普通高等学历教育的研究生和本科学生(以下称学生)的管理。

第三条 学校坚持社会主义办学方向,坚持马克思主义的指导地位,全面贯彻国家教育方针;坚持以立德树人为根本,以理想信念教育为核心,培育和践行社会主义核心价值观,弘扬中华优秀传统文化和革命文化、社会主义先进文化,培养学生的社会责任感、创新精神和实践能力;坚持依法治校,科学管理,健全和完善管理制度,规范管理行为,将管理与育人相结合,不断提高管理和服务水平;着力培养“适应能力强、实干精神强、创新意识强”具有卓越追求与卓越能力的卓越人才。

第四条 学生应当拥护中国共产党领导,努力学习马克思列宁主义、毛泽东思想、中国特色社会主义理论体系,深入学习习近平总书记系列重要讲话精神和治国理政新理念新思想新战略,坚定中国特色社会主义道路自信、理论自信、制度自信、文化自信,树立中国特色社会主义共同理想;应当树立爱国主义思想,具有团结统一、爱好和平、勤劳勇敢、自强不息的精神;应当增强法治观念,遵守宪法、法律、法规,遵守公民道德规范,遵守学校管理制度,具有良好的道德品质和行为习惯;应当刻苦学习,勇于探索,积极实践,努力掌握现代科学文化知识和专业技能;应当积极锻炼身体,增进身心健康,提高个人修养,培养审美情趣。

第五条 学校在学生管理中,尊重和保护学生的合法权利,教育和引导学生承担应尽的义务与责任,鼓励和支持学生实行自我管理、自我服务、自我教育、自我监督。



第二章 学生的权利与义务

第六条 学生在校期间依法享有下列权利:

(一)参加学校教育教学计划安排的各项活动,使用学校提供的教育教学资源;

(二)参加社会实践、志愿服务、勤工助学、文娱体育及科技文化创新等活动,获得就业创业指导和服务;

(三)申请奖学金、助学金及助学贷款;

(四)在思想品德、学业成绩等方面获得科学、公正评价,完成学校规定学业后获得相应的学历证书、学位证书;

(五)在校内组织、参加学生团体,以适当方式参与学校管理,对学校与学生权益相关事务享有知情权、参与权、表达权和监督权;

(六)对学校给予的处理或者处分有异议,向学校、教育行政部门提出申诉,对学校、教职员工侵犯其人身权、财产权等合法权益的行为,提出申诉或者依法提起诉讼;

(七)法律、法规及学校章程规定的其他权利。

第七条 学生在校期间依法履行下列义务:

(一)遵守宪法和法律、法规;

(二)遵守学校章程和规章制度;

(三)恪守学术道德,完成规定学业;

(四)按规定缴纳学费及有关费用,履行获得贷学金及助学金的相应义务;

(五)遵守学生行为规范,尊敬师长,养成良好的思想品德和行为习惯;

(六)珍惜学校声誉,维护学校利益;

(七)法律、法规及学校章程规定的其他义务。



第三章 学籍管理

第一节   入学与注册

第八条 按国家招生规定录取的我校新生,持武汉理工大学录取通知书和学校规定的有关证件,按期到校办理入学手续。因故不能按期入学的,研究生应当向培养单位请假,本科生应当向学校招生部门请假,假期不得超过2周。未请假或者请假逾期的,除因不可抗力等正当事由以外,视为放弃入学资格。

第九条 学校在新生报到时对新生入学资格进行初步审查,审查合格的办理入学手续,予以注册学籍;审查发现新生的录取通知、考生信息等证明材料,与本人实际情况不符,或者有其他违反国家招生考试规定情形的,取消入学资格。

第十条 新生可以申请保留入学资格。保留入学资格期间不具有学籍。新生因身体、创业等原因,可在入学报到前或开学2周内向学校提出申请保留入学资格,保留入学资格年限不超过2年,应征参加中国人民解放军(含中国人民武装警察部队)的,可保留入学资格至退役后2年。

新生保留入学资格期满前应向学校申请入学,经学校审查合格后,办理入学手续。审查不合格的,取消入学资格;逾期不办理入学手续且无不可抗力等正当理由的,视为放弃入学资格。

第十一条 学生入学后,学校在3个月内按照国家招生规定进行复查。复查内容主要包括以下方面:

(一)录取手续及程序等是否合乎国家招生规定;

(二)所获得的录取资格是否真实、合乎相关规定;

(三)本人及身份证明与录取通知、考生档案等是否一致;

(四)身心健康状况是否符合报考专业或者专业类别体检要求,能否保证在校正常学习、生活;

(五)艺术、体育等特殊类型录取学生的专业水平是否符合录取要求。

复查中发现学生存在弄虚作假、徇私舞弊等情形的,确定为复查不合格,取消学籍;情节严重的,移交有关部门调查处理。

复查中发现学生身心状况不适宜在校学习,经学校指定的二级甲等以上医院诊断,需要在家休养的,可以按照第十条的规定保留入学资格。

第十二条 每学期开学时,学生应当按学校规定和要求报到和办理注册手续。不能如期注册的,应当履行暂缓注册手续。未按学校规定缴纳学费或者有其他不符合注册条件的,不予注册,无正当理由不注册或不办理暂缓注册手续逾期2周的,作退学处理(不可抗力等正当理由除外)。

家庭经济困难的学生可以申请助学贷款或者其他形式资助,办理有关手续后注册。

学校按照国家有关规定为家庭经济困难学生实行国家助学贷款、学费减(免)、困难补助、勤工助学等方式对学生进行资助,保证学生不因家庭经济困难而放弃学业。

第二节 考核与成绩记载

第十三条 学生应当参加学校教育教学计划规定的课程和各种教育教学环节(以下统称课程)的考核,考核成绩记入成绩册,并归入学籍档案。  

考核分为考试和考查两种。可采取开闭卷、论文等方式进行考核,成绩评定可采取百分制、五级制及两级制等形式,对于考核不合格的课程可进行重修或者补考。

第十四条 学校建立健全学生思想品德的考核、鉴定体系。对学生的思想品德的考核和鉴定通过本人小结、师生民主评议等形式进行。

学生体育成绩评定突出过程管理,可以根据考勤、课内教学、课外锻炼活动和体质健康等情况综合评定,学生因体残、体弱等原因无法参加体育课学习的,经本人申请,学校批准后,可参加体育保健课的学习。

第十五条 学生应参加学校规定的教学活动,学校为学生设定学期、学年所修课程或者应修学分数。对未取得学校规定学分数的学生予以学业警示。

第十六条 学校鼓励学生申请辅修校内其他专业或者选修其他专业课程;学生可以申请到其他高校进行跨校辅修专业或者修读课程,参加经申请并得到学校同意、认可的开放式网络课程学习。学生修读的课程成绩(学分),学校审核同意后,予以承认。

第十七条 学校鼓励、支持和指导学生参加创新创业、社会实践等活动,学生参加创新创业、社会实践等活动以及发表论文、获得专利授权等与专业学习、学业要求相关的经历、成果,按学校关于此类活动的学分折算办法折算为学分,计入学业成绩、并为学生建立创新创业档案。

第十八条 学校健全学生学业成绩和学籍档案管理制度,真实、完整地记载、出具学生学业成绩,对通过补考、重修获得的成绩,予以标注。

学生严重违反考核纪律或者作弊的,该课程考核成绩记为无效,并视其违纪或者作弊情节,给予相应的纪律处分。给予警告、严重警告、记过及留校察看处分的,经教育表现较好,对该课程给予重修机会。

学生因退学等情况中止学业,其在校学习期间所修课程及已获得学分,予以记录。学生重新参加入学考试、符合录取条件,再次入学的,其已获得学分经学校认定的,予以承认,学生也可申请重新学习。学校为录入其他学校的再次入校生出具相关证明。

第十九条 学生应当按时参加教育教学计划规定的活动。不能按时参加的,应当事先请假并获得批准。无故缺席的,根据学校有关规定给予批评教育,情节严重的,给予相应的纪律处分。累计缺课时数达该课程1/3学时者,不得参加该课程的考核。

第二十条 学校开展“责任、诚信、成才”三项教育,通过学年鉴定、学籍表等记录学生学业、学术、品行等方面的诚信信息,建立对失信行为的约束和惩戒机制;对有严重失信行为的,给予相应的纪律处分,对违背学术诚信的,可以对其获得学位及学术称号、荣誉等作出限制。

第三节 转专业与转学

第二十一条 学生在学习期间对其他专业有兴趣和专长或在某一专业无法正常完成学业的,可以申请转专业,学生转专业原则上应在低年级完成;以特殊招生形式录取的学生,国家、湖北省有相关规定或者录取前与学校有明确约定的,正在保留入学资格、休学、保留学籍的,应予退学处理或开除学籍处分的不得转专业。

学校根据社会对人才需求情况的发展变化,需要适当调整专业的,允许在读学生转到其他相关专业就读。

休学创业或退役后复学的学生,因自身情况需要转专业的,学校优先考虑。

第二十二条 被我校录取的学生应在我校完成学业。因患病或者有特殊困难、特别需要,无法继续在我校学习或者不适应我校学习要求的,可以申请转学。有下列情形之一,不得转学:

(一)入学未满一学期或者毕业前一年的;

(二)高考成绩低于拟转入学校相关专业同一生源地相应年份录取成绩的;

(三)由低学历层次转为高学历层次的;

(四)以定向就业招生录取的;

(五)研究生拟转入学校、专业的录取控制标准高于其所在学校、专业的;

(六)无正当转学理由或上级主管部门规定不得转学的。

学生因学校培养条件改变等非本人原因需要转学的,学校出具证明,由所在地省级教育行政部门协调转学到同层次学校。

第二十三条 学生转学由学生本人提出申请,说明理由,经所在学校和拟转入学校同意,由拟转入学校负责审核转学条件及相关证明,认为符合本校培养要求且学校有培养能力的,经学校校长办公会或者专题会议研究决定,可以转入。研究生转学还应当经拟转入专业导师同意。

跨省转学的,由转出地省级教育行政部门商转入地省级教育行政部门,按转学条件确认后办理转学手续。须转户口的由转入地省级教育行政部门将有关文件抄送转入学校所在地的公安机关。

第二十四条  学生转学一般在每年6月份或12月份申请,在符合国家政策并得到拟转入学校同意接收的条件下,学校将按规定对转学情况进行答复与公示,并在转学完成后3个月内,将转入学生名单报湖北省教育厅备案。

第四节 休学与复学

第二十五条 学生可以分阶段完成学业,除另有规定外。本科生应当在主修专业规定的学制加4年的学习年限(含休学和保留学籍,下同)内完成学业;硕士研究生学习年限不超过6年,博士研究生学习年限不超过8年。

   学生申请休学或者学校认为应当休学的,经学校批准,可以休学。休学次数原则上不超过2次,不具有完全民事行为能力的学生休学需经其监护人同意。

第二十六条 学校根据情况建立并实行灵活的学习制度。对学生休学创业,简化休学批准程序,休学创业学生的最长学习年限不超过10年。

第二十七条 新生和在校学生应征参加中国人民解放军(含中国人民武装警察部队),学校保留其入学资格或者学籍至退役后2年。

学生参加学校组织的跨校联合培养项目,在联合培养学校学习期间,学校同时为其保留学籍。

学生保留学籍期间,与其实际所在的部队、学校等组织建立管理关系。

第二十八条 休学学生应当办理手续离校并在5个工作日内离开学校。学生休学期间,学校为其保留学籍,但不享受在校学习学生待遇。因病休学学生的医疗费按国家及当地的有关规定处理。

第二十九条 学生应在休学或保留学籍期满前2周内提出复学申请,经学校复查合格,方可复学,无正当理由期满2周后仍未提出复学申请(不可抗力除外)的作退学处理。

第五节 退学

第三十条 学生有下列情形之一,学校可予以退学处理:

(一)学业成绩未达到学校要求或者在学校规定的学习年限内未完成学业的;

(二)休学、保留学籍期满,在学校规定期限内未提出复学申请或者申请复学经复查不合格的;

(三)根据学校指定医院诊断,患有疾病或者意外伤残不能继续在校学习的;

(四)未经批准连续两周未参加学校规定的教学活动的;

(五)超过学校规定期限未注册而又未履行暂缓注册手续的;

(六)学校规定的不能完成学业、应予退学的其他情形。

学生本人申请退学的,经学校审核同意后,办理退学手续。

第三十一条 退学学生,应当在10个工作日内办理退学手续离校。退学的研究生,按已有毕业学历和就业政策可以就业的,由学校报湖北省毕业生就业部门办理相关手续;在学校规定期限内没有聘用单位的,应当办理退学手续离校。

对退学学生,学校发给肄业证书或者写实性学习证明。

退学学生的档案由学校退回其家庭所在地,户口按照国家相关规定迁回原户籍地或者家庭户籍所在地。

第六节 毕业与结业

第三十二条 学生在学校规定学习年限内,修完教育教学计划规定内容,成绩合格,本科生达到学校毕业要求的,学校准予毕业,并在学生离校前发给毕业证书;研究生通过毕业/学位论文答辩,并经学位评定分委员会表决通过后,准予毕业并发给毕业证书。

符合学校学位授予条件的学生,学校为其颁发学位证书。

学校允许学生提前毕业,学生提前毕业的条件与程序按提前毕业的相关规定执行。

第三十三条 学生在学校规定学习年限内,修完教育教学计划规定内容,但未达到学校毕业要求的,学校可以准予结业,发给结业证书。

研究生在结业后,若完成毕业/学位论文撰写并经导师同意,且达到申请相应学位条件的,可提出一次毕业/学位论文答辩申请。合格后颁发的毕业证书、学位证书,毕业时间、获得学位时间按发证日期填写。

本科生在结业后不再具有我校学籍,不得继续在校参加教学活动,学校不再向其颁发毕业证书。

第七节 学业证书管理

第三十四条 学校严格按照招生时确定的办学类型和学习形式,以及学生招生录取时填报的个人信息,填写、颁发学历证书、学位证书及其他学业证书。

学生在校期间变更姓名、出生日期等证书需填写的个人信息的,应当有合理、充分的理由,并提供有法定效力的相应证明文件。学校对相关证明文件进行审查时,应请学生生源地省级教育行政部门及有关部门协助核查。

第三十五条 学校执行高等教育学籍学历电子注册管理制度,完善学籍学历信息管理办法,按相关规定及时完成学生学籍学历电子注册。

第三十六条 对完成本专业学业同时辅修其他专业并达到该专业辅修要求的学生,由学校发给辅修专业证书。

第三十七条 对违反国家招生规定取得入学资格或者学籍的,学校取消其学籍,不发给学历证书、学位证书;已发的学历证书、学位证书,学校依法予以撤销。对以作弊、剽窃、抄袭等学术不端行为或者其他不正当手段获得学历证书、学位证书的,学校依法予以撤销。

被撤销的学历证书、学位证书已注册的,学校予以注销并报教育行政部门宣布无效。

第三十八条 学历证书和学位证书遗失或者损坏,经本人申请,学校核实后出具相应的证明书。证明书与原证书具有同等效力。



第四章 校园秩序与课外活动

第三十九条 学校、学生应当共同维护校园正常秩序,保障学校环境安全、稳定,保障学生的正常学习和生活。

第四十条 学校建立和完善学生参与管理的组织形式,支持和保障学生依法、依章程参与学校管理。

第四十一条 学生应当自觉遵守公民道德规范,自觉遵守学校管理制度,创造和维护文明、整洁、优美、安全的学习和生活环境,树立安全风险防范和自我保护意识,保障自身合法权益。

第四十二条 学生不得有酗酒、打架斗殴、赌博、吸毒,传播、复制、贩卖非法书刊和音像制品等违法行为;不得参与非法传销和进行邪教、封建迷信活动;不得从事或者参与有损大学生形象、有悖社会公序良俗的活动。

学校发现学生在校内有违法行为或者严重精神疾病可能对他人造成伤害的,可以依法采取或者协助有关部门采取必要措施。

第四十三条 学校坚持教育与宗教相分离原则。任何组织和个人不得在学校进行宗教活动。

第四十四条 学校建立健全学生代表大会制度,为学生会、研究生会等开展活动提供必要条件,支持其在学生管理中发挥作用。

学生可以在校内成立、参加学生团体。学生成立团体,应当按学校有关规定提出书面申请,报学校批准并施行登记和年检制度。

学生团体应当在宪法、法律、法规和学校管理制度范围内活动,接受学校的领导和管理。学生团体邀请校外组织、人员到校举办讲座等活动,需经学校批准。

第四十五条 学校提倡并支持学生及学生团体开展有益于身心健康、成长成才的学术、科技、艺术、文娱、体育等活动。

学生进行课外活动不得影响学校正常的教育教学秩序和生活秩序。

学生参加勤工助学活动应当遵守法律、法规以及学校、用工单位的管理制度,履行勤工助学活动的有关协议。

第四十六条 学生举行大型集会、游行、示威等活动,应当按法律程序和有关规定获得批准。对未获批准的,学校依法劝阻或者制止。

第四十七条 学生应当遵守国家和学校关于网络使用的有关规定,不得登录非法网站和传播非法文字、音频、视频资料等,不得编造或者传播虚假、有害信息;不得攻击、侵入他人计算机和移动通讯网络系统。

第四十八条 学校建立健全学生住宿管理制度。学生应当遵守学校关于学生住宿管理的规定,参加文明宿舍的创建。学校鼓励和支持学生通过制定公约,实施自我管理,学生参与文明宿舍的建设情况应纳入对学生的综合评价中。



第五章 奖励与处分

第四十九条 学校对在德、智、体、美等方面全面发展或者在思想品德、学业成绩、科技创造、体育竞赛、文艺活动、志愿服务及社会实践等方面表现突出的学生,给予表彰和奖励。

第五十条 学校对学生的表彰和奖励采取授予“三好学生”称号或者其他荣誉称号、颁发奖学金等多种形式,给予相应的精神鼓励或者物质奖励。

学校接受社会捐助扩大对学生的物质奖励。

学校对学生予以表彰和奖励,以及确定推荐免试研究生、国家奖学金、公派出国留学人选等赋予学生利益的行为,建立公开、公平、公正的程序和规定,建立和完善相应的选拔、公示等制度。

第五十一条 对有违反法律法规、本规定以及学校纪律行为的学生,学校给予批评教育,并可视情节轻重,给予如下纪律处分:

(一)警告;

(二)严重警告;

(三)记过;

(四)留校察看;

(五)开除学籍。

第五十二条 学生有下列情形之一,学校可以给予开除学籍处分:

(一)违反宪法,反对四项基本原则、破坏安定团结、扰乱社会秩序的;

(二)触犯国家法律,构成刑事犯罪的;

(三)受到治安管理处罚,情节严重、性质恶劣的;

(四)代替他人或者让他人代替自己参加考试、组织作弊、使用通讯设备或其他器材作弊、向他人出售考试试题或答案牟取利益,以及其他严重作弊或扰乱考试秩序行为的;

(五)学位论文、公开发表的研究成果存在抄袭、篡改、伪造等学术不端行为,情节严重的,或者代写论文、买卖论文的;

(六)违反本规定和学校其他管理规定,严重影响学校教育教学秩序、生活秩序以及公共场所管理秩序的;

(七)侵害其他个人、组织合法权益,造成严重后果的;

(八)屡次违反学校规定受到纪律处分,经教育不改的。

第五十三条 学校对学生作出处分,出具处分决定书。处分决定书应当包括下列内容:

(一)学生的基本信息;

(二)作出处分的事实和证据;

(三)处分的种类、依据、期限;

(四)申诉的途径和期限;

(五)其他必要内容。

第五十四条 学校坚持教育与惩戒相结合,与学生违法、违纪行为的性质和过错的严重程度相适应原则给予学生处分,学校对学生的处分,做到证据充分、依据明确、定性准确、程序正当、处分适当。

第五十五条 在对学生作出处分或者其他不利决定之前,学校告知学生作出决定的事实、理由及依据,并告知学生享有陈述和申辩的权利,听取学生的陈述和申辩。

处理、处分决定以及处分告知书等,直接送达学生本人,学生拒绝签收的,可以以留置方式送达;已离校的,可以采取邮寄方式送达;难于联系的,可以利用学校网站、新闻媒体等以公告方式送达。

第五十六条 对学生作出取消入学资格、取消学籍、退学、开除学籍或者其他涉及学生重大利益的处理或者处分决定的,提交校长办公会或者校长授权的专门会议研究决定,并事先进行合法性审查。

第五十七条 除开除学籍处分以外,给予学生处分设置6到12个月期限。解除处分由学生本人申请,经评议后由学校按规定程序予以解除。解除处分后,学生获得表彰、奖励及其他权益,不再受原处分的影响。

纪律处分期限:

(一)警告,6个月;

(二)严重警告,8个月;

(三)记过,10个月;

(四)留校察看,12个月。

第五十八条 对学生的奖励、处理、处分及解除处分材料,学校真实完整地归入学校文书档案和本人档案。

被开除学籍的学生,由学校发给学习证明。学生按学校规定期限离校,档案由学校退回其家庭所在地,户口按照国家相关规定迁回原户籍地或者家庭户籍所在地。



第六章 学生申诉

第五十九条 学校成立学生申诉处理委员会,负责受理学生对处理或者处分决定不服提起的申诉。

学生申诉处理委员会由学校相关负责人、职能部门负责人、教师代表、学生代表、负责法律事务的相关机构负责人等组成,可以聘请校外法律、教育等方面专家参加。

学校制定学生申诉的具体办法,健全学生申诉处理委员会的组成与工作规则,提供必要条件,保证其能够客观、公正地履行职责。  

第六十条 学生对学校的处理或者处分决定有异议的,可以在接到学校处理或者处分决定书之日起10日内,向学校学生申诉处理委员会提出书面申诉。

第六十一条 学生申诉处理委员会对学生提出的申诉进行复查,并在接到书面申诉之日起15日内作出复查结论并告知申诉人。情况复杂不能在规定限期内作出结论的,经学校负责人批准,可延长15日。学生申诉处理委员会认为必要的,可以建议学校暂缓执行有关决定。

学生申诉处理委员会经复查,认为做出处理或者处分的事实、依据、程序等存在不当,可以作出建议撤销或变更的复查意见,要求相关职能部门予以研究,重新提交校长办公会或者专门会议作出决定。

第六十二条 学生对复查决定有异议的,在接到学校复查决定书之日起15日内,可以向湖北省教育厅提出书面申诉。

第六十三条 自处理、处分或者复查决定书送达之日起,学生在申诉期内未提出申诉的视为放弃申诉,学校不再受理其提出的申诉。

处理、处分或者复查决定书未告知学生申诉期限的,申诉期限自学生知道或者应当知道处理或者处分决定之日起计算,但最长不得超过6个月。

第六十四条 学生认为学校及其工作人员违反本规定,侵害其合法权益的;或者学校制定的规章制度与法律法规和本规定抵触的,可以向湖北省教育厅提出书面投诉。



第七章 附 则

第六十五条 对在我校接受高等学历继续教育的学生、港澳台侨学生、留学生的管理,参照本规定执行。

第六十六条  本规定自2017年9月1日起施行,学校其他学生管理类规定和办法与本规定不一致的,以本规定为准。

第六十七条  此规定由教务处、研究生院、学生工作部(处)负责解释。
	\part{学籍管理}
		\chapter{武汉理工大学 普通全日制本科学生学籍管理规定}
第一章  总则

第一条  为全面贯彻执行国家教育方针,维护学校正常的教育教学秩序和生活秩序,保障学生合法权益,不断提高教育质量,培养德、智、体、美等方面全面发展的社会主义建设者和接班人,依据《普通高等学校学生管理规定》(教育部令第41号)等法律法规,结合我校实际,制定本规定。

第二条  本规定适用于我校按照国家规定录取的接受普通高等学历教育的全日制本科学生学籍管理。



第二章 入学与注册

第三条  按国家招生规定录取的我校普通全日制新生,必须持《武汉理工大学入学录取通知书》和学校规定的有关证件,按期到校办理入学手续。因故不能按期入学者,应在报到截止日期前向学校招生部门请假,假期不得超过2周。未经请假或请假逾期的,除因不可抗力等正当事由以外,视为放弃入学资格。

第四条  学校在新生报到时,对新生入学资格进行初步审查,审查合格的办理入学手续,予以注册学籍;审查发现新生的录取通知、考生信息等证明材料,与本人实际情况不符,或者有其他违反国家招生考试规定情形的,取消入学资格。

第五条  新生因创新创业、应征入伍、身心状况不佳等原因,可在入学报到前或开学2周内向学校招生部门申请保留入学资格。保留入学资格期间不具有我校学籍。

保留入学资格的期限不超过2年,应征参加中国人民解放军(含中国人民武装警察部队)的,可保留入学资格到退役后2年。

新生保留入学资格期满前应向学校招生部门申请入学,经学校审查合格后,与应届新生一同办理入学手续。审查不合格的,取消入学资格;逾期不办理入学手续且未有因不可抗力延迟等正当理由的,视为放弃入学资格。

第六条  新生入学后,学校在3个月内按照国家规定进行复查。复查内容主要包括以下方面:

(一)录取手续及程序等是否合乎国家招生规定;

(二)所获得的录取资格是否真实、合乎相关规定;

(三)本人及身份证明与录取通知、考生档案等是否一致;

(四)身心健康状况是否符合报考专业或者专业类别体检要求,能否保证在校正常学习、生活;

(五)艺术、体育等特殊类型录取学生的专业水平是否符合录取要求。

复查中发现学生存在弄虚作假、徇私舞弊等情形的,确定为复查不合格,取消学籍;情节严重的,学校将移交有关部门调查处理。

复查中发现学生身心状况不适宜在校学习,经学校指定的二级甲等(含)以上医院诊断,需要在家休养的,学生可以按照第五条的规定保留入学资格。

第七条  每学期开学时,学生须按学校规定和要求到校报到,办理注册手续。不能如期注册的,应当履行暂缓注册手续。未按学校规定缴纳学费或者有其他不符合注册条件的,不予注册。

家庭经济困难的学生可以申请助学贷款或者其他形式资助,办理有关手续后注册。

未履行暂缓注册手续、逾期2周不注册的,作退学处理(不可抗力等正当事由除外)。



第三章  学制与学习年限

第八条  学制是指国家对各级各类学校各专业设定的课程所需的必要学习时间的规定,即学生完成专业设定的课程、学习任务一般所需要的学习年限。学生一般应在学制年限内完成学业。

第九条  学生学习年限(含休学、保留学籍)为学生主修专业规定的学制加4年。因创新创业休学的,经本人申请、学校批淮,可再延长学习年限,但最长不超过10年。  



第四章  考勤与纪律

第十条  学生上课、实习、实训、实验、课程设计及毕业设计(论文)、政治理论学习、劳动等都应该实行考勤。

教学活动的考勤由任课教师负责,其他活动由组织单位负责。

第十一条  学生应按照学校教育教学计划要求,认真学习各门课程,积极参加学校规定的各项活动。

学生上课时应遵守课堂纪律,认真听课,不得无故迟到、早退,未经教师同意不得擅自离开教室。学生未经批准,不得缺勤,违者以旷课论处。

第十二条  学生因病或其他原因不能参加学校教育教学活动时,须事先请假并获得批准。未经批准而缺席者,根据学校有关规定给予批评教育,情节严重的给予纪律处分。除因不可抗力等正当事由外,不得事后补假。请假程序及准假权限为:

(一)3天以内(含3天),报辅导员批准(在外实习期间由带队老师批准);

(二)3天以上1个月以内,由学院主管学生工作的领导批准,报学生工作部(处)备案;

(三)1个月以上2个月以内,由学院主管教学工作的领导和主管学生工作的领导批准,报教务处、学生工作部(处)备案。

学生一学期内请假累计不得超过2个月。超过的,按休学或保留学籍相关规定办理。

第十三条  学生旷课时数按实际课时计。

学生上课迟到或早退达15分钟以上者,按旷课1学时计算;迟到、早退少于15分钟的,累计3次按旷课1学时计算。

学生一学期内旷课时数累计超过该门课程教学时数1/3,取消其该课程考核资格,成绩记为零分。



第五章  课程与学分

第十四条  专业培养计划所设置的课程分为必修课程和选修课程。

必修课程指培养计划规定学生必须修读的理论课程和实践性教学环节。

选修课程指培养计划中列出的可以由学生结合个人志愿和专长选修的课程,包括全校开设的选修课程以及各专业开设的选修课程。

第十五条  学分是计算学生学习份量的单位。培养计划中设定的各类课程均规定一定的学分。

学生在校期间,必须修完培养计划中规定的课程,达到培养计划的学分要求,方能毕业。

第十六条  学生学习的努力程度,采用学年总学分数作为评价指标;学生学习的质量水平,采用平均学分绩点(GPA)作为评价指标。

课程学分绩点 = 课程绩点$\times$课程学分

课程绩点根据课程考核成绩确定,具体折算标准如下:\\

\href{http://img01.fs.yiban.cn/out/thumb_550x0/aHR0cDovL3lmczAxLmZzLnlpYmFuLmNuL3dlYi83NTg4OTE0L3VwbG9hZC8xNTA0NzY5NDc2Mzc4OTYyLnBuZw==}{\textit{图片链接}}\\

补考不合格、通过重修合格的课程,课程绩点按“1.0”折算;补考合格的课程,课程绩点按“1.0”折算;

考核(包括正考、补考或重修考核)合格后再次重修的课程,在推荐免试研究生、评优评先时,课程绩点取第一次考核合格时的折算绩点;在申请学位、对外出具成绩单时,取最高折算绩点。



第六章  课程修读

第十七条  学生根据培养计划要求及自身情况,办理相应的选课手续后进行课程修读。选课的要求及程序按照学校学生选课管理办法执行。

学生每学期选择修读的课程学分数(不含课外学分)原则上最低不得少于15学分,最高不得超过35学分。

学校对毕业设计(论文)实行准入制度。学生所修学分达到专业规定的毕业设计(论文)准入学分,方允许进入毕业设计(论文)阶段。

第十八条  进校满一年的学生,成绩优良、已修课程平均学分绩点达到3.5及以上、平均每学期取得的学分数达到25学分及以上(不含课外学分)的,可申请免听部分课程。

学生已经考核合格的课程,因个人需要再次重修的,可以申请免听。

政治理论课、军事理论课、实验课、体育课、各类实践教学环节以及各专业规定不能免听的课程不得免听。

学生每学期可以免听的课程总学分不超过8学分或课程门数不超过2门。

第十九条  符合规定的学生,可在开学后2周内提出课程免听申请,经任课教师同意,学生所在学院审核、批准后报教务处备案。免听的学生必须完成教师指定学习任务,如作业、实验等,经教师同意后方可参加该课程的期末考试。成绩合格,取得该课程学分。

第二十条  进校满一年的学生,自学能力强、成绩优良、已修课程平均学分绩点达到3.5及以上、平均每学期取得的学分数达到25学分及以上(不含课外学分)的,可以申请免修通过自学已经掌握的课程。

政治理论课、军事理论课、实验课、体育课、各类实践教学环节以及各专业规定不能免修的课程不得免修。

第二十一条  符合规定的学生,可在开学第1周提交课程免修申请,同时提供自学材料(作业、读书笔记等),经任课老师同意,学生所在学院审核、批准后,参加免修考试。考试成绩70分以上(含)的,准予免修,取得该门课程学分,成绩按免修考试实际成绩记。

免修考试由开课学院命题,试题的份量和难度应与该课程期末考试相同。考试由教务处组织,一般安排在开学后第二周进行。

退役士兵复学后,可申请免修体育课、军事训练以及军事理论课程,成绩按80分(或合格)记。

第二十二条  学生必修课程考核不合格,必须重修。

学生选修课程考核不合格,可以重修或者改选其他课程。

学生课程考核合格但成绩不够理想,也可以重修,但只能重修一次。

第二十三条  重修课程必须办理选课手续。

学生重修后取得的课程成绩,以实际成绩记,并注明“重修”字样。

第二十四条  学生可以申请选修本校或学校认可的外校其他专业的课程,参加学校认可的开放式网络课程学习。

学生修读的课程成绩(学分),经学院、教务处审核同意后,予以承认。



第七章  课程考核与成绩记载

第二十五条  学生应参加所修课程的考核,考核合格方能获得学分。

学生不得参加未选课程的考核。自行参加考核者,成绩无效。

办理了选课手续、但未按规定参加考核且未办理退选或缓考手续者,视为旷考,该门课程成绩记为零分。

第二十六条  考核分为考试和考查两种,具体考核方式由任课教师或课程负责人根据课程特点及教学要求确定,学院批准后报教务处备案。除课程结束时的考核外,任课教师要加强对学生日常学习过程的考核,如期中考试、小测验、大作业、课堂讨论、实验、论文、考勤等。

课程考核成绩,由平时成绩(如期中考试、小测验、大作业、课堂讨论、实验、论文、考勤等)和课程结束时的考试成绩综合评定。课程结束时的考试成绩占总成绩的比例,由任课教师或课程负责人确定,学院批准后报教务处备案。

体育课成绩应根据考勤、课内教学、课外锻炼活动和体质健康等情况综合评定。学生因体残、体弱无法参加体育课学习的,本人申请,经校医院诊断证明,教务处批准,可参加体育保健课的学习,成绩记载时注明“保健课”。

第二十七条  考核成绩的评定,采用百分制、五级制或两级制。60分(及格或者合格)以上即取得相应学分。考试课的成绩评定采用百分制,考查课程及所有实践环节成绩评定采用五级制(优秀、良好、中等、及格、不及格)或两级制(合格、不合格)。

百分制换算为五级制:90-100,优秀;80-89,良好;70-79,中等;60-69,及格;60分以下,不及格。

五级制换算为百分制:优秀,95;良好,85;中等,75;及格,65;不及格,50。

百分制换算为两级制:60-100,合格;60分以下,不合格。

两级制换算为百分制:合格,80;不合格,50。

第二十八条  学生应按老师的要求按时完成课程实验(包括实验报告)及作业。缺交作业或实验报告超过应交总数的1/3者,取消考核资格,成绩记为零分。

第二十九条  学生应按学校公布的考试日程安排,按时参加考试。凡未经批准不参加考试者视为旷考,成绩记为零分。

学生学习及考核过程应诚实守信,遵守学校学习和考核纪律。严重违反考核纪律或作弊的,该门课程考核成绩无效,以零分记,并按学校考试违规处理办法给予相应的纪律处分,处分材料归入学校档案及学生学籍档案。

第三十条  学生因病或其他特殊原因不能参加考核的,必须在考前提出缓考申请,经学院审核,教务处批准后方能生效。

学生因病申请缓考由校医院出具证明;因公缓考由相关单位出具证明;因考试时间冲突申请缓考由学生所在学院教学办公室核实。

缓考课程的考试原则上随该门课程的补考进行,成绩的评定、记载与正常考试相同。缓考不及格,不予补考。

第三十一条  学生第一次修读且考核不合格的课程,可补考一次。全校开设的通识教育选修课、个性课,体育课,单独设课的实验课,实践性教学环节不能补考,只能重修。

学生补考后取得的成绩,以60分记,并注明“补考”字样。

第三十二条  有下列情形之一的,取消补考资格:

(一)被取消考核资格的;

(二)无故旷考的;

(三)因违反考试纪律或考试作弊等成绩无效的。

第三十三条  课程考核成绩由任课教师于考后5~7日内上网录入。学生如对本人的考核成绩有异议,可在课程考核成绩公布后至下学期开学2周内书面提出复查申请,经教务处批准,由开课学院复查并作出结论。超过规定时限不予复查。

第三十四条  学生课程考核成绩及学分载入学生成绩表,并归入学生个人学籍档案及学校档案。

参加了多次考核的课程,归入个人学籍档案时,以最优成绩记,最优成绩如果是通过补考或重修取得的,成绩表中予以标记;归入学校档案时,历次成绩及学分均载入。

第三十五条  学生因个人需要,可以在满足毕业学分要求的情况下,申请放弃1-2门选修课程或不在培养计划要求内的课程。放弃的课程,成绩及学分不再记入学生个人学籍档案。无论何种原因,已放弃的课程不再恢复。

第三十六条  学生参加创新创业、社会实践等活动以及发表论文、获得专利授权等与专业学习、学业要求相关的经历、成果,按学校有关规定折算为相应学分,计入学业成绩。

第三十七条  学生因退学等情况中止学业,其在校学习期间所修课程及已获得学分,予以记录。学生重新参加入学考试、符合录取条件,再次入学的,其已获得学分,经学校认定,予以承认。



第八章  主修专业确认、转专业、转学

第三十八条  按学科大类入校的学生,一般在入学1-2年内进行主修专业确认。主修专业确认办法由各学院负责制定,报教务处审批后向学生公布。学院应在尊重学生意愿的基础上,加强指导。主修专业确认工作完成后,学院报教务处办理学籍异动手续。

按专业入校的学生,一般以录取专业作为主修专业。

第三十九条  学生有以下情况之一者,可以提出转专业申请:

(一)确有拟转入专业的专长和兴趣,转专业后更能发挥其专长和兴趣的。优先考虑参与创新创业、并取得一定成绩的学生转入相关主修专业的需求;

(二)因某种疾病或生理缺陷(隐瞒既往病史者除外),经校医院检查证明确实不能在原专业学习,但尚能在其他专业学习的;

(三)确有某种特殊困难,在原专业无法继续学习的;

(四)因社会对人才需求情况发生变化,学校专业发生合并、撤消等,需要调整到其他专业就读的。

大学生士兵退役后复学,因自身情况需要调整专业的优先考虑。

按学科大类入校的学生,主修专业确认前,学科大类视同专业。

第四十条  有下列情况之一,不予转专业:

(一)国家有相关规定或录取前与学校有明确约定不得转专业的;

(二)正在保留入学资格、休学、保留学籍的;

(三)应予退学处理或开除学籍处分的;

(四)无正当理由的。

第四十一条  学生转专业原则上应在低年级完成。教务处于每年四月发布转专业通知,各学院负责制定本院转专业办法,报教务处审批后向学生公布。需要转专业的学生,提出转专业申请,经双方学院考核同意后,报教务处审批。

第四十二条  学生一般应在被录取学校完成学业,有下列情况之一者,可准予转学:

(一)学生入学后发现某种疾病或生理缺陷,经学校指定校医院检查证明不能在本校学习,尚能在其他高校学习的;  

(二)学生有特殊困难、特别需要,不转学则无法继续学习的。

第四十三条  有下列情况之一,不予转学:

(一)入学未满一学期的或者毕业年级的;

(二)高考成绩低于拟转入学校相关专业同一生源地相应年份录取成绩的;

(三)由低学历层次转为高学历层次的;

(四)以定向就业招生录取的;

(五)无正当转学理由的;

(六)因其他上级主管部门规定不得转学的。

第四十四条  转学的具体程序如下:

(一)学生申请转出的:本人提出申请,说明理由,学校提供相关材料,提交转入学校审核,转入学校同意后发接收函通知本校,可以转出。跨省转学的,由学校报湖北省教育厅商转入地省级教育行政部门,按转学条件确认后办理转学手续。

(二)外校学生申请转入的:经转出学校同意后,学生向拟转入学院提出申请,说明理由,提供相关证明材料,学院党政联席会议研究认为符合我校培养要求且学校有教学能力的,提交教务处审核条件及相关证明,审核通过后,报学校专题会议研究决定,由分管校领导签署接收函,可以转入。跨省转学的,由转出地省级教育行政部门商湖北省教育厅,按转学条件确认后办理转学手续。

第四十五条  学生转学一般在每年6月份或12月份申请。学校将按规定对转学情况进行公示,并在转学完成后3个月内,将转入学生名单报湖北省教育厅备案。

第四十六条  转出学生的转学申请一经批准,必须在10个工作日内办理离校手续并离校。凡不按照规定离校的学生,产生的后果,由学生本人负责。

第九章  休学、保留学籍、复学

第四十七条  学生可以分阶段完成学业。中途中断学业的须办理休学或保留学籍手续。休学或保留学籍期满须办理复学手续后方可继续在校学习。

第四十八条  学生申请休学(不具有完全行为能力的学生,需经监护人同意)或学校认为应当休学的,经批准,可以休学。

学生有下列情况之一者,应予休学:

(一)因病经指定医院诊断,须停课治疗、病休时间在一学期内超过2个月的;

(二)一学期内因病、因事请假累计超过2个月的;

(三)因某种特殊原因,学校认为应当休学的。

第四十九条  学生有下列情形之一者,可以申请保留学籍:

(一)应征参加中国人民解放军(含中国人民武装警察部队);

(二)参加学校组织的跨校联合培养项目;

(三)学生个人联系并自费出国留学;

(四)到国际组织实习。

第五十条  休学期限一般为1年,因病重或其他原因经教务处批准,可连续休学,但累计不能超过2年;休学创业的学生可再延3年。

保留学籍的期限以实际情况为准,应征参加中国人民解放军(含中国人民武装警察部队)的,可保留学籍至退役后2年(学生服兵役的时间不计入学习年限);到国际组织实习的,保留学籍的最长期限为2年。

第五十一条  学生申请休学或保留学籍需由本人提出书面申请,并附相关证明,经学院主管领导签署意见后,报教务处批准。学院认为应当休学的,由学院提出书面报告送教务处审批。

按学校学籍管理规定应予退学或开除学籍的学生,不得办理休学或保留学籍手续。

第五十二条  休学或保留学籍期间,学校保留其学籍,学生户口可不迁出学校,但不享受在校生的待遇。休学和保留学籍期间的医疗待遇按国家及武汉市的有关规定处理。

第五十三条  学生办理休学或保留学籍,一般不得迟于学期考试周(毕业答辩)开始前2周。如遇重大疾病(附医院证明)或重要事故(附相关证明),原则上在学期考试周(毕业答辩)开始前提出申请。

第五十四条  学生休学或保留学籍一经批准,应立即停止校内一切活动,在5个工作日内办理相关手续离校。学生休学或保留学籍期间自行参加考试,成绩无效。

第五十五条  学生休学或保留学籍期满,因个人原因需要续休或继续保留学籍的,可在期满前或期满后2周内提出申请,程序同上。

第五十六条  学生休学或保留学籍期满、需要复学的,应在期满2周内(遇寒暑假,顺延)提出复学申请,经学院主管领导同意、教务处批准后复学。

休学或保留学籍学生原则上不能提前复学。

第五十七条  因病休学的学生,申请复学时必须由二级甲等以上医院诊断,证明恢复健康,并经学校医院(心理疾病经学校心理健康教育中心)复查合格,方可复学。

第五十八条  教务处根据复学学生已修课程及取得学分的情况,将其编入原专业相应年级学习。如原专业已发生变化(调整,合并或中断招生),安排到其他相近专业学习。

第五十九条  学生休学或保留学籍期间,如有严重违纪违法犯罪行为者,一经查明,取消其复学资格,作退学处理。

第六十条  学生休学或保留学籍期满,未在2周内提出复学申请的,除不可抗力等正当事由外,取消复学资格并作退学处理。



第十章  辅修

第六十一条  学生入校后,在学有余力的情况下,经本人申请、教务处批准,可辅修本校或外校其他专业。优先满足创新创业的学生对辅修专业学习的需求。

第六十二条  学生在完成主修专业学业的同时,按规定和要求修完辅修专业的全部课程,经考核成绩合格者,可发给辅修证书;符合开设学校辅修学位授予条件的,由开设学校授予辅修学位。

第六十三条  辅修学生管理按学校辅修管理相关规定执行。



第十一章  出国(境)学习

第六十四条  学生参加学校公派项目出国(境)学习,需本人申请,学院审核同意后报教务处审批。

第六十五条  学生在国(境)外学习期间,学校保留武汉理工大学学籍。学生学习期满应履行协议按时返校,不得擅自延长或转往其他地区。在规定期限内不能按时回国者,必须及时书面告知学校在外滞留原因以及延期期限,并征得学校同意。未经批准逾期不归超过2周者,除不可抗力等正当事由外,作退学处理。

第六十六条  学生需按协议要求完成学习任务,并按期将学习成绩及学习情况反馈到所在学院。期间所修课程及学分,按学校国际合作教育与交流管理规定进行认定。

第六十七条  学生因私出国(境)学习,可以申请保留学籍。保留学籍的学生,不享受在校生待遇,保留学籍期满不办理复学手续且未申请延期(或申请未批准)超过2周的,除不可抗力等正当事由外,作退学处理。

第六十八条  学校鼓励学生到国(境)外知名高校(世界排名前200名的学校)学习。学生在上述学校取得的学分,可由本人申请、学院审核、报教务处批准后认定为本校相应学分,学分认定办法参照学校国际合作教育与交流管理规定执行。



第十二章  学业警示与退学

第六十九条  学校对学生实行学业警示制度。学生取得的主修专业学年总学分数(不含课外学分)低于30学分(2017年及以后入学的学生,低于27学分)将被给予学业警示。

学院负责在每学年结束时对学生所修学分进行统计,对应受学业警示学生出具学业警示通知单,并报教务处备案。

第七十条  学生有下列情形之一,应作退学处理:

(一)在读期间累计两次受到学业警示的(主修专业平均学年总学分数(不计课外学分)达到30学分(2017年及以后入学的学生,达到27学分)除外);   

(二)本科学生在籍时间超过其最长学习年限的;

(三)经学校动员,因病该休学而不休学,且在一学期内缺课超过总学时1/3的;

(四)经二级甲等以上医院诊断,患有疾病或意外伤残无法继续在校学习的;

(五)休学或保留学籍期满,未在期满2周内提出复学申请或申请复学经复查不合格的;

(六)每学期开学时,未经批准逾期2周不报到注册的;

(七)未经批准连续2周未参加学校规定的教学活动的;

(八)公派出国(境)学生未经批准逾期不归超过2周以上的;

(九)本人申请退学,经劝说无效的。

有前款第(五)至第(八)项的情形,但有不可抗力等正当理由的除外。

第七十一条  因本办法第七十条第一款第(一)至第(八)项原因退学的学生,由学生所在学院提出书面报告,并附相关材料,报送教务处。在进行合法性审查后,提交校长办公会议研究决定。

学院提出处理报告前,应告知学生退学处理的理由和依据,并告知学生享有陈述和申辩的权利。学生有陈述和申辩要求的,以书面形式提交。

第七十二条  学生本人申请退学的,由本人填写退学申请表,家长签名、学院签署意见后报送教务处,经分管校领导审核同意后,办理退学手续。

第七十三条  对退学学生的处理,由学校出具退学通知,学生所在学院负责送交学生本人。学生本人拒绝签收的,可以以留置方式送达;已离校的,可以采取邮寄方式送达;无法取得联系的,可以利用学院网站以公告方式送达。

第七十四条  退学学生应在通知送达(或公告)之日起10个工作日内办理退学手续并离校,档案由学校退回其家庭所在地,户口按照国家规定迁回原户籍地或者家庭户籍所在地。逾期不办理离校手续的,产生的后果,由学生本人承担。

经二级甲等以上医院诊断,患有疾病或意外伤残无法继续在校学习者,由家长或抚养人负责领回。

第七十五条  退学学生,不得申请复学。

第七十六条  学生对退学处理有异议的,可以依据学校学生申诉处理办法提出申诉。



第十三章  毕业、结业、肄业

第七十七条  具有正式学籍的学生,在毕业时应全面鉴定,其内容包括德、智、体三方面。学生经鉴定德育、体育合格,在学校规定的学习年限内,修完培养计划规定的全部课程并完成规定的实践性环节,取得规定的学分者,准予毕业,由学校发给毕业证书。

第七十八条  学籍异动学生,因培养计划调整按现所在年级培养计划毕业有困难的,可申请按照入学年级至当前毕业年级间任一年级的培养计划审核毕业。

因违纪等待学校处分或受到学校纪律处分尚未解除的学生,暂不予毕业。学生处分解除后,按前款规定达到毕业条件的,准予毕业,发给毕业证书。

对完成本专业学业同时辅修其他专业并达到该专业辅修标准者,由学校颁发辅修专业证书。

第七十九条  学生有下列情况之一,准予结业,发给结业证书:

(一)学习年限已满,修完培养计划规定的教学环节和内容,未达到毕业要求,但已取得毕业规定学分数的90%及以上(不含课外学分)的;

(二)学习年限已满,学业虽已达到毕业要求,但因违纪受到学校纪律处分尚未解除的;

(三)学习年限未满,修完培养计划规定的教学环节和内容,未达到毕业要求,但已取得规定学分数的90%及以上(不含课外学分),学生本人申请结业的。

结业的学生不再具有我校学籍,不得继续在校参加教学活动,学校不再向其颁发毕业证书。

第八十条  学满1年以上退学的学生,可发给肄业证书;未满1年的学生,发给学习证明书。

肄业的学生不再具有我校学籍,不得继续在校参加教学活动,学校不再向其颁发毕业证书。

第八十一条  开除学籍的学生,无论学习时间长短,均不能发给肄业证书,只发给学习证明书。



第十四章  学士学位授予与学业证书管理

第八十二条  本科学生学士学位的授予按学校学士学位授予相关办法执行。

第八十三条  学校严格按照国家相关规定填写、颁发学生学历证书、学位证书及其他学业证书。

学生在校期间变更姓名、出生日期、身份证号等个人信息的,应当有合理、充分的理由,并需本人填写《在校生学籍信息变更申请表》,提供有法定效力的相应证明文件,经学院初审、教务处复核后,提交教育部学籍学历信息管理平台进行变更。

第八十四条  学校按照高等教育学籍学历电子注册管理制度进行学生学籍学历电子注册,按照教育部学位中心要求进行学士学位授予信息的报送、备案工作。

学生应根据教育部及学校的要求配合做好学籍自查、个人信息核对及毕业生图像采集等电子注册的相关准备工作。

第八十五条  对违反国家招生规定取得入学资格或者学籍的,取消其学籍,不颁发学历证书、学位证书;已发的学历证书、学位证书,依法予以撤销。对以作弊、剽窃、抄袭等学术不端行为或者其他不正当手段获得学历证书、学位证书的,依法予以撤销。

被撤销的学历证书、学位证书已注册的,学校予以注销并报教育行政部门宣布无效。

第八十六条  学历证书和学位证书遗失或损坏,经本人申请,学校核实后出具相应的证明书,证明书与原证书具有同等效力。



第十五章  附则

第八十七条  学校其他有关文件与本规定不一致的,以本规定为准。

第八十八条  本规定由教务处负责解释。

第八十九条  本规定自2017年9月1日起执行。原《武汉理工大学普通全日制本科学生学籍管理规定》(校教字〔2015〕53号)同时废止。

\chapter{武汉理工大学本科学生 国际合作教育与交流管理办法}
第一章 总 则

第一条 为进一步规范我校学生参加国际合作教育与交流项目的管理,保证学生学习的连续性和有效性,特制定本办法。

第二条 本办法中的“国际合作教育与交流项目”是指由武汉理工大学与国(境)外学校共同签订协议进行联合培养或学分互认的人才培养活动。

第三条 本办法中的“出国(境)留学学生”是指参加学校国际合作教育与交流项目学习的我校在籍普通全日制本科学生。

第二章 选拔、推荐、录取程序

第四条 教务处依据学校与国(境)外学校的协议要求,按时公布国际合作教育与交流项目的相关信息。

第五条 学生申请

(一)符合国际合作与交流项目要求的学生,按自愿原则申请参加国际合作与交流项目。学生在教务处网站下载并填写《武汉理工大学学生交流学习申请表》。

(二)学生将填写完整的申请表、在校期间的成绩单以及其他项目申请材料交所在学院教学办公室。

第六条 学院审查

学生所在学院对学生的学习成绩、思想品德和奖惩情况等进行审查,签署意见后,将申请材料交教务处学籍管理办公室。

第七条 教务处推荐

教务处根据相关协议要求进行选拔,并将推荐名单送国际交流与合作处或国际教育学院。

第八条 国(境)外学校录取

学生获得学校推荐后,在国际交流与合作处或国际教育学院的指导下按国(境)外学校的要求提交相关材料,办理相关手续。国(境)外学校审核学生材料,确定最终录取名单。

第九条 学校国际合作教育本科班学生,其国际合作与交流项目的申请、选拔、推荐工作由所在学院负责,学院确定人选后将学生名单报教务处备案。

第三章 办理出国(境)留学的程序

第十条 办理出国(境)留学签证

出国(境)留学申请获准后,申请人在国际交流与合作处或国际教育学院的指导下办理出国(境)留学签证手续。

第十一条 办理学籍异动手续

取得出国(境)留学签证后,申请人需于离校前到所在学院教学办公室和教务处学籍管理办公室办理休学保留学籍手续。

第四章 学籍管理

第十二条 办理出国(境)留学手续期间的要求

出国(境)留学学生,出国(境)手续办理完毕之前,必须随原班继续学习,相应管理按学校普通全日制本科学生学籍管理规定要求执行。

第十三条 出国(境)留学期间的要求

(一)学生在国(境)外学校学习期间,相应管理按国(境)外学校的有关规定执行。

(二)学生在出国(境)前的考试不及格课程,可以在国(境)外学校选修相关课程,以充抵国内不及格课程的学分。所修课程由国(境)外学校提供课程及学分证明,学校承认其课程及学分。

(三)毕业设计(论文)要求

需要在国(境)外完成毕业设计(论文)的学生,可选择下列方式进行毕业设计(论文):

1.在国(境)外学校教师的指导下、在国(境)外学校进行毕业设计(论文)。学生需向我校提交其毕业设计(论文)原件、教师评语(有指导教师签名)、成绩单(有校方签章)等材料。经审查符合我校毕业设计(论文)要求者,学校认可其毕业设计(论文)成绩及学分。

2.参加国内毕业设计(论文)。学生向学院提出参加毕业设计(论文)的申请,学院将其编入毕业设计(论文)小组,安排毕业设计(论文)指导教师。学生应与指导设计教师保持联系,按指导教师要求提交设计(论文)成果,通过网络视频等方式进行答辩或者与学院协商返校后再答辩。

(四)学生在国(境)外学校就读期间,必须每年向原学院寄送成绩单,报告学习状况。因学生未及时报送成绩单导致的后果,由学生自行承担。

第十四条 出国(境)留学期满后的要求

学生学习期满应履行协议规定按时返校、办理复学手续,不得擅自延长或转往其它地区。在规定期限内不能按时回国者,必须及时书面告知学院在外滞留原因以及延期期限,并征得学院及教务处同意。未经批准逾期不归超过2周者,作退学处理。

第十五条 学分认定

(一)学生在国(境)外学校所取得的各类学分均需认定。

(二)学分认定分为“学年相抵”和“课程相抵”二种方式。

“学年相抵”适用于国(境)外学习时间较长且国(境)外所学专业与原国内所学专业的培养要求相同或相近的学生。学生在国(境)外学习期间按已获得审批的国外修读计划进行学习,毕业时分别按国内阶段和国(境)外阶段进行审核。

“课程相抵”适用于短期交流的学生。学生在回国后提交申请,用国外学习期间所修的课程冲抵国内培养计划中要求的课程,毕业时按国内专业的培养计划进行审核。

(三)“学年相抵”的申请及认定程序

1.学生在详细了解国(境)外学校相关专业教学计划的基础上,结合国内所学专业的培养计划,拟定国(境)外修读计划。国(境)外修读计划中,所修的学分(或学时)总量原则上应与国内应修的学分(或学时)总量相当,所修课程应与国内应修课程内容相近或相当。

2.学生将国(境)外修读计划交学院教学办公室。学生所在专业负责人和教学院长审核、同意后报教务处审批。

3.学生的国(境)外修读计划一般应在出国(境)前提交。确有困难无法在出国(境)前完成的学生,可在到达国(境)外学校的2个月内提交,逾期不再受理,学生只能在回国后进行“课程相抵”的认定。

4.学生的国(境)外修读计划一经批准,即成为其在国(境)外学习期间的指导性文件,无特殊原因,不予变更。确需变更的,学生本人应提前一学期提出变更申请,经学生所在专业负责人及教学院长签字后报教务处批准。

5.学生返校办理复学手续后2周内,提交“学年相抵”的申请及国(境)外学校出具的成绩单及课程介绍(含课程名称、类别、学分、学时、主要内容等信息),学院按其国(境)外修读计划进行审核后报教务处审批。

6.“学年相抵”的申请获得批准后,学生国(境)外学习成绩单(纸质)直接归入学生个人成绩单,国(境)外学习的课程不再录入我校的教务管理系统。

7.对于国(境)外修读计划中有、但未能取得相应学分的课程,学生可在返校后在国内补修相同或相近课程。

(四)“课程相抵”的申请及认定程序

1.学生返校办理复学手续后2周内,将“课程相抵”认定申请、国(境)外学校出具的成绩单及课程介绍(含课程名称、类别、学分、学时、主要内容等信息)交学院教学办公室,由学生所在专业负责人、主管教学工作副院长审核后报教务处审批。

2.以“课程相抵”的方式进行学分认定时,对于与我校培养计划中课程内容及教学要求相同或者相近的课程,经学院认可后,可转为我校相应课程。对于课程内容及教学要求与我校差异较大的课程,不予认可。

3.国(境)外学校的课程成绩,根据国(境)外学校出具的成绩折算标准换算为我校相应课程的百分制或等级制成绩后,录入我校的教务管理系统。

第十六条 关于学习学校及专业

(一)出国(境)学习的学生原则上应在本人申请参加项目时获批的国(境)外学校及专业学习。

(二)确有特殊原因需转往其它学校学习或者转学其它专业者,应及时向学院提出申请,学院同意并报教务处批准后方可转往其它学校学习或者转学其它专业。

(三)未经批准,擅自转到其它学校或者转学其它专业者,所获学分我校不予认可。

第十七条 毕业及学位要求

(一)对于按“学年相抵”认定学分的学生,毕业时分别按国内阶段及国(境)外阶段进行审核。学生在学校规定的学习年限内,德育、体育合格,分别修完国内及国外阶段培养计划中规定的内容并取得相应的学分者,准予毕业,颁发武汉理工大学毕业证书,符合武汉理工大学学士学位授予条件者,可获得武汉理工大学学士学位证书。

(二)对于按“课程相抵”认定学分的学生,毕业时按国内专业的培养计划进行审核。学生在学校规定的学习年限内,德育、体育合格,修完国内专业培养计划规定的全部内容并取得相应的学分者,颁发武汉理工大学毕业证书,符合武汉理工大学学士学位授予条件者,可获得武汉理工大学学士学位证书。

第十八条 成绩单管理

出国(境)留学学生的成绩单包括我校成绩单和国(境)外学校成绩单两部分,由学生所在学院负责归入学校档案和学生个人档案。

第五章 其 他

第十九条 申请出国(境)留学学生,确认参加该项目前需签署《武汉理工大学学生国(境)交流责任承诺书》。为维护校际交流协议的严肃性和延续性,学生进入国(境)外学校交换生申请程序后,原则上不可中途退出。学生在国(境)外学习期间,不得随意变更交流计划,提前或滞后回国。

第二十条 出国(境)留学学生,在国(境)外学习期间,必须按国(境)外学校要求参加医疗保险、人身安全保险等。学生自办理休学手续起,到办理复学手续止,期间发生的医疗费用由学生自行负责。学生赴国(境)外学校途中及学习结束返回学校途中发生意外及突发事件,由学生及家长负责处理,学校可应学生要求予以协助。

第二十一条 学生在国(境)外学习期间,须严格遵守所在国家(地区)的法律法规以及所在学校的规章制度。遇不可抗力导致学生在国(境)外学习期间发生的意外及突发事件,按所在国家(地区)法律法规以及所在学校的规章制度处理,我校可应学生要求与国(境)外学校就有关问题进行联络沟通或者协助处理。

第二十二条 本管理办法未尽事宜,按照学校普通全日制本科学生学籍管理规定执行。

第二十三条 本管理办法由教务处负责解释。

第二十四条 本管理办法自发布之日执行,原《武汉理工大学国际合作教育与交流学生学籍(学分)管理办法(试行)》(校教字〔2007〕42号)同时废止。

\chapter{武汉理工大学 普通全日制本科生辅修管理办法}
第一条  为了更好地培养全面发展的复合型人才,加强我校普通全日制本科生辅修管理,保证辅修质量,制定本办法。

第二条  本办法所称辅修,指在学好一个主修专业的基础上,学生根据个人的志趣和发展的需要,参加本校或武汉大学、华中科技大学、华中师范大学、中南财经政法大学、华中农业大学、中国地质大学另一个专业的辅修学习。

第三条  申请辅修学习的条件如下:

(一)具有我校学籍、且学满一年的本科学生,学有余力(原则上已修课程全部合格)、学习能力较强的,可申请参加校内辅修学习;

(二)符合第(一)项条件,必修课程平均学分绩点达到3.0及以上,且满足外校辅修申请条件的,可申请参加校外辅修学习;

(三)因违纪受到学校处分尚未解除的,不能参加辅修学习。

第四条  辅修报名与注册程序如下:

(一)教务处于每年秋季发布报名通知,公布本校和校外各学校开设的辅修专业;

(二)符合申请条件的学生按通知要求在规定时间报名,学院审核合格后报送教务处,教务处根据相关要求择优确定拟录取学生名单;

(三)拟录取学生按学校要求缴纳辅修费后,予以注册,未按要求缴纳辅修费的,视为自动放弃辅修。

第五条  辅修专业所在学院负责制定辅修培养计划报教务处批准,教务处根据开课需要从开课学院择优聘用教师任课。

第六条  辅修专业的课程学习与主修专业的课程学习同时进行,从学生进校后第四学期开始,到学生离校时结束。

第七条  辅修的学生,如主修专业的教学安排(如校外实习、其它实践环节等)与辅修专业的教学出现冲突,应服从主修的教学安排,并向辅修专业的任课教师请假,经任课教师同意后,可通过自学完成学习任务并参加该课程的考核以取得成绩和学分。

第八条  辅修专业培养计划中某门课程的要求和学分高于主修专业的同一门课程,且学生所取得的该门辅修课程成绩在80分以上者,可以向所在学院申请免修主修专业的该门课程,并附辅修专业出具的成绩证明和学分证明。如主修专业培养计划中某门课程的要求和学分高于辅修专业的同一门课程,且学生所取得的该门主修课程成绩在80分以上者,学生可申请免修辅修专业的该门课程,并附主修专业学院出具的成绩证明和学分证明。

学生辅修期间申请免修课程总计不超过2门课程。

第九条  学生修读辅修课程不及格,可参加学校安排的补考,补考不合格者可以重修。学生在修读辅修专业过程中,如辅修课程累计3门不及格(补考或重修及格除外),终止其继续修读辅修专业。

第十条  学生因病或其它特殊原因不能参加辅修课程考试的,应提前到教务处办理缓考手续。未经批准不参加考试者,以旷考论处,课程成绩记为零分。旷考的学生不能参加学校安排的补考,只能重修。缓考学生的考试随不及格课程的补考一同进行。

第十一条  学生在辅修课程考试中违纪,按照我校考试违规处理有关规定处理,不能参加学校安排的补考,只能重修。

第十二条  辅修学生按规定修完辅修专业的全部课程(25学分左右),经考核成绩合格者可发给辅修证书;修完双学位(或第二专业学士学位)规定的全部课程(50学分左右),经考核成绩合格,通过毕业论文(毕业设计)答辩,且符合开设学校辅修学位授予要求者,在获得第一学士学位的基础上由开设学校授予相应的辅修学士学位。

第十三条  学生中途中止辅修,其在辅修专业所修合格的课程,可申请以通识教育选修课程或个性选修课程的形式记入学生主修专业学籍成绩档案。

第十四条  校内辅修学生每学年按应修学分缴纳辅修费,校外辅修学生按要求缴纳辅修费。

学生中途终止辅修,不退还已交费用。

第十五条  参加校外辅修的学生,教学管理按辅修学校的要求执行。

第十六条  学生辅修课程成绩不作为评定奖学金的依据。

第十七条  本办法中未尽事宜,按我校普通全日制本科学生学籍管理相关规定执行。

第十八条  本办法由教务处负责解释。

第十九条  本办法自2017年9月1日起施行,原《武汉理工大学普通全日制本科生辅修管理办法》

		\chapter{武汉理工大学 推荐优秀应届本科毕业生 免试就读硕士研究生实施办法}
第一条 为选拔优秀人才,促进学风建设和提高本科教育质量,根据国家关于从本科毕业生中推荐优秀应届毕业生免试入学就读硕士研究生的有关精神,结合我校实际情况,制定本办法。

第二条 推荐条件

(一)申请免试就读硕士研究生的学生,必须具备下列条件:

1.具有高尚的爱国主义情操和集体主义精神,社会主义信念坚定,社会责任感强,遵纪守法,积极向上;

2.勤奋学习,刻苦钻研,成绩优秀;学术研究兴趣浓厚,有较强的创新意识、创新能力和专业能力倾向;

3.诚实守信,学风端正,无任何考试违纪和剽窃他人学术成果记录;

4.品性表现优良,无任何违法违纪受处分记录;

5.综合测评为主修专业前20%,德育为优秀;

6.学风严谨,基础扎实,学习成绩优良,主修专业必修课程平均学分绩点在3.0以上,学业水平为主修专业前20%,无不及格课程,学习过程中表现出学术思想活跃,思维敏锐,具有较强的自学能力、实践能力及分析问题和解决问题的能力;

7.取得总学分140分(五年制专业180分)以上,已完成主修专业规定的专业主干课程的学习且成绩合格,经学院初审能预期毕业;

8.达到学校规定的国家大学英语四级标准,其中对外语成绩有特殊要求的专业,由所在专业提出外语水平要求;

9.身心健康,达到国家体质健康评定标准。

(二)对具有特殊学术专长或具有突出培养潜质的学生,经三名以上本校本专业教授联名推荐,符合申请条件的1-4条以及第7、9条且主修专业已修课程平均学分绩点在2.0以上者,可不受综合排名限制,直接申请推荐为免试研究生,推荐指标单列,择优选拔。

具有特殊学术专长或具有突出培养潜质者一般指:

1.有较强的创新意识,在科研实践中有突出表现,并在学校认定的国家重要期刊上以第一作者发表过二篇以上学术论文(含二篇),经答辩证明本人的学术贡献;

2.在全国、省部级学术科技竞赛、发明创造竞赛中荣获国家三等奖(含三等奖)以上或省部级二等奖(含二等奖)以上奖励;

3.创业实践中取得突出成效的。

第三条 推荐时间和范围

(一)推荐时间:每年9月。

(二)推荐范围:纳入国家普通本科招生计划录取的本校应届本科毕业生。

第四条 推荐程序

(一)领导与组织:学校和学院分别成立免试推荐研究生领导小组和工作小组,具体负责免试研究生推荐的组织和考核工作。学校领导小组的办公机构设在教务处,由主管教学校长担任组长,学院免试推荐研究生工作小组由主管教学院长担任组长。

(二)推荐名额:全校推荐名额按当年教育部规定人数执行,各学院的推荐名额由学校免试推荐研究生领导小组确定。

(三)推荐程序:

1.公布各学院的推荐名额及部分硕士点分配名额;

2.符合条件的学生向所在学院提出申请(学生在校期间,只能申请一次);

3.学院推荐工作小组组织审查申请者资格,并广泛征求教师和学生意见后,择优提出推荐人名单,在学院公示三天;

4.教务处对各学院推荐的候选名单进行审核,经学校免试推荐研究生领导小组审批后,将推荐候选人名单张榜公布。

第五条 其它

(一)免试就读硕士研究生攻读的方向应与所学专业相近,原则上不跨专业推荐。

(二)已获推荐的免试就读硕士研究生,本科阶段不得办理自费出国和毕业派遣手续。

(三)对已获推荐的免试就读硕士研究生,凡有下列情况之一者,取消免试推荐资格

1.受到纪律处分;

2.未取得本科毕业证书或学士学位证书。

(四)对在申请免试就读硕士研究生过程中弄虚作假的学生,一经发现,即取消免试推荐资格,对已录取者取消录取资格或研究生学籍。

第六条 本办法由教务处负责解释。

第七条 本办法自发布之日起实行。

\chapter{武汉理工大学学士学位授予办法}
第一章 总则

第一条  为规范我校学士学位授予工作,依据国家相关法律法规及学校相关制度,结合我校实际情况,制定本办法。

第二条  凡我校本科毕业生,拥护中国共产党的领导,拥护社会主义制度,遵守国家法律法规、校纪校规,品德良好,学业优良,可以按照本办法的规定申请学士学位。



第二章 普通高等教育本科毕业生学士学位

第三条  符合本办法第二条的普通本科毕业生,按照培养计划要求完成全部课程的学习并取得规定学分,按规定准予毕业者可授予学士学位。符合第二学士学位(或双学位)规定的可授予第二学士学位(或双学位)。

第四条  普通本科毕业生学士学位的评定对象为普通全日制本科毕业生。各学院按学士学位的授予标准对本科毕业生进行逐个审核,并提出学院拟授予学士学位人员名单,报教务处核查后,提交校学位委员会审定。

第五条  普通全日制本科毕业生有下列情形之一,不能授予学士学位:

(一)明显反对四项基本原则;

(二)未获得毕业证书;

(三)必修课程平均学分绩点在2.0以下(有特殊规定者除外)。  



第三章 成人高等教育本科毕业生学士学位

第六条  符合本办法第二条,按照培养计划的要求完成全部课程的学习任务,按规定准予毕业,达到下列要求者,可授予成人高等教育学士学位:

(一)在校期间公共课、基础课和专业主干课平均成绩应在75分以上,普通高等教育自学考试的本科生所学课程应全部合格;

(二)参加规定的学位课程(一门外国语和一门专业基础课、两门专业课)考试,成绩合格。

第七条  外国语考试由省学位办统一组织。英语专业本科生需考其它外国语。自学考试本科生的另三门学位课程具体科目及考试方式,由省高等教育自学考试委员会与我校确定,并在考试计划或考试大纲中加以明确,学位课程考试的合格标准为:三门学位课程平均成绩应达到70分,其中每门课的成绩不低于65分。其它成人本科生的另三门课程,由教务处确定。考试组织工作,以教务处为主,成教部门协助,合格标准为:一次通过。

第八条  成人本科毕业生有本办法第五条第(一)项和第(二)项情形之一者不能授予学士学位。



第四章 附则

第九条  来华留学毕业生的学士学位授予按我校来华留学生学位授予相关制度执行。

第十条  本办法由教务处负责解释。

第十一条  本办法自2017年9月1日起施行, 原《武汉理工大学学士学位授予办法》

\part{}
\chapter{}
\end{document}