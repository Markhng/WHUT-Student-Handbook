\documentclass[UTF8,12pt,a4paper]{report}
\usepackage{ctex}
\usepackage{geometry}
\usepackage{hyperref}

\geometry{left=2cm,right=2cm,top=2.5cm,bottom=2cm}
\renewcommand{\abstractname}{说明}

\title{\textbf{武汉理工大学} \\ \textbf{学生手册}}
\author{武汉理工大学}
\date{2017}

\begin{document}
	
	\maketitle
\begin{abstract}
鉴于本人在网上没有找到方便检索的电子版2017学生手册,死活通不过入学测试,于是我就闷声大发财,自己搞了一个,供广大学生线下学习。\\
由于制作时间仓促,许多图片,公式等内容没有转换,望各位记者谅解。\\
该手册\LaTeX 源码见 \href{https://github.com/Markhng/WHUT-Student-Handbook/}{\underline{GitHub}},\textbf{\textit{欢迎前来打卡}}\\
\href{http://www.yiban.cn/t/student/showtk/}{\underline{易班测试链接}}\\
测试前,请翻开纸质版第50页,以便测试时随手查找图表。\\\\\\
\copyright 全书版权归武汉理工大学所有
\end{abstract}
	\tableofcontents
\part{总则}
\chapter{高等学校学生行为准则}
  一、志存高远,坚定信念。努力学习马克思列宁主义、毛泽东思想、邓小平理论和“三个代表”重要思想,面向世界,了解国情,确立在中国共产党领导下走社会主义道路、实现中华民族伟大复兴的共同理想和坚定信念,努力成为有理想、有道德、有文化、有纪律的社会主义新人。

  二、热爱祖国,服务人民。弘扬民族精神,维护国家利益和民族团结。不参与违反四项基本原则、影响国家统一和社会稳定的活动。培养同人民群众的深厚感情,正确处理国家、集体和个人三者利益关系,增强社会责任感,甘愿为祖国为人民奉献。

  三、勤奋学习,自强不息。追求真理,崇尚科学;刻苦钻研,严谨求实;积极实践,勇于创新;珍惜时间,学业有成。

  四、遵纪守法,弘扬正气。遵守宪法、法律法规,遵守校纪校规;正确行使权利,依法履行义务;敬廉崇洁,公道正派;敢于并善于同各种违法违纪行为作斗争。

  五、诚实守信,严于律己。履约践诺,知行统一;遵从学术规范,恪守学术道德,不作弊,不剽窃;自尊自爱,自省自律;文明使用互联网;自觉抵制黄、赌、毒等不良诱惑。

  六、明礼修身,团结友爱。弘扬传统美德,遵守社会公德,男女交往文明;关心集体,爱护公物,热心公益;尊敬师长,友爱同学,团结合作;仪表整洁,待人礼貌;豁达宽容,积极向上。

  七、勤俭节约,艰苦奋斗。热爱劳动,珍惜他人和社会劳动成果;生活俭朴,杜绝浪费;不追求超越自身和家庭实际的物质享受。

  八、热爱生活,强健体魄。积极参加文体活动,提高身体素质,保持心理健康;磨砺意志,不怕挫折,提高适应能力;增强安全意识,防止意外事故;关爱自然,爱护环境,珍惜资源。	
		\chapter{武汉理工大学学生管理规定}
第一章    总 则

第一条 为规范我校学生管理行为,维护学校正常的教育教学秩序和生活秩序,保障学生合法权益,培养德、智、体、美等方面全面发展的社会主义建设者和接班人,依据教育部《普通高等学校学生管理规定》(教育部令第41号)等有关法律、法规及《武汉理工大学章程》制定本规定。

第二条 本规定适用于我校对接受普通高等学历教育的研究生和本科学生(以下称学生)的管理。

第三条 学校坚持社会主义办学方向,坚持马克思主义的指导地位,全面贯彻国家教育方针;坚持以立德树人为根本,以理想信念教育为核心,培育和践行社会主义核心价值观,弘扬中华优秀传统文化和革命文化、社会主义先进文化,培养学生的社会责任感、创新精神和实践能力;坚持依法治校,科学管理,健全和完善管理制度,规范管理行为,将管理与育人相结合,不断提高管理和服务水平;着力培养“适应能力强、实干精神强、创新意识强”具有卓越追求与卓越能力的卓越人才。

第四条 学生应当拥护中国共产党领导,努力学习马克思列宁主义、毛泽东思想、中国特色社会主义理论体系,深入学习习近平总书记系列重要讲话精神和治国理政新理念新思想新战略,坚定中国特色社会主义道路自信、理论自信、制度自信、文化自信,树立中国特色社会主义共同理想;应当树立爱国主义思想,具有团结统一、爱好和平、勤劳勇敢、自强不息的精神;应当增强法治观念,遵守宪法、法律、法规,遵守公民道德规范,遵守学校管理制度,具有良好的道德品质和行为习惯;应当刻苦学习,勇于探索,积极实践,努力掌握现代科学文化知识和专业技能;应当积极锻炼身体,增进身心健康,提高个人修养,培养审美情趣。

第五条 学校在学生管理中,尊重和保护学生的合法权利,教育和引导学生承担应尽的义务与责任,鼓励和支持学生实行自我管理、自我服务、自我教育、自我监督。



第二章 学生的权利与义务

第六条 学生在校期间依法享有下列权利:

(一)参加学校教育教学计划安排的各项活动,使用学校提供的教育教学资源;

(二)参加社会实践、志愿服务、勤工助学、文娱体育及科技文化创新等活动,获得就业创业指导和服务;

(三)申请奖学金、助学金及助学贷款;

(四)在思想品德、学业成绩等方面获得科学、公正评价,完成学校规定学业后获得相应的学历证书、学位证书;

(五)在校内组织、参加学生团体,以适当方式参与学校管理,对学校与学生权益相关事务享有知情权、参与权、表达权和监督权;

(六)对学校给予的处理或者处分有异议,向学校、教育行政部门提出申诉,对学校、教职员工侵犯其人身权、财产权等合法权益的行为,提出申诉或者依法提起诉讼;

(七)法律、法规及学校章程规定的其他权利。

第七条 学生在校期间依法履行下列义务:

(一)遵守宪法和法律、法规;

(二)遵守学校章程和规章制度;

(三)恪守学术道德,完成规定学业;

(四)按规定缴纳学费及有关费用,履行获得贷学金及助学金的相应义务;

(五)遵守学生行为规范,尊敬师长,养成良好的思想品德和行为习惯;

(六)珍惜学校声誉,维护学校利益;

(七)法律、法规及学校章程规定的其他义务。



第三章 学籍管理

第一节   入学与注册

第八条 按国家招生规定录取的我校新生,持武汉理工大学录取通知书和学校规定的有关证件,按期到校办理入学手续。因故不能按期入学的,研究生应当向培养单位请假,本科生应当向学校招生部门请假,假期不得超过2周。未请假或者请假逾期的,除因不可抗力等正当事由以外,视为放弃入学资格。

第九条 学校在新生报到时对新生入学资格进行初步审查,审查合格的办理入学手续,予以注册学籍;审查发现新生的录取通知、考生信息等证明材料,与本人实际情况不符,或者有其他违反国家招生考试规定情形的,取消入学资格。

第十条 新生可以申请保留入学资格。保留入学资格期间不具有学籍。新生因身体、创业等原因,可在入学报到前或开学2周内向学校提出申请保留入学资格,保留入学资格年限不超过2年,应征参加中国人民解放军(含中国人民武装警察部队)的,可保留入学资格至退役后2年。

新生保留入学资格期满前应向学校申请入学,经学校审查合格后,办理入学手续。审查不合格的,取消入学资格;逾期不办理入学手续且无不可抗力等正当理由的,视为放弃入学资格。

第十一条 学生入学后,学校在3个月内按照国家招生规定进行复查。复查内容主要包括以下方面:

(一)录取手续及程序等是否合乎国家招生规定;

(二)所获得的录取资格是否真实、合乎相关规定;

(三)本人及身份证明与录取通知、考生档案等是否一致;

(四)身心健康状况是否符合报考专业或者专业类别体检要求,能否保证在校正常学习、生活;

(五)艺术、体育等特殊类型录取学生的专业水平是否符合录取要求。

复查中发现学生存在弄虚作假、徇私舞弊等情形的,确定为复查不合格,取消学籍;情节严重的,移交有关部门调查处理。

复查中发现学生身心状况不适宜在校学习,经学校指定的二级甲等以上医院诊断,需要在家休养的,可以按照第十条的规定保留入学资格。

第十二条 每学期开学时,学生应当按学校规定和要求报到和办理注册手续。不能如期注册的,应当履行暂缓注册手续。未按学校规定缴纳学费或者有其他不符合注册条件的,不予注册,无正当理由不注册或不办理暂缓注册手续逾期2周的,作退学处理(不可抗力等正当理由除外)。

家庭经济困难的学生可以申请助学贷款或者其他形式资助,办理有关手续后注册。

学校按照国家有关规定为家庭经济困难学生实行国家助学贷款、学费减(免)、困难补助、勤工助学等方式对学生进行资助,保证学生不因家庭经济困难而放弃学业。

第二节 考核与成绩记载

第十三条 学生应当参加学校教育教学计划规定的课程和各种教育教学环节(以下统称课程)的考核,考核成绩记入成绩册,并归入学籍档案。  

考核分为考试和考查两种。可采取开闭卷、论文等方式进行考核,成绩评定可采取百分制、五级制及两级制等形式,对于考核不合格的课程可进行重修或者补考。

第十四条 学校建立健全学生思想品德的考核、鉴定体系。对学生的思想品德的考核和鉴定通过本人小结、师生民主评议等形式进行。

学生体育成绩评定突出过程管理,可以根据考勤、课内教学、课外锻炼活动和体质健康等情况综合评定,学生因体残、体弱等原因无法参加体育课学习的,经本人申请,学校批准后,可参加体育保健课的学习。

第十五条 学生应参加学校规定的教学活动,学校为学生设定学期、学年所修课程或者应修学分数。对未取得学校规定学分数的学生予以学业警示。

第十六条 学校鼓励学生申请辅修校内其他专业或者选修其他专业课程;学生可以申请到其他高校进行跨校辅修专业或者修读课程,参加经申请并得到学校同意、认可的开放式网络课程学习。学生修读的课程成绩(学分),学校审核同意后,予以承认。

第十七条 学校鼓励、支持和指导学生参加创新创业、社会实践等活动,学生参加创新创业、社会实践等活动以及发表论文、获得专利授权等与专业学习、学业要求相关的经历、成果,按学校关于此类活动的学分折算办法折算为学分,计入学业成绩、并为学生建立创新创业档案。

第十八条 学校健全学生学业成绩和学籍档案管理制度,真实、完整地记载、出具学生学业成绩,对通过补考、重修获得的成绩,予以标注。

学生严重违反考核纪律或者作弊的,该课程考核成绩记为无效,并视其违纪或者作弊情节,给予相应的纪律处分。给予警告、严重警告、记过及留校察看处分的,经教育表现较好,对该课程给予重修机会。

学生因退学等情况中止学业,其在校学习期间所修课程及已获得学分,予以记录。学生重新参加入学考试、符合录取条件,再次入学的,其已获得学分经学校认定的,予以承认,学生也可申请重新学习。学校为录入其他学校的再次入校生出具相关证明。

第十九条 学生应当按时参加教育教学计划规定的活动。不能按时参加的,应当事先请假并获得批准。无故缺席的,根据学校有关规定给予批评教育,情节严重的,给予相应的纪律处分。累计缺课时数达该课程1/3学时者,不得参加该课程的考核。

第二十条 学校开展“责任、诚信、成才”三项教育,通过学年鉴定、学籍表等记录学生学业、学术、品行等方面的诚信信息,建立对失信行为的约束和惩戒机制;对有严重失信行为的,给予相应的纪律处分,对违背学术诚信的,可以对其获得学位及学术称号、荣誉等作出限制。

第三节 转专业与转学

第二十一条 学生在学习期间对其他专业有兴趣和专长或在某一专业无法正常完成学业的,可以申请转专业,学生转专业原则上应在低年级完成;以特殊招生形式录取的学生,国家、湖北省有相关规定或者录取前与学校有明确约定的,正在保留入学资格、休学、保留学籍的,应予退学处理或开除学籍处分的不得转专业。

学校根据社会对人才需求情况的发展变化,需要适当调整专业的,允许在读学生转到其他相关专业就读。

休学创业或退役后复学的学生,因自身情况需要转专业的,学校优先考虑。

第二十二条 被我校录取的学生应在我校完成学业。因患病或者有特殊困难、特别需要,无法继续在我校学习或者不适应我校学习要求的,可以申请转学。有下列情形之一,不得转学:

(一)入学未满一学期或者毕业前一年的;

(二)高考成绩低于拟转入学校相关专业同一生源地相应年份录取成绩的;

(三)由低学历层次转为高学历层次的;

(四)以定向就业招生录取的;

(五)研究生拟转入学校、专业的录取控制标准高于其所在学校、专业的;

(六)无正当转学理由或上级主管部门规定不得转学的。

学生因学校培养条件改变等非本人原因需要转学的,学校出具证明,由所在地省级教育行政部门协调转学到同层次学校。

第二十三条 学生转学由学生本人提出申请,说明理由,经所在学校和拟转入学校同意,由拟转入学校负责审核转学条件及相关证明,认为符合本校培养要求且学校有培养能力的,经学校校长办公会或者专题会议研究决定,可以转入。研究生转学还应当经拟转入专业导师同意。

跨省转学的,由转出地省级教育行政部门商转入地省级教育行政部门,按转学条件确认后办理转学手续。须转户口的由转入地省级教育行政部门将有关文件抄送转入学校所在地的公安机关。

第二十四条  学生转学一般在每年6月份或12月份申请,在符合国家政策并得到拟转入学校同意接收的条件下,学校将按规定对转学情况进行答复与公示,并在转学完成后3个月内,将转入学生名单报湖北省教育厅备案。

第四节 休学与复学

第二十五条 学生可以分阶段完成学业,除另有规定外。本科生应当在主修专业规定的学制加4年的学习年限(含休学和保留学籍,下同)内完成学业;硕士研究生学习年限不超过6年,博士研究生学习年限不超过8年。

   学生申请休学或者学校认为应当休学的,经学校批准,可以休学。休学次数原则上不超过2次,不具有完全民事行为能力的学生休学需经其监护人同意。

第二十六条 学校根据情况建立并实行灵活的学习制度。对学生休学创业,简化休学批准程序,休学创业学生的最长学习年限不超过10年。

第二十七条 新生和在校学生应征参加中国人民解放军(含中国人民武装警察部队),学校保留其入学资格或者学籍至退役后2年。

学生参加学校组织的跨校联合培养项目,在联合培养学校学习期间,学校同时为其保留学籍。

学生保留学籍期间,与其实际所在的部队、学校等组织建立管理关系。

第二十八条 休学学生应当办理手续离校并在5个工作日内离开学校。学生休学期间,学校为其保留学籍,但不享受在校学习学生待遇。因病休学学生的医疗费按国家及当地的有关规定处理。

第二十九条 学生应在休学或保留学籍期满前2周内提出复学申请,经学校复查合格,方可复学,无正当理由期满2周后仍未提出复学申请(不可抗力除外)的作退学处理。

第五节 退学

第三十条 学生有下列情形之一,学校可予以退学处理:

(一)学业成绩未达到学校要求或者在学校规定的学习年限内未完成学业的;

(二)休学、保留学籍期满,在学校规定期限内未提出复学申请或者申请复学经复查不合格的;

(三)根据学校指定医院诊断,患有疾病或者意外伤残不能继续在校学习的;

(四)未经批准连续两周未参加学校规定的教学活动的;

(五)超过学校规定期限未注册而又未履行暂缓注册手续的;

(六)学校规定的不能完成学业、应予退学的其他情形。

学生本人申请退学的,经学校审核同意后,办理退学手续。

第三十一条 退学学生,应当在10个工作日内办理退学手续离校。退学的研究生,按已有毕业学历和就业政策可以就业的,由学校报湖北省毕业生就业部门办理相关手续;在学校规定期限内没有聘用单位的,应当办理退学手续离校。

对退学学生,学校发给肄业证书或者写实性学习证明。

退学学生的档案由学校退回其家庭所在地,户口按照国家相关规定迁回原户籍地或者家庭户籍所在地。

第六节 毕业与结业

第三十二条 学生在学校规定学习年限内,修完教育教学计划规定内容,成绩合格,本科生达到学校毕业要求的,学校准予毕业,并在学生离校前发给毕业证书;研究生通过毕业/学位论文答辩,并经学位评定分委员会表决通过后,准予毕业并发给毕业证书。

符合学校学位授予条件的学生,学校为其颁发学位证书。

学校允许学生提前毕业,学生提前毕业的条件与程序按提前毕业的相关规定执行。

第三十三条 学生在学校规定学习年限内,修完教育教学计划规定内容,但未达到学校毕业要求的,学校可以准予结业,发给结业证书。

研究生在结业后,若完成毕业/学位论文撰写并经导师同意,且达到申请相应学位条件的,可提出一次毕业/学位论文答辩申请。合格后颁发的毕业证书、学位证书,毕业时间、获得学位时间按发证日期填写。

本科生在结业后不再具有我校学籍,不得继续在校参加教学活动,学校不再向其颁发毕业证书。

第七节 学业证书管理

第三十四条 学校严格按照招生时确定的办学类型和学习形式,以及学生招生录取时填报的个人信息,填写、颁发学历证书、学位证书及其他学业证书。

学生在校期间变更姓名、出生日期等证书需填写的个人信息的,应当有合理、充分的理由,并提供有法定效力的相应证明文件。学校对相关证明文件进行审查时,应请学生生源地省级教育行政部门及有关部门协助核查。

第三十五条 学校执行高等教育学籍学历电子注册管理制度,完善学籍学历信息管理办法,按相关规定及时完成学生学籍学历电子注册。

第三十六条 对完成本专业学业同时辅修其他专业并达到该专业辅修要求的学生,由学校发给辅修专业证书。

第三十七条 对违反国家招生规定取得入学资格或者学籍的,学校取消其学籍,不发给学历证书、学位证书;已发的学历证书、学位证书,学校依法予以撤销。对以作弊、剽窃、抄袭等学术不端行为或者其他不正当手段获得学历证书、学位证书的,学校依法予以撤销。

被撤销的学历证书、学位证书已注册的,学校予以注销并报教育行政部门宣布无效。

第三十八条 学历证书和学位证书遗失或者损坏,经本人申请,学校核实后出具相应的证明书。证明书与原证书具有同等效力。



第四章 校园秩序与课外活动

第三十九条 学校、学生应当共同维护校园正常秩序,保障学校环境安全、稳定,保障学生的正常学习和生活。

第四十条 学校建立和完善学生参与管理的组织形式,支持和保障学生依法、依章程参与学校管理。

第四十一条 学生应当自觉遵守公民道德规范,自觉遵守学校管理制度,创造和维护文明、整洁、优美、安全的学习和生活环境,树立安全风险防范和自我保护意识,保障自身合法权益。

第四十二条 学生不得有酗酒、打架斗殴、赌博、吸毒,传播、复制、贩卖非法书刊和音像制品等违法行为;不得参与非法传销和进行邪教、封建迷信活动;不得从事或者参与有损大学生形象、有悖社会公序良俗的活动。

学校发现学生在校内有违法行为或者严重精神疾病可能对他人造成伤害的,可以依法采取或者协助有关部门采取必要措施。

第四十三条 学校坚持教育与宗教相分离原则。任何组织和个人不得在学校进行宗教活动。

第四十四条 学校建立健全学生代表大会制度,为学生会、研究生会等开展活动提供必要条件,支持其在学生管理中发挥作用。

学生可以在校内成立、参加学生团体。学生成立团体,应当按学校有关规定提出书面申请,报学校批准并施行登记和年检制度。

学生团体应当在宪法、法律、法规和学校管理制度范围内活动,接受学校的领导和管理。学生团体邀请校外组织、人员到校举办讲座等活动,需经学校批准。

第四十五条 学校提倡并支持学生及学生团体开展有益于身心健康、成长成才的学术、科技、艺术、文娱、体育等活动。

学生进行课外活动不得影响学校正常的教育教学秩序和生活秩序。

学生参加勤工助学活动应当遵守法律、法规以及学校、用工单位的管理制度,履行勤工助学活动的有关协议。

第四十六条 学生举行大型集会、游行、示威等活动,应当按法律程序和有关规定获得批准。对未获批准的,学校依法劝阻或者制止。

第四十七条 学生应当遵守国家和学校关于网络使用的有关规定,不得登录非法网站和传播非法文字、音频、视频资料等,不得编造或者传播虚假、有害信息;不得攻击、侵入他人计算机和移动通讯网络系统。

第四十八条 学校建立健全学生住宿管理制度。学生应当遵守学校关于学生住宿管理的规定,参加文明宿舍的创建。学校鼓励和支持学生通过制定公约,实施自我管理,学生参与文明宿舍的建设情况应纳入对学生的综合评价中。



第五章 奖励与处分

第四十九条 学校对在德、智、体、美等方面全面发展或者在思想品德、学业成绩、科技创造、体育竞赛、文艺活动、志愿服务及社会实践等方面表现突出的学生,给予表彰和奖励。

第五十条 学校对学生的表彰和奖励采取授予“三好学生”称号或者其他荣誉称号、颁发奖学金等多种形式,给予相应的精神鼓励或者物质奖励。

学校接受社会捐助扩大对学生的物质奖励。

学校对学生予以表彰和奖励,以及确定推荐免试研究生、国家奖学金、公派出国留学人选等赋予学生利益的行为,建立公开、公平、公正的程序和规定,建立和完善相应的选拔、公示等制度。

第五十一条 对有违反法律法规、本规定以及学校纪律行为的学生,学校给予批评教育,并可视情节轻重,给予如下纪律处分:

(一)警告;

(二)严重警告;

(三)记过;

(四)留校察看;

(五)开除学籍。

第五十二条 学生有下列情形之一,学校可以给予开除学籍处分:

(一)违反宪法,反对四项基本原则、破坏安定团结、扰乱社会秩序的;

(二)触犯国家法律,构成刑事犯罪的;

(三)受到治安管理处罚,情节严重、性质恶劣的;

(四)代替他人或者让他人代替自己参加考试、组织作弊、使用通讯设备或其他器材作弊、向他人出售考试试题或答案牟取利益,以及其他严重作弊或扰乱考试秩序行为的;

(五)学位论文、公开发表的研究成果存在抄袭、篡改、伪造等学术不端行为,情节严重的,或者代写论文、买卖论文的;

(六)违反本规定和学校其他管理规定,严重影响学校教育教学秩序、生活秩序以及公共场所管理秩序的;

(七)侵害其他个人、组织合法权益,造成严重后果的;

(八)屡次违反学校规定受到纪律处分,经教育不改的。

第五十三条 学校对学生作出处分,出具处分决定书。处分决定书应当包括下列内容:

(一)学生的基本信息;

(二)作出处分的事实和证据;

(三)处分的种类、依据、期限;

(四)申诉的途径和期限;

(五)其他必要内容。

第五十四条 学校坚持教育与惩戒相结合,与学生违法、违纪行为的性质和过错的严重程度相适应原则给予学生处分,学校对学生的处分,做到证据充分、依据明确、定性准确、程序正当、处分适当。

第五十五条 在对学生作出处分或者其他不利决定之前,学校告知学生作出决定的事实、理由及依据,并告知学生享有陈述和申辩的权利,听取学生的陈述和申辩。

处理、处分决定以及处分告知书等,直接送达学生本人,学生拒绝签收的,可以以留置方式送达;已离校的,可以采取邮寄方式送达;难于联系的,可以利用学校网站、新闻媒体等以公告方式送达。

第五十六条 对学生作出取消入学资格、取消学籍、退学、开除学籍或者其他涉及学生重大利益的处理或者处分决定的,提交校长办公会或者校长授权的专门会议研究决定,并事先进行合法性审查。

第五十七条 除开除学籍处分以外,给予学生处分设置6到12个月期限。解除处分由学生本人申请,经评议后由学校按规定程序予以解除。解除处分后,学生获得表彰、奖励及其他权益,不再受原处分的影响。

纪律处分期限:

(一)警告,6个月;

(二)严重警告,8个月;

(三)记过,10个月;

(四)留校察看,12个月。

第五十八条 对学生的奖励、处理、处分及解除处分材料,学校真实完整地归入学校文书档案和本人档案。

被开除学籍的学生,由学校发给学习证明。学生按学校规定期限离校,档案由学校退回其家庭所在地,户口按照国家相关规定迁回原户籍地或者家庭户籍所在地。



第六章 学生申诉

第五十九条 学校成立学生申诉处理委员会,负责受理学生对处理或者处分决定不服提起的申诉。

学生申诉处理委员会由学校相关负责人、职能部门负责人、教师代表、学生代表、负责法律事务的相关机构负责人等组成,可以聘请校外法律、教育等方面专家参加。

学校制定学生申诉的具体办法,健全学生申诉处理委员会的组成与工作规则,提供必要条件,保证其能够客观、公正地履行职责。  

第六十条 学生对学校的处理或者处分决定有异议的,可以在接到学校处理或者处分决定书之日起10日内,向学校学生申诉处理委员会提出书面申诉。

第六十一条 学生申诉处理委员会对学生提出的申诉进行复查,并在接到书面申诉之日起15日内作出复查结论并告知申诉人。情况复杂不能在规定限期内作出结论的,经学校负责人批准,可延长15日。学生申诉处理委员会认为必要的,可以建议学校暂缓执行有关决定。

学生申诉处理委员会经复查,认为做出处理或者处分的事实、依据、程序等存在不当,可以作出建议撤销或变更的复查意见,要求相关职能部门予以研究,重新提交校长办公会或者专门会议作出决定。

第六十二条 学生对复查决定有异议的,在接到学校复查决定书之日起15日内,可以向湖北省教育厅提出书面申诉。

第六十三条 自处理、处分或者复查决定书送达之日起,学生在申诉期内未提出申诉的视为放弃申诉,学校不再受理其提出的申诉。

处理、处分或者复查决定书未告知学生申诉期限的,申诉期限自学生知道或者应当知道处理或者处分决定之日起计算,但最长不得超过6个月。

第六十四条 学生认为学校及其工作人员违反本规定,侵害其合法权益的;或者学校制定的规章制度与法律法规和本规定抵触的,可以向湖北省教育厅提出书面投诉。



第七章 附 则

第六十五条 对在我校接受高等学历继续教育的学生、港澳台侨学生、留学生的管理,参照本规定执行。

第六十六条  本规定自2017年9月1日起施行,学校其他学生管理类规定和办法与本规定不一致的,以本规定为准。

第六十七条  此规定由教务处、研究生院、学生工作部(处)负责解释。
	\part{学籍管理}
		\chapter{武汉理工大学 普通全日制本科学生学籍管理规定}
第一章  总则

第一条  为全面贯彻执行国家教育方针,维护学校正常的教育教学秩序和生活秩序,保障学生合法权益,不断提高教育质量,培养德、智、体、美等方面全面发展的社会主义建设者和接班人,依据《普通高等学校学生管理规定》(教育部令第41号)等法律法规,结合我校实际,制定本规定。

第二条  本规定适用于我校按照国家规定录取的接受普通高等学历教育的全日制本科学生学籍管理。



第二章 入学与注册

第三条  按国家招生规定录取的我校普通全日制新生,必须持《武汉理工大学入学录取通知书》和学校规定的有关证件,按期到校办理入学手续。因故不能按期入学者,应在报到截止日期前向学校招生部门请假,假期不得超过2周。未经请假或请假逾期的,除因不可抗力等正当事由以外,视为放弃入学资格。

第四条  学校在新生报到时,对新生入学资格进行初步审查,审查合格的办理入学手续,予以注册学籍;审查发现新生的录取通知、考生信息等证明材料,与本人实际情况不符,或者有其他违反国家招生考试规定情形的,取消入学资格。

第五条  新生因创新创业、应征入伍、身心状况不佳等原因,可在入学报到前或开学2周内向学校招生部门申请保留入学资格。保留入学资格期间不具有我校学籍。

保留入学资格的期限不超过2年,应征参加中国人民解放军(含中国人民武装警察部队)的,可保留入学资格到退役后2年。

新生保留入学资格期满前应向学校招生部门申请入学,经学校审查合格后,与应届新生一同办理入学手续。审查不合格的,取消入学资格;逾期不办理入学手续且未有因不可抗力延迟等正当理由的,视为放弃入学资格。

第六条  新生入学后,学校在3个月内按照国家规定进行复查。复查内容主要包括以下方面:

(一)录取手续及程序等是否合乎国家招生规定;

(二)所获得的录取资格是否真实、合乎相关规定;

(三)本人及身份证明与录取通知、考生档案等是否一致;

(四)身心健康状况是否符合报考专业或者专业类别体检要求,能否保证在校正常学习、生活;

(五)艺术、体育等特殊类型录取学生的专业水平是否符合录取要求。

复查中发现学生存在弄虚作假、徇私舞弊等情形的,确定为复查不合格,取消学籍;情节严重的,学校将移交有关部门调查处理。

复查中发现学生身心状况不适宜在校学习,经学校指定的二级甲等(含)以上医院诊断,需要在家休养的,学生可以按照第五条的规定保留入学资格。

第七条  每学期开学时,学生须按学校规定和要求到校报到,办理注册手续。不能如期注册的,应当履行暂缓注册手续。未按学校规定缴纳学费或者有其他不符合注册条件的,不予注册。

家庭经济困难的学生可以申请助学贷款或者其他形式资助,办理有关手续后注册。

未履行暂缓注册手续、逾期2周不注册的,作退学处理(不可抗力等正当事由除外)。



第三章  学制与学习年限

第八条  学制是指国家对各级各类学校各专业设定的课程所需的必要学习时间的规定,即学生完成专业设定的课程、学习任务一般所需要的学习年限。学生一般应在学制年限内完成学业。

第九条  学生学习年限(含休学、保留学籍)为学生主修专业规定的学制加4年。因创新创业休学的,经本人申请、学校批淮,可再延长学习年限,但最长不超过10年。  



第四章  考勤与纪律

第十条  学生上课、实习、实训、实验、课程设计及毕业设计(论文)、政治理论学习、劳动等都应该实行考勤。

教学活动的考勤由任课教师负责,其他活动由组织单位负责。

第十一条  学生应按照学校教育教学计划要求,认真学习各门课程,积极参加学校规定的各项活动。

学生上课时应遵守课堂纪律,认真听课,不得无故迟到、早退,未经教师同意不得擅自离开教室。学生未经批准,不得缺勤,违者以旷课论处。

第十二条  学生因病或其他原因不能参加学校教育教学活动时,须事先请假并获得批准。未经批准而缺席者,根据学校有关规定给予批评教育,情节严重的给予纪律处分。除因不可抗力等正当事由外,不得事后补假。请假程序及准假权限为:

(一)3天以内(含3天),报辅导员批准(在外实习期间由带队老师批准);

(二)3天以上1个月以内,由学院主管学生工作的领导批准,报学生工作部(处)备案;

(三)1个月以上2个月以内,由学院主管教学工作的领导和主管学生工作的领导批准,报教务处、学生工作部(处)备案。

学生一学期内请假累计不得超过2个月。超过的,按休学或保留学籍相关规定办理。

第十三条  学生旷课时数按实际课时计。

学生上课迟到或早退达15分钟以上者,按旷课1学时计算;迟到、早退少于15分钟的,累计3次按旷课1学时计算。

学生一学期内旷课时数累计超过该门课程教学时数1/3,取消其该课程考核资格,成绩记为零分。



第五章  课程与学分

第十四条  专业培养计划所设置的课程分为必修课程和选修课程。

必修课程指培养计划规定学生必须修读的理论课程和实践性教学环节。

选修课程指培养计划中列出的可以由学生结合个人志愿和专长选修的课程,包括全校开设的选修课程以及各专业开设的选修课程。

第十五条  学分是计算学生学习份量的单位。培养计划中设定的各类课程均规定一定的学分。

学生在校期间,必须修完培养计划中规定的课程,达到培养计划的学分要求,方能毕业。

第十六条  学生学习的努力程度,采用学年总学分数作为评价指标;学生学习的质量水平,采用平均学分绩点(GPA)作为评价指标。

课程学分绩点 = 课程绩点$\times$课程学分

课程绩点根据课程考核成绩确定,具体折算标准如下:\\

\href{http://img01.fs.yiban.cn/out/thumb_550x0/aHR0cDovL3lmczAxLmZzLnlpYmFuLmNuL3dlYi83NTg4OTE0L3VwbG9hZC8xNTA0NzY5NDc2Mzc4OTYyLnBuZw==}{image}\\

补考不合格、通过重修合格的课程,课程绩点按“1.0”折算;补考合格的课程,课程绩点按“1.0”折算;

考核(包括正考、补考或重修考核)合格后再次重修的课程,在推荐免试研究生、评优评先时,课程绩点取第一次考核合格时的折算绩点;在申请学位、对外出具成绩单时,取最高折算绩点。



第六章  课程修读

第十七条  学生根据培养计划要求及自身情况,办理相应的选课手续后进行课程修读。选课的要求及程序按照学校学生选课管理办法执行。

学生每学期选择修读的课程学分数(不含课外学分)原则上最低不得少于15学分,最高不得超过35学分。

学校对毕业设计(论文)实行准入制度。学生所修学分达到专业规定的毕业设计(论文)准入学分,方允许进入毕业设计(论文)阶段。

第十八条  进校满一年的学生,成绩优良、已修课程平均学分绩点达到3.5及以上、平均每学期取得的学分数达到25学分及以上(不含课外学分)的,可申请免听部分课程。

学生已经考核合格的课程,因个人需要再次重修的,可以申请免听。

政治理论课、军事理论课、实验课、体育课、各类实践教学环节以及各专业规定不能免听的课程不得免听。

学生每学期可以免听的课程总学分不超过8学分或课程门数不超过2门。

第十九条  符合规定的学生,可在开学后2周内提出课程免听申请,经任课教师同意,学生所在学院审核、批准后报教务处备案。免听的学生必须完成教师指定学习任务,如作业、实验等,经教师同意后方可参加该课程的期末考试。成绩合格,取得该课程学分。

第二十条  进校满一年的学生,自学能力强、成绩优良、已修课程平均学分绩点达到3.5及以上、平均每学期取得的学分数达到25学分及以上(不含课外学分)的,可以申请免修通过自学已经掌握的课程。

政治理论课、军事理论课、实验课、体育课、各类实践教学环节以及各专业规定不能免修的课程不得免修。

第二十一条  符合规定的学生,可在开学第1周提交课程免修申请,同时提供自学材料(作业、读书笔记等),经任课老师同意,学生所在学院审核、批准后,参加免修考试。考试成绩70分以上(含)的,准予免修,取得该门课程学分,成绩按免修考试实际成绩记。

免修考试由开课学院命题,试题的份量和难度应与该课程期末考试相同。考试由教务处组织,一般安排在开学后第二周进行。

退役士兵复学后,可申请免修体育课、军事训练以及军事理论课程,成绩按80分(或合格)记。

第二十二条  学生必修课程考核不合格,必须重修。

学生选修课程考核不合格,可以重修或者改选其他课程。

学生课程考核合格但成绩不够理想,也可以重修,但只能重修一次。

第二十三条  重修课程必须办理选课手续。

学生重修后取得的课程成绩,以实际成绩记,并注明“重修”字样。

第二十四条  学生可以申请选修本校或学校认可的外校其他专业的课程,参加学校认可的开放式网络课程学习。

学生修读的课程成绩(学分),经学院、教务处审核同意后,予以承认。



第七章  课程考核与成绩记载

第二十五条  学生应参加所修课程的考核,考核合格方能获得学分。

学生不得参加未选课程的考核。自行参加考核者,成绩无效。

办理了选课手续、但未按规定参加考核且未办理退选或缓考手续者,视为旷考,该门课程成绩记为零分。

第二十六条  考核分为考试和考查两种,具体考核方式由任课教师或课程负责人根据课程特点及教学要求确定,学院批准后报教务处备案。除课程结束时的考核外,任课教师要加强对学生日常学习过程的考核,如期中考试、小测验、大作业、课堂讨论、实验、论文、考勤等。

课程考核成绩,由平时成绩(如期中考试、小测验、大作业、课堂讨论、实验、论文、考勤等)和课程结束时的考试成绩综合评定。课程结束时的考试成绩占总成绩的比例,由任课教师或课程负责人确定,学院批准后报教务处备案。

体育课成绩应根据考勤、课内教学、课外锻炼活动和体质健康等情况综合评定。学生因体残、体弱无法参加体育课学习的,本人申请,经校医院诊断证明,教务处批准,可参加体育保健课的学习,成绩记载时注明“保健课”。

第二十七条  考核成绩的评定,采用百分制、五级制或两级制。60分(及格或者合格)以上即取得相应学分。考试课的成绩评定采用百分制,考查课程及所有实践环节成绩评定采用五级制(优秀、良好、中等、及格、不及格)或两级制(合格、不合格)。

百分制换算为五级制:90-100,优秀;80-89,良好;70-79,中等;60-69,及格;60分以下,不及格。

五级制换算为百分制:优秀,95;良好,85;中等,75;及格,65;不及格,50。

百分制换算为两级制:60-100,合格;60分以下,不合格。

两级制换算为百分制:合格,80;不合格,50。

第二十八条  学生应按老师的要求按时完成课程实验(包括实验报告)及作业。缺交作业或实验报告超过应交总数的1/3者,取消考核资格,成绩记为零分。

第二十九条  学生应按学校公布的考试日程安排,按时参加考试。凡未经批准不参加考试者视为旷考,成绩记为零分。

学生学习及考核过程应诚实守信,遵守学校学习和考核纪律。严重违反考核纪律或作弊的,该门课程考核成绩无效,以零分记,并按学校考试违规处理办法给予相应的纪律处分,处分材料归入学校档案及学生学籍档案。

第三十条  学生因病或其他特殊原因不能参加考核的,必须在考前提出缓考申请,经学院审核,教务处批准后方能生效。

学生因病申请缓考由校医院出具证明;因公缓考由相关单位出具证明;因考试时间冲突申请缓考由学生所在学院教学办公室核实。

缓考课程的考试原则上随该门课程的补考进行,成绩的评定、记载与正常考试相同。缓考不及格,不予补考。

第三十一条  学生第一次修读且考核不合格的课程,可补考一次。全校开设的通识教育选修课、个性课,体育课,单独设课的实验课,实践性教学环节不能补考,只能重修。

学生补考后取得的成绩,以60分记,并注明“补考”字样。

第三十二条  有下列情形之一的,取消补考资格:

(一)被取消考核资格的;

(二)无故旷考的;

(三)因违反考试纪律或考试作弊等成绩无效的。

第三十三条  课程考核成绩由任课教师于考后5~7日内上网录入。学生如对本人的考核成绩有异议,可在课程考核成绩公布后至下学期开学2周内书面提出复查申请,经教务处批准,由开课学院复查并作出结论。超过规定时限不予复查。

第三十四条  学生课程考核成绩及学分载入学生成绩表,并归入学生个人学籍档案及学校档案。

参加了多次考核的课程,归入个人学籍档案时,以最优成绩记,最优成绩如果是通过补考或重修取得的,成绩表中予以标记;归入学校档案时,历次成绩及学分均载入。

第三十五条  学生因个人需要,可以在满足毕业学分要求的情况下,申请放弃1-2门选修课程或不在培养计划要求内的课程。放弃的课程,成绩及学分不再记入学生个人学籍档案。无论何种原因,已放弃的课程不再恢复。

第三十六条  学生参加创新创业、社会实践等活动以及发表论文、获得专利授权等与专业学习、学业要求相关的经历、成果,按学校有关规定折算为相应学分,计入学业成绩。

第三十七条  学生因退学等情况中止学业,其在校学习期间所修课程及已获得学分,予以记录。学生重新参加入学考试、符合录取条件,再次入学的,其已获得学分,经学校认定,予以承认。



第八章  主修专业确认、转专业、转学

第三十八条  按学科大类入校的学生,一般在入学1-2年内进行主修专业确认。主修专业确认办法由各学院负责制定,报教务处审批后向学生公布。学院应在尊重学生意愿的基础上,加强指导。主修专业确认工作完成后,学院报教务处办理学籍异动手续。

按专业入校的学生,一般以录取专业作为主修专业。

第三十九条  学生有以下情况之一者,可以提出转专业申请:

(一)确有拟转入专业的专长和兴趣,转专业后更能发挥其专长和兴趣的。优先考虑参与创新创业、并取得一定成绩的学生转入相关主修专业的需求;

(二)因某种疾病或生理缺陷(隐瞒既往病史者除外),经校医院检查证明确实不能在原专业学习,但尚能在其他专业学习的;

(三)确有某种特殊困难,在原专业无法继续学习的;

(四)因社会对人才需求情况发生变化,学校专业发生合并、撤消等,需要调整到其他专业就读的。

大学生士兵退役后复学,因自身情况需要调整专业的优先考虑。

按学科大类入校的学生,主修专业确认前,学科大类视同专业。

第四十条  有下列情况之一,不予转专业:

(一)国家有相关规定或录取前与学校有明确约定不得转专业的;

(二)正在保留入学资格、休学、保留学籍的;

(三)应予退学处理或开除学籍处分的;

(四)无正当理由的。

第四十一条  学生转专业原则上应在低年级完成。教务处于每年四月发布转专业通知,各学院负责制定本院转专业办法,报教务处审批后向学生公布。需要转专业的学生,提出转专业申请,经双方学院考核同意后,报教务处审批。

第四十二条  学生一般应在被录取学校完成学业,有下列情况之一者,可准予转学:

(一)学生入学后发现某种疾病或生理缺陷,经学校指定校医院检查证明不能在本校学习,尚能在其他高校学习的;  

(二)学生有特殊困难、特别需要,不转学则无法继续学习的。

第四十三条  有下列情况之一,不予转学:

(一)入学未满一学期的或者毕业年级的;

(二)高考成绩低于拟转入学校相关专业同一生源地相应年份录取成绩的;

(三)由低学历层次转为高学历层次的;

(四)以定向就业招生录取的;

(五)无正当转学理由的;

(六)因其他上级主管部门规定不得转学的。

第四十四条  转学的具体程序如下:

(一)学生申请转出的:本人提出申请,说明理由,学校提供相关材料,提交转入学校审核,转入学校同意后发接收函通知本校,可以转出。跨省转学的,由学校报湖北省教育厅商转入地省级教育行政部门,按转学条件确认后办理转学手续。

(二)外校学生申请转入的:经转出学校同意后,学生向拟转入学院提出申请,说明理由,提供相关证明材料,学院党政联席会议研究认为符合我校培养要求且学校有教学能力的,提交教务处审核条件及相关证明,审核通过后,报学校专题会议研究决定,由分管校领导签署接收函,可以转入。跨省转学的,由转出地省级教育行政部门商湖北省教育厅,按转学条件确认后办理转学手续。

第四十五条  学生转学一般在每年6月份或12月份申请。学校将按规定对转学情况进行公示,并在转学完成后3个月内,将转入学生名单报湖北省教育厅备案。

第四十六条  转出学生的转学申请一经批准,必须在10个工作日内办理离校手续并离校。凡不按照规定离校的学生,产生的后果,由学生本人负责。

第九章  休学、保留学籍、复学

第四十七条  学生可以分阶段完成学业。中途中断学业的须办理休学或保留学籍手续。休学或保留学籍期满须办理复学手续后方可继续在校学习。

第四十八条  学生申请休学(不具有完全行为能力的学生,需经监护人同意)或学校认为应当休学的,经批准,可以休学。

学生有下列情况之一者,应予休学:

(一)因病经指定医院诊断,须停课治疗、病休时间在一学期内超过2个月的;

(二)一学期内因病、因事请假累计超过2个月的;

(三)因某种特殊原因,学校认为应当休学的。

第四十九条  学生有下列情形之一者,可以申请保留学籍:

(一)应征参加中国人民解放军(含中国人民武装警察部队);

(二)参加学校组织的跨校联合培养项目;

(三)学生个人联系并自费出国留学;

(四)到国际组织实习。

第五十条  休学期限一般为1年,因病重或其他原因经教务处批准,可连续休学,但累计不能超过2年;休学创业的学生可再延3年。

保留学籍的期限以实际情况为准,应征参加中国人民解放军(含中国人民武装警察部队)的,可保留学籍至退役后2年(学生服兵役的时间不计入学习年限);到国际组织实习的,保留学籍的最长期限为2年。

第五十一条  学生申请休学或保留学籍需由本人提出书面申请,并附相关证明,经学院主管领导签署意见后,报教务处批准。学院认为应当休学的,由学院提出书面报告送教务处审批。

按学校学籍管理规定应予退学或开除学籍的学生,不得办理休学或保留学籍手续。

第五十二条  休学或保留学籍期间,学校保留其学籍,学生户口可不迁出学校,但不享受在校生的待遇。休学和保留学籍期间的医疗待遇按国家及武汉市的有关规定处理。

第五十三条  学生办理休学或保留学籍,一般不得迟于学期考试周(毕业答辩)开始前2周。如遇重大疾病(附医院证明)或重要事故(附相关证明),原则上在学期考试周(毕业答辩)开始前提出申请。

第五十四条  学生休学或保留学籍一经批准,应立即停止校内一切活动,在5个工作日内办理相关手续离校。学生休学或保留学籍期间自行参加考试,成绩无效。

第五十五条  学生休学或保留学籍期满,因个人原因需要续休或继续保留学籍的,可在期满前或期满后2周内提出申请,程序同上。

第五十六条  学生休学或保留学籍期满、需要复学的,应在期满2周内(遇寒暑假,顺延)提出复学申请,经学院主管领导同意、教务处批准后复学。

休学或保留学籍学生原则上不能提前复学。

第五十七条  因病休学的学生,申请复学时必须由二级甲等以上医院诊断,证明恢复健康,并经学校医院(心理疾病经学校心理健康教育中心)复查合格,方可复学。

第五十八条  教务处根据复学学生已修课程及取得学分的情况,将其编入原专业相应年级学习。如原专业已发生变化(调整,合并或中断招生),安排到其他相近专业学习。

第五十九条  学生休学或保留学籍期间,如有严重违纪违法犯罪行为者,一经查明,取消其复学资格,作退学处理。

第六十条  学生休学或保留学籍期满,未在2周内提出复学申请的,除不可抗力等正当事由外,取消复学资格并作退学处理。



第十章  辅修

第六十一条  学生入校后,在学有余力的情况下,经本人申请、教务处批准,可辅修本校或外校其他专业。优先满足创新创业的学生对辅修专业学习的需求。

第六十二条  学生在完成主修专业学业的同时,按规定和要求修完辅修专业的全部课程,经考核成绩合格者,可发给辅修证书;符合开设学校辅修学位授予条件的,由开设学校授予辅修学位。

第六十三条  辅修学生管理按学校辅修管理相关规定执行。



第十一章  出国(境)学习

第六十四条  学生参加学校公派项目出国(境)学习,需本人申请,学院审核同意后报教务处审批。

第六十五条  学生在国(境)外学习期间,学校保留武汉理工大学学籍。学生学习期满应履行协议按时返校,不得擅自延长或转往其他地区。在规定期限内不能按时回国者,必须及时书面告知学校在外滞留原因以及延期期限,并征得学校同意。未经批准逾期不归超过2周者,除不可抗力等正当事由外,作退学处理。

第六十六条  学生需按协议要求完成学习任务,并按期将学习成绩及学习情况反馈到所在学院。期间所修课程及学分,按学校国际合作教育与交流管理规定进行认定。

第六十七条  学生因私出国(境)学习,可以申请保留学籍。保留学籍的学生,不享受在校生待遇,保留学籍期满不办理复学手续且未申请延期(或申请未批准)超过2周的,除不可抗力等正当事由外,作退学处理。

第六十八条  学校鼓励学生到国(境)外知名高校(世界排名前200名的学校)学习。学生在上述学校取得的学分,可由本人申请、学院审核、报教务处批准后认定为本校相应学分,学分认定办法参照学校国际合作教育与交流管理规定执行。



第十二章  学业警示与退学

第六十九条  学校对学生实行学业警示制度。学生取得的主修专业学年总学分数(不含课外学分)低于30学分(2017年及以后入学的学生,低于27学分)将被给予学业警示。

学院负责在每学年结束时对学生所修学分进行统计,对应受学业警示学生出具学业警示通知单,并报教务处备案。

第七十条  学生有下列情形之一,应作退学处理:

(一)在读期间累计两次受到学业警示的(主修专业平均学年总学分数(不计课外学分)达到30学分(2017年及以后入学的学生,达到27学分)除外);   

(二)本科学生在籍时间超过其最长学习年限的;

(三)经学校动员,因病该休学而不休学,且在一学期内缺课超过总学时1/3的;

(四)经二级甲等以上医院诊断,患有疾病或意外伤残无法继续在校学习的;

(五)休学或保留学籍期满,未在期满2周内提出复学申请或申请复学经复查不合格的;

(六)每学期开学时,未经批准逾期2周不报到注册的;

(七)未经批准连续2周未参加学校规定的教学活动的;

(八)公派出国(境)学生未经批准逾期不归超过2周以上的;

(九)本人申请退学,经劝说无效的。

有前款第(五)至第(八)项的情形,但有不可抗力等正当理由的除外。

第七十一条  因本办法第七十条第一款第(一)至第(八)项原因退学的学生,由学生所在学院提出书面报告,并附相关材料,报送教务处。在进行合法性审查后,提交校长办公会议研究决定。

学院提出处理报告前,应告知学生退学处理的理由和依据,并告知学生享有陈述和申辩的权利。学生有陈述和申辩要求的,以书面形式提交。

第七十二条  学生本人申请退学的,由本人填写退学申请表,家长签名、学院签署意见后报送教务处,经分管校领导审核同意后,办理退学手续。

第七十三条  对退学学生的处理,由学校出具退学通知,学生所在学院负责送交学生本人。学生本人拒绝签收的,可以以留置方式送达;已离校的,可以采取邮寄方式送达;无法取得联系的,可以利用学院网站以公告方式送达。

第七十四条  退学学生应在通知送达(或公告)之日起10个工作日内办理退学手续并离校,档案由学校退回其家庭所在地,户口按照国家规定迁回原户籍地或者家庭户籍所在地。逾期不办理离校手续的,产生的后果,由学生本人承担。

经二级甲等以上医院诊断,患有疾病或意外伤残无法继续在校学习者,由家长或抚养人负责领回。

第七十五条  退学学生,不得申请复学。

第七十六条  学生对退学处理有异议的,可以依据学校学生申诉处理办法提出申诉。



第十三章  毕业、结业、肄业

第七十七条  具有正式学籍的学生,在毕业时应全面鉴定,其内容包括德、智、体三方面。学生经鉴定德育、体育合格,在学校规定的学习年限内,修完培养计划规定的全部课程并完成规定的实践性环节,取得规定的学分者,准予毕业,由学校发给毕业证书。

第七十八条  学籍异动学生,因培养计划调整按现所在年级培养计划毕业有困难的,可申请按照入学年级至当前毕业年级间任一年级的培养计划审核毕业。

因违纪等待学校处分或受到学校纪律处分尚未解除的学生,暂不予毕业。学生处分解除后,按前款规定达到毕业条件的,准予毕业,发给毕业证书。

对完成本专业学业同时辅修其他专业并达到该专业辅修标准者,由学校颁发辅修专业证书。

第七十九条  学生有下列情况之一,准予结业,发给结业证书:

(一)学习年限已满,修完培养计划规定的教学环节和内容,未达到毕业要求,但已取得毕业规定学分数的90%及以上(不含课外学分)的;

(二)学习年限已满,学业虽已达到毕业要求,但因违纪受到学校纪律处分尚未解除的;

(三)学习年限未满,修完培养计划规定的教学环节和内容,未达到毕业要求,但已取得规定学分数的90%及以上(不含课外学分),学生本人申请结业的。

结业的学生不再具有我校学籍,不得继续在校参加教学活动,学校不再向其颁发毕业证书。

第八十条  学满1年以上退学的学生,可发给肄业证书;未满1年的学生,发给学习证明书。

肄业的学生不再具有我校学籍,不得继续在校参加教学活动,学校不再向其颁发毕业证书。

第八十一条  开除学籍的学生,无论学习时间长短,均不能发给肄业证书,只发给学习证明书。



第十四章  学士学位授予与学业证书管理

第八十二条  本科学生学士学位的授予按学校学士学位授予相关办法执行。

第八十三条  学校严格按照国家相关规定填写、颁发学生学历证书、学位证书及其他学业证书。

学生在校期间变更姓名、出生日期、身份证号等个人信息的,应当有合理、充分的理由,并需本人填写《在校生学籍信息变更申请表》,提供有法定效力的相应证明文件,经学院初审、教务处复核后,提交教育部学籍学历信息管理平台进行变更。

第八十四条  学校按照高等教育学籍学历电子注册管理制度进行学生学籍学历电子注册,按照教育部学位中心要求进行学士学位授予信息的报送、备案工作。

学生应根据教育部及学校的要求配合做好学籍自查、个人信息核对及毕业生图像采集等电子注册的相关准备工作。

第八十五条  对违反国家招生规定取得入学资格或者学籍的,取消其学籍,不颁发学历证书、学位证书;已发的学历证书、学位证书,依法予以撤销。对以作弊、剽窃、抄袭等学术不端行为或者其他不正当手段获得学历证书、学位证书的,依法予以撤销。

被撤销的学历证书、学位证书已注册的,学校予以注销并报教育行政部门宣布无效。

第八十六条  学历证书和学位证书遗失或损坏,经本人申请,学校核实后出具相应的证明书,证明书与原证书具有同等效力。



第十五章  附则

第八十七条  学校其他有关文件与本规定不一致的,以本规定为准。

第八十八条  本规定由教务处负责解释。

第八十九条  本规定自2017年9月1日起执行。原《武汉理工大学普通全日制本科学生学籍管理规定》(校教字〔2015〕53号)同时废止。

\chapter{武汉理工大学本科学生 国际合作教育与交流管理办法}
第一章 总 则

第一条 为进一步规范我校学生参加国际合作教育与交流项目的管理,保证学生学习的连续性和有效性,特制定本办法。

第二条 本办法中的“国际合作教育与交流项目”是指由武汉理工大学与国(境)外学校共同签订协议进行联合培养或学分互认的人才培养活动。

第三条 本办法中的“出国(境)留学学生”是指参加学校国际合作教育与交流项目学习的我校在籍普通全日制本科学生。

第二章 选拔、推荐、录取程序

第四条 教务处依据学校与国(境)外学校的协议要求,按时公布国际合作教育与交流项目的相关信息。

第五条 学生申请

(一)符合国际合作与交流项目要求的学生,按自愿原则申请参加国际合作与交流项目。学生在教务处网站下载并填写《武汉理工大学学生交流学习申请表》。

(二)学生将填写完整的申请表、在校期间的成绩单以及其他项目申请材料交所在学院教学办公室。

第六条 学院审查

学生所在学院对学生的学习成绩、思想品德和奖惩情况等进行审查,签署意见后,将申请材料交教务处学籍管理办公室。

第七条 教务处推荐

教务处根据相关协议要求进行选拔,并将推荐名单送国际交流与合作处或国际教育学院。

第八条 国(境)外学校录取

学生获得学校推荐后,在国际交流与合作处或国际教育学院的指导下按国(境)外学校的要求提交相关材料,办理相关手续。国(境)外学校审核学生材料,确定最终录取名单。

第九条 学校国际合作教育本科班学生,其国际合作与交流项目的申请、选拔、推荐工作由所在学院负责,学院确定人选后将学生名单报教务处备案。

第三章 办理出国(境)留学的程序

第十条 办理出国(境)留学签证

出国(境)留学申请获准后,申请人在国际交流与合作处或国际教育学院的指导下办理出国(境)留学签证手续。

第十一条 办理学籍异动手续

取得出国(境)留学签证后,申请人需于离校前到所在学院教学办公室和教务处学籍管理办公室办理休学保留学籍手续。

第四章 学籍管理

第十二条 办理出国(境)留学手续期间的要求

出国(境)留学学生,出国(境)手续办理完毕之前,必须随原班继续学习,相应管理按学校普通全日制本科学生学籍管理规定要求执行。

第十三条 出国(境)留学期间的要求

(一)学生在国(境)外学校学习期间,相应管理按国(境)外学校的有关规定执行。

(二)学生在出国(境)前的考试不及格课程,可以在国(境)外学校选修相关课程,以充抵国内不及格课程的学分。所修课程由国(境)外学校提供课程及学分证明,学校承认其课程及学分。

(三)毕业设计(论文)要求

需要在国(境)外完成毕业设计(论文)的学生,可选择下列方式进行毕业设计(论文):

1.在国(境)外学校教师的指导下、在国(境)外学校进行毕业设计(论文)。学生需向我校提交其毕业设计(论文)原件、教师评语(有指导教师签名)、成绩单(有校方签章)等材料。经审查符合我校毕业设计(论文)要求者,学校认可其毕业设计(论文)成绩及学分。

2.参加国内毕业设计(论文)。学生向学院提出参加毕业设计(论文)的申请,学院将其编入毕业设计(论文)小组,安排毕业设计(论文)指导教师。学生应与指导设计教师保持联系,按指导教师要求提交设计(论文)成果,通过网络视频等方式进行答辩或者与学院协商返校后再答辩。

(四)学生在国(境)外学校就读期间,必须每年向原学院寄送成绩单,报告学习状况。因学生未及时报送成绩单导致的后果,由学生自行承担。

第十四条 出国(境)留学期满后的要求

学生学习期满应履行协议规定按时返校、办理复学手续,不得擅自延长或转往其它地区。在规定期限内不能按时回国者,必须及时书面告知学院在外滞留原因以及延期期限,并征得学院及教务处同意。未经批准逾期不归超过2周者,作退学处理。

第十五条 学分认定

(一)学生在国(境)外学校所取得的各类学分均需认定。

(二)学分认定分为“学年相抵”和“课程相抵”二种方式。

“学年相抵”适用于国(境)外学习时间较长且国(境)外所学专业与原国内所学专业的培养要求相同或相近的学生。学生在国(境)外学习期间按已获得审批的国外修读计划进行学习,毕业时分别按国内阶段和国(境)外阶段进行审核。

“课程相抵”适用于短期交流的学生。学生在回国后提交申请,用国外学习期间所修的课程冲抵国内培养计划中要求的课程,毕业时按国内专业的培养计划进行审核。

(三)“学年相抵”的申请及认定程序

1.学生在详细了解国(境)外学校相关专业教学计划的基础上,结合国内所学专业的培养计划,拟定国(境)外修读计划。国(境)外修读计划中,所修的学分(或学时)总量原则上应与国内应修的学分(或学时)总量相当,所修课程应与国内应修课程内容相近或相当。

2.学生将国(境)外修读计划交学院教学办公室。学生所在专业负责人和教学院长审核、同意后报教务处审批。

3.学生的国(境)外修读计划一般应在出国(境)前提交。确有困难无法在出国(境)前完成的学生,可在到达国(境)外学校的2个月内提交,逾期不再受理,学生只能在回国后进行“课程相抵”的认定。

4.学生的国(境)外修读计划一经批准,即成为其在国(境)外学习期间的指导性文件,无特殊原因,不予变更。确需变更的,学生本人应提前一学期提出变更申请,经学生所在专业负责人及教学院长签字后报教务处批准。

5.学生返校办理复学手续后2周内,提交“学年相抵”的申请及国(境)外学校出具的成绩单及课程介绍(含课程名称、类别、学分、学时、主要内容等信息),学院按其国(境)外修读计划进行审核后报教务处审批。

6.“学年相抵”的申请获得批准后,学生国(境)外学习成绩单(纸质)直接归入学生个人成绩单,国(境)外学习的课程不再录入我校的教务管理系统。

7.对于国(境)外修读计划中有、但未能取得相应学分的课程,学生可在返校后在国内补修相同或相近课程。

(四)“课程相抵”的申请及认定程序

1.学生返校办理复学手续后2周内,将“课程相抵”认定申请、国(境)外学校出具的成绩单及课程介绍(含课程名称、类别、学分、学时、主要内容等信息)交学院教学办公室,由学生所在专业负责人、主管教学工作副院长审核后报教务处审批。

2.以“课程相抵”的方式进行学分认定时,对于与我校培养计划中课程内容及教学要求相同或者相近的课程,经学院认可后,可转为我校相应课程。对于课程内容及教学要求与我校差异较大的课程,不予认可。

3.国(境)外学校的课程成绩,根据国(境)外学校出具的成绩折算标准换算为我校相应课程的百分制或等级制成绩后,录入我校的教务管理系统。

第十六条 关于学习学校及专业

(一)出国(境)学习的学生原则上应在本人申请参加项目时获批的国(境)外学校及专业学习。

(二)确有特殊原因需转往其它学校学习或者转学其它专业者,应及时向学院提出申请,学院同意并报教务处批准后方可转往其它学校学习或者转学其它专业。

(三)未经批准,擅自转到其它学校或者转学其它专业者,所获学分我校不予认可。

第十七条 毕业及学位要求

(一)对于按“学年相抵”认定学分的学生,毕业时分别按国内阶段及国(境)外阶段进行审核。学生在学校规定的学习年限内,德育、体育合格,分别修完国内及国外阶段培养计划中规定的内容并取得相应的学分者,准予毕业,颁发武汉理工大学毕业证书,符合武汉理工大学学士学位授予条件者,可获得武汉理工大学学士学位证书。

(二)对于按“课程相抵”认定学分的学生,毕业时按国内专业的培养计划进行审核。学生在学校规定的学习年限内,德育、体育合格,修完国内专业培养计划规定的全部内容并取得相应的学分者,颁发武汉理工大学毕业证书,符合武汉理工大学学士学位授予条件者,可获得武汉理工大学学士学位证书。

第十八条 成绩单管理

出国(境)留学学生的成绩单包括我校成绩单和国(境)外学校成绩单两部分,由学生所在学院负责归入学校档案和学生个人档案。

第五章 其 他

第十九条 申请出国(境)留学学生,确认参加该项目前需签署《武汉理工大学学生国(境)交流责任承诺书》。为维护校际交流协议的严肃性和延续性,学生进入国(境)外学校交换生申请程序后,原则上不可中途退出。学生在国(境)外学习期间,不得随意变更交流计划,提前或滞后回国。

第二十条 出国(境)留学学生,在国(境)外学习期间,必须按国(境)外学校要求参加医疗保险、人身安全保险等。学生自办理休学手续起,到办理复学手续止,期间发生的医疗费用由学生自行负责。学生赴国(境)外学校途中及学习结束返回学校途中发生意外及突发事件,由学生及家长负责处理,学校可应学生要求予以协助。

第二十一条 学生在国(境)外学习期间,须严格遵守所在国家(地区)的法律法规以及所在学校的规章制度。遇不可抗力导致学生在国(境)外学习期间发生的意外及突发事件,按所在国家(地区)法律法规以及所在学校的规章制度处理,我校可应学生要求与国(境)外学校就有关问题进行联络沟通或者协助处理。

第二十二条 本管理办法未尽事宜,按照学校普通全日制本科学生学籍管理规定执行。

第二十三条 本管理办法由教务处负责解释。

第二十四条 本管理办法自发布之日执行,原《武汉理工大学国际合作教育与交流学生学籍(学分)管理办法(试行)》(校教字〔2007〕42号)同时废止。

\chapter{武汉理工大学 普通全日制本科生辅修管理办法}
第一条  为了更好地培养全面发展的复合型人才,加强我校普通全日制本科生辅修管理,保证辅修质量,制定本办法。

第二条  本办法所称辅修,指在学好一个主修专业的基础上,学生根据个人的志趣和发展的需要,参加本校或武汉大学、华中科技大学、华中师范大学、中南财经政法大学、华中农业大学、中国地质大学另一个专业的辅修学习。

第三条  申请辅修学习的条件如下:

(一)具有我校学籍、且学满一年的本科学生,学有余力(原则上已修课程全部合格)、学习能力较强的,可申请参加校内辅修学习;

(二)符合第(一)项条件,必修课程平均学分绩点达到3.0及以上,且满足外校辅修申请条件的,可申请参加校外辅修学习;

(三)因违纪受到学校处分尚未解除的,不能参加辅修学习。

第四条  辅修报名与注册程序如下:

(一)教务处于每年秋季发布报名通知,公布本校和校外各学校开设的辅修专业;

(二)符合申请条件的学生按通知要求在规定时间报名,学院审核合格后报送教务处,教务处根据相关要求择优确定拟录取学生名单;

(三)拟录取学生按学校要求缴纳辅修费后,予以注册,未按要求缴纳辅修费的,视为自动放弃辅修。

第五条  辅修专业所在学院负责制定辅修培养计划报教务处批准,教务处根据开课需要从开课学院择优聘用教师任课。

第六条  辅修专业的课程学习与主修专业的课程学习同时进行,从学生进校后第四学期开始,到学生离校时结束。

第七条  辅修的学生,如主修专业的教学安排(如校外实习、其它实践环节等)与辅修专业的教学出现冲突,应服从主修的教学安排,并向辅修专业的任课教师请假,经任课教师同意后,可通过自学完成学习任务并参加该课程的考核以取得成绩和学分。

第八条  辅修专业培养计划中某门课程的要求和学分高于主修专业的同一门课程,且学生所取得的该门辅修课程成绩在80分以上者,可以向所在学院申请免修主修专业的该门课程,并附辅修专业出具的成绩证明和学分证明。如主修专业培养计划中某门课程的要求和学分高于辅修专业的同一门课程,且学生所取得的该门主修课程成绩在80分以上者,学生可申请免修辅修专业的该门课程,并附主修专业学院出具的成绩证明和学分证明。

学生辅修期间申请免修课程总计不超过2门课程。

第九条  学生修读辅修课程不及格,可参加学校安排的补考,补考不合格者可以重修。学生在修读辅修专业过程中,如辅修课程累计3门不及格(补考或重修及格除外),终止其继续修读辅修专业。

第十条  学生因病或其它特殊原因不能参加辅修课程考试的,应提前到教务处办理缓考手续。未经批准不参加考试者,以旷考论处,课程成绩记为零分。旷考的学生不能参加学校安排的补考,只能重修。缓考学生的考试随不及格课程的补考一同进行。

第十一条  学生在辅修课程考试中违纪,按照我校考试违规处理有关规定处理,不能参加学校安排的补考,只能重修。

第十二条  辅修学生按规定修完辅修专业的全部课程(25学分左右),经考核成绩合格者可发给辅修证书;修完双学位(或第二专业学士学位)规定的全部课程(50学分左右),经考核成绩合格,通过毕业论文(毕业设计)答辩,且符合开设学校辅修学位授予要求者,在获得第一学士学位的基础上由开设学校授予相应的辅修学士学位。

第十三条  学生中途中止辅修,其在辅修专业所修合格的课程,可申请以通识教育选修课程或个性选修课程的形式记入学生主修专业学籍成绩档案。

第十四条  校内辅修学生每学年按应修学分缴纳辅修费,校外辅修学生按要求缴纳辅修费。

学生中途终止辅修,不退还已交费用。

第十五条  参加校外辅修的学生,教学管理按辅修学校的要求执行。

第十六条  学生辅修课程成绩不作为评定奖学金的依据。

第十七条  本办法中未尽事宜,按我校普通全日制本科学生学籍管理相关规定执行。

第十八条  本办法由教务处负责解释。

第十九条  本办法自2017年9月1日起施行,原《武汉理工大学普通全日制本科生辅修管理办法》

		\chapter{武汉理工大学 推荐优秀应届本科毕业生 免试就读硕士研究生实施办法}
第一条 为选拔优秀人才,促进学风建设和提高本科教育质量,根据国家关于从本科毕业生中推荐优秀应届毕业生免试入学就读硕士研究生的有关精神,结合我校实际情况,制定本办法。

第二条 推荐条件

(一)申请免试就读硕士研究生的学生,必须具备下列条件:

1.具有高尚的爱国主义情操和集体主义精神,社会主义信念坚定,社会责任感强,遵纪守法,积极向上;

2.勤奋学习,刻苦钻研,成绩优秀;学术研究兴趣浓厚,有较强的创新意识、创新能力和专业能力倾向;

3.诚实守信,学风端正,无任何考试违纪和剽窃他人学术成果记录;

4.品性表现优良,无任何违法违纪受处分记录;

5.综合测评为主修专业前20%,德育为优秀;

6.学风严谨,基础扎实,学习成绩优良,主修专业必修课程平均学分绩点在3.0以上,学业水平为主修专业前20%,无不及格课程,学习过程中表现出学术思想活跃,思维敏锐,具有较强的自学能力、实践能力及分析问题和解决问题的能力;

7.取得总学分140分(五年制专业180分)以上,已完成主修专业规定的专业主干课程的学习且成绩合格,经学院初审能预期毕业;

8.达到学校规定的国家大学英语四级标准,其中对外语成绩有特殊要求的专业,由所在专业提出外语水平要求;

9.身心健康,达到国家体质健康评定标准。

(二)对具有特殊学术专长或具有突出培养潜质的学生,经三名以上本校本专业教授联名推荐,符合申请条件的1-4条以及第7、9条且主修专业已修课程平均学分绩点在2.0以上者,可不受综合排名限制,直接申请推荐为免试研究生,推荐指标单列,择优选拔。

具有特殊学术专长或具有突出培养潜质者一般指:

1.有较强的创新意识,在科研实践中有突出表现,并在学校认定的国家重要期刊上以第一作者发表过二篇以上学术论文(含二篇),经答辩证明本人的学术贡献;

2.在全国、省部级学术科技竞赛、发明创造竞赛中荣获国家三等奖(含三等奖)以上或省部级二等奖(含二等奖)以上奖励;

3.创业实践中取得突出成效的。

第三条 推荐时间和范围

(一)推荐时间:每年9月。

(二)推荐范围:纳入国家普通本科招生计划录取的本校应届本科毕业生。

第四条 推荐程序

(一)领导与组织:学校和学院分别成立免试推荐研究生领导小组和工作小组,具体负责免试研究生推荐的组织和考核工作。学校领导小组的办公机构设在教务处,由主管教学校长担任组长,学院免试推荐研究生工作小组由主管教学院长担任组长。

(二)推荐名额:全校推荐名额按当年教育部规定人数执行,各学院的推荐名额由学校免试推荐研究生领导小组确定。

(三)推荐程序:

1.公布各学院的推荐名额及部分硕士点分配名额;

2.符合条件的学生向所在学院提出申请(学生在校期间,只能申请一次);

3.学院推荐工作小组组织审查申请者资格,并广泛征求教师和学生意见后,择优提出推荐人名单,在学院公示三天;

4.教务处对各学院推荐的候选名单进行审核,经学校免试推荐研究生领导小组审批后,将推荐候选人名单张榜公布。

第五条 其它

(一)免试就读硕士研究生攻读的方向应与所学专业相近,原则上不跨专业推荐。

(二)已获推荐的免试就读硕士研究生,本科阶段不得办理自费出国和毕业派遣手续。

(三)对已获推荐的免试就读硕士研究生,凡有下列情况之一者,取消免试推荐资格

1.受到纪律处分;

2.未取得本科毕业证书或学士学位证书。

(四)对在申请免试就读硕士研究生过程中弄虚作假的学生,一经发现,即取消免试推荐资格,对已录取者取消录取资格或研究生学籍。

第六条 本办法由教务处负责解释。

第七条 本办法自发布之日起实行。

\chapter{武汉理工大学学士学位授予办法}
第一章 总则

第一条  为规范我校学士学位授予工作,依据国家相关法律法规及学校相关制度,结合我校实际情况,制定本办法。

第二条  凡我校本科毕业生,拥护中国共产党的领导,拥护社会主义制度,遵守国家法律法规、校纪校规,品德良好,学业优良,可以按照本办法的规定申请学士学位。



第二章 普通高等教育本科毕业生学士学位

第三条  符合本办法第二条的普通本科毕业生,按照培养计划要求完成全部课程的学习并取得规定学分,按规定准予毕业者可授予学士学位。符合第二学士学位(或双学位)规定的可授予第二学士学位(或双学位)。

第四条  普通本科毕业生学士学位的评定对象为普通全日制本科毕业生。各学院按学士学位的授予标准对本科毕业生进行逐个审核,并提出学院拟授予学士学位人员名单,报教务处核查后,提交校学位委员会审定。

第五条  普通全日制本科毕业生有下列情形之一,不能授予学士学位:

(一)明显反对四项基本原则;

(二)未获得毕业证书;

(三)必修课程平均学分绩点在2.0以下(有特殊规定者除外)。  



第三章 成人高等教育本科毕业生学士学位

第六条  符合本办法第二条,按照培养计划的要求完成全部课程的学习任务,按规定准予毕业,达到下列要求者,可授予成人高等教育学士学位:

(一)在校期间公共课、基础课和专业主干课平均成绩应在75分以上,普通高等教育自学考试的本科生所学课程应全部合格;

(二)参加规定的学位课程(一门外国语和一门专业基础课、两门专业课)考试,成绩合格。

第七条  外国语考试由省学位办统一组织。英语专业本科生需考其它外国语。自学考试本科生的另三门学位课程具体科目及考试方式,由省高等教育自学考试委员会与我校确定,并在考试计划或考试大纲中加以明确,学位课程考试的合格标准为:三门学位课程平均成绩应达到70分,其中每门课的成绩不低于65分。其它成人本科生的另三门课程,由教务处确定。考试组织工作,以教务处为主,成教部门协助,合格标准为:一次通过。

第八条  成人本科毕业生有本办法第五条第(一)项和第(二)项情形之一者不能授予学士学位。



第四章 附则

第九条  来华留学毕业生的学士学位授予按我校来华留学生学位授予相关制度执行。

第十条  本办法由教务处负责解释。

第十一条  本办法自2017年9月1日起施行, 原《武汉理工大学学士学位授予办法》

\part{评估奖励}
\chapter{武汉理工大学 普通全日制本科学生综合素质测评办法}
为了加强学生思想教育和管理工作,促进学生综合素质的全面发展,制定本规范。

一、综合素质测评体系

(一)指标体系

\href{http://img01.fs.yiban.cn/out/thumb_550x0/aHR0cDovL3lmczAxLmZzLnlpYmFuLmNuL3dlYi83NTg4OTE0L3VwbG9hZC8xNTA0NzY5Mjc4NTE2NTc3LnBuZw==}{image}

(三)测评总分计算办法学生综合素质包括思想道德素质、科学文化素质、健康素质、拓展素质四个部分。思想道德素质、科学文化素质、健康素质总分为100分,其中思想道德素质25分、科学文化素质70分、健康素质5分;拓展素质分6分。

综合素质测评总分=思想道德素质分+科学文化素质分+健康素质分+拓展素质分-处罚分

二、评分细则

(一)思想道德素质分:该项由六项内容构成,其中,25-22(含22)分为“优”,22-19(含19)分为“良”,19-17(含17)分为“合格”,17分以下为“不合格”,具体评分标准见附件1。

(二)科学文化素质分:该项由一学年的学业水平综合评定

科学文化素质分= [(平均学分绩点-1)$\times$10+60 ]$\times$70%

1504767765245106.png

课程学分绩点=课程绩点$\times$课程学分

注:

1.主修专业课程是指本专业指导性教学计划所要求修读的课程,包含实践性教学环节、理论课程、军训、体育。

2.一学年内,如有不及格课程重修(补考)后通过的学分绩点以1计;如重修(补考)后没有通过学分绩点记为0;已获学分课程再次重修的,成绩及学分不计入综合测评体系。

3.“旷考”课程学分计入,学分绩点记为0。

4. 百分制、五级制、二级制及课程绩点的换算关系为:

\href{http://img01.fs.yiban.cn/out/thumb_550x0/aHR0cDovL3lmczAxLmZzLnlpYmFuLmNuL3dlYi83NTg4OTE0L3VwbG9hZC8xNTA0NzY5Mjc4NTE2NTc3LnBuZw==}{image}

(三)健康素质分:该项分为体质健康和课外锻炼两部分,具体评分标准见附件1。

(四)拓展素质分:该项分为人文素质和创新与实践素质两部分,计算方法为:基本分+奖励分,其中基本分为1分,奖励分5分。

1.基本分:积极参加人文素质和创新与实践素质教育活动,满足以下五项要求中的四项要求即可获得基本分1分。如不足四项,每缺一项扣0.2分。

(1)每学年至少参加一次(项)公益性志愿服务活动;

(2)每学年至少参加一次(项)年级以上的文化、体育、艺术以及有益于身心健康的校园文化活动;

(3)每学年至少参加一次院级以上人文素质或学术科技报告、讲座等;

(4)每学年至少参加一次院级以上各类学术、科技及竞赛活动;

(5)关注社会热点,广泛涉猎各领域文化知识,积极参加社会实践活动,每学年撰写不少于1500字的学习心得体会或社会实践报告。

2.奖励分:分为人文素质奖励分和创新与实践素质奖励分。

人文素质奖励分主要包括社会工作、精神文明、文体活动等方面的内容。奖励分见附件2。

创新与实践素质奖励分主要包括学生参加科技学术活动及各类学习竞赛等方面的内容。奖励分见附件3。

创业实践奖励加分细则另行制定。

(五)处罚分:综合测评时学生如有处罚及违纪处分情况,从综合测评中扣除相应分值。处罚分见附件4。

三、组织实施

(一)测评工作一般安排在新学年初进行,在综合素质测评的基础上进行三好学生、优秀学生干部评比和各类奖学金评定工作。

(二)测评工作在学校的统一布置下,由各学院具体组织实施。各学院成立“大学生综合素质测评暨评先、奖(助)学金评定”工作领导小组;各学生班成立“综合素质测评暨评先、奖(助)学金评定”工作小组(人数一般为本班级人数的1/3,也可根据班级具体情况确定适当人数,但不得低于本班级人数的1/4),其成员由班主任(或辅导员)、党支部书记(党小组组长)、班长、团支书及学生代表(应有寝室长作为学生代表)组成,班主任(或辅导员)任组长,党支部书记(党小组组长)、班长、团支部书记任副组长。

(三)拓展素质项中的基本分由学生提供相应证明材料,各班 “综合素质测评暨评先、奖(助)学金评定”工作小组做好相应活动的记录及材料的认定。

(四)实施步骤

1.学年鉴定

(1)个人小结:学生本人对上学年表现进行个人小结,自我鉴定。

(2)班集体鉴定意见:在个人小结的基础上,由班级测评小组对学生进行评议,写出班级评语。

2.测评评分

(1)自我评分:按测评标准逐项实事求是地自我评分,累积得出自评分。

(2)互评:由各学生班测评工作小组按“学生综合素质基本评价体系(见附件1)”执行。测评工作小组成员对每位同学的每项指标要素做出评价,逐项给出分数,按照去掉一个最高分,去掉一个最低分取平均值的办法,最后逐项算出互评分。

互评成绩=(本班学生测评工作小组成员评议分数之和-一个最高分-一个最低分)$\div$(本班学生测评工作小组成员数-2)。

(3)辅导员评分:由辅导员按测评标准评分。

(4)测评综合分:分值保留小数点后两位数字。

测评综合分=自评分$\times$10% + 互评分$\times$60%+ 辅导员评分$\times$30%

(5)按班级公布测评结果。

附件:1.学生综合素质基本评价体系

2.人文素质奖励量表

3.创新与实践素质奖励量表

4.处罚量表 


 

附件1

  学生综合素质基本评价体系

1504769192397166.png

1504769210142993.png

注:思想道德素质中的各项目评分标准,各班级可根据本学院、本班级的具体情况和实际工作,在不违背本条基本要求的前提下,制定更具操作性的评分细则,结合学生的实际表现进行记实考核评分。 

A:指能很好的做到“评分标准”中所述的要求;

B:指能较好的做到“评分标准”中所述的要求,个别方面有些欠缺;

C:指能基本做到“评分标准”中所述的要求;

D:指在很多方面存在缺欠,基本达不到“评分标准”中所述的要求。


 

附件2

人文素质奖励量表



一、社会工作奖励分



注:学生担任干部任期未满一学年者最高只能按“合格”档评分;学生担任几种职务,只取一项最高分;校团委、学生会干部、学生社团负责人由校团委认定并评分,学生宿舍学生管理委员会主任、副主任、各部部长由学生住宿管理科认定并评分,其余学生干部由学院评分。

二、精神文明奖励分

1504769133791621.png

注:除获国家及省级荣誉称号外,精神文明奖励分最高为2分,其中学院、年级通报表扬累计最高分为1分。同一事迹受到多级奖励,只记其中最高奖励分。“授予荣誉称号”是指获得“学生精神文明建设奖励办法”中的“授予荣誉称号”奖励者。学院认定的年级、学院级的通报表扬需报学生工作部(处)备案。

三、学生文明宿舍建设奖励分

1504769102676672.png

2. 宿舍长上浮0.1分。注:1. 凡当年受表彰的宿舍成员均获同样奖分,一学年内同获校级和院级表彰取最高分。

 

四、文体奖励分

1.体育比赛奖

1504767331423603.png

注:凡集体项目应区分主力与非主力队员,非主力队员得分减半;同一项目多次获奖,只计高分,不累计加分,主力与非主力由体育部确认。院级、年级体育比赛累计加分不超过0.5分。

2.文艺类奖

1504769075438629.png

注:凡集体项目应区分主演与非主演,非主演得分减半。同一项目多次获奖,只计高分,不累计加分,主演与非主演由校团委确认。院级、年级文艺比赛累计加分不超过0.5分。

 

附件3

创新与实践素质奖励量表



一、学习竞赛奖励分

1504767228704048.png

注:1.不同项目、不同级别竞赛获奖可累计加分,同项目、不同级别只计一项最高分;不同项目、同级别可累计加分。

2.竞赛名次折算:1、2名按一等奖计,3、4名按二等奖计,5-8名按三等奖计。

3.竞赛如设有特等奖,奖励分特等奖按一等奖计,其它等级奖励分顺延。

二、学习科研奖励分

1504767163143820.png

注:同一成果获不同奖,只记高分;不同成果获同一奖可累计加分;公开发表的论文及新闻稿件只计第一作者。通过学校规定的国家大学英语六级考试标准在通过当学年加分。
附件4

  处罚量表



一、处罚扣分

1504768902848180.png

二、违纪处分扣分 

1504767092197796.png
\chapter{武汉理工大学 普通全日制本科学生班级 建设及先进班集体评定、奖励办法}
一、学生班级工作评估指标体系

注:1.思想政治教育理论课:包括“马克思主义基本原理概论”、“毛泽东思想、邓小平理论和'三个代表'重要思想概论”、“中国近现代史纲要”、“思想道德修养与法律基础”、“形势与政策”等。

2.全班一学年平均学分绩点=∑个人平均学分绩点/班级人数。

3.全班已修学年平均学分不低于45,计算公式为:∑个人已修学年平均学分/班级人数。

4.本办法所指文科为管理学院、经济学院、文法学院、外国语学院、政治与行政学院所属学科,其余学院学科为理科。

5.艺术类、英语及小语种“国家英语四级考试合格率”标准由教务处确定。

6.航海技术、轮机工程两个专业“学年平均学分绩”考核点中一、二年级考核值下调0.2,英语四级达标率考核点的考核值下调5个百分点。

二、班级评估的实施

1.班级评估按项目由有关部门考核,有关材料由学院向相关部门取得。

2.所有申请参加先进班集体评选的班级必须在每学年初提出班级建设目标及申请报告,申请报告于9月底前报学院。申报先进班集体建设的班级应接受学校、学院和全校师生的检查和监督,学院开展学年中期检查和学年考核。

3.每学年结束后,各学院要组织所有学生班开展年度班级评估工作,各学生班评先、奖(助)学金评定工作小组,根据“学生班级工作评估指标体系”逐项自评,确定本班的先进等级(标兵、优秀),并在班级评估的基础上开展先进班集体的评选工作。

4.“学生班级工作评估指标体系”全部达到标兵标准的为标兵班集体;指标中有一项为合格,其余全部达到优秀标准为优秀班集体;各项指标均合格的班级为合格班集体;有未达标项目的班级为未达标班集体。

5.班级评估工作结束后,各学院应将所有学生班级评估结果(标兵、优秀、合格、未达标)报学生工作部。

三、表彰及奖励

1.标兵班集体、优秀班集体由学校授予称号,颁发奖状和奖金。

2.奖励标准:标兵班集体,2000元/班;优秀班集体,1000元/班。

3.先进班集体在学校计划财务处设立帐户,奖金划入班级帐户,辅导员审批用于该班班级建设。
\chapter{武汉理工大学 全日制普通本科学生文明寝室评选办法}
为了促进学生宿舍文明建设,培养学生文明行为养成,建设学习、文明、和谐、安全宿舍,特制订本办法。

第一条 评选项目

武汉理工大学全日制普通本科学生文明寝室评选项目分为:星级寝室、院文明寝室、校文明寝室、校标兵文明寝室。

第二条 评选方式与标准

(一)星级寝室

1.评选采取学院自查与学校评比相结合的方式对各楼栋各寝室文明建设和环境卫生进行检查,全学年学院及学校检查次数合计不少于八次。

2.星级寝室是年度评选校文明寝室、校标兵文明寝室的过程考核和重要依据。星级寝室分为二、三、四、五星级寝室。第一次被评为“星级寝室”称号的优秀寝室授予“二星级寝室”称号,下次检查中连续被评为优秀者则星级增加一个等级,最高为“五星级寝室”称号;被授予“星级”称号的寝室在下次检查中若未能连续评为优秀或不在学院的初评名单内,则自动将星级降低一个等级。二、三、四、五星级寝室的比例分别不超过全校本科学生宿舍总数的16%、8%、4%、2%。

3.校文明寝室、校标兵寝室的评选只能从学年末星级寝室中产生。

4.学生星级寝室检查、考核及评分标准

(1)门窗:门窗无积灰,玻璃明亮的得20分;二者较好的得10-15分;一般的得5-10分;门窗、玻璃积灰多,长期无人揩擦的得0-5分。

(2)物品放置:书籍、文具、箱子、洗漱用具、餐具、鞋子、衣服等清洁,并且放置整齐、统一、美观的得20分;较好的得10-15分;一般的得5-10分;脏、乱的得0-5分。

(3)被褥、蚊帐:被褥、蚊帐清洁,被褥叠放整齐,蚊帐张挂统一的得20分;较好的得10-15分;一般的得5-10分;被褥蚊帐较脏,被褥有一处不叠的得0-5分。

(4)蛛网积灰:天花板、墙角无蛛网,家具,灯具、箱子、地面等处无积灰、宿舍内独立卫生间干净卫生、无异味的得20分;较好的得10-15分;一般的得5-10分;蛛网多、地面脏、积灰厚、卫生间脏乱的得0-5分。

(5)寝室整体效果好,无私拉电线、网线、使用电炉、功率转换器等违章用电者得20分;否则扣20分,没收电器并通报批评;多次受到通报批评的,按学校学生违纪处罚相关规定给予相应纪律处分。

(6)星级寝室检查考评满分为100分。

优:85——100分 良:75——84分

中:60——74分 差:60分以下

(二)院文明寝室

院文明寝室的评选,由学院参照校文明寝室的评选标准,自行制定标准,组织评选。评选标准、组织过程及评选结果报学生工作部(处)备案。

(三)校文明寝室

1.评选标准

(1)全室人员积极参加学校举办的各项精神文明建设活动,自觉遵守学校及学生住宿管理的各项规章制度,无违纪违规现象,勇于制止不文明行为,能够尊重和配合管理人员的工作,同学之间能够做到团结友爱、和睦相处。

(2)全室人员学习刻苦,主修专业必修课程合格,人均学分绩点一、二年级不低于2.5分,三年级不低于3.0分。

2.校文明寝室从学年末三星、四星、五星级寝室中评选产生。

(四)校标兵文明寝室

1.评选标准

(1)全室人员积极参加学校举办的各项精神文明建设活动,自觉遵守学校及学生住宿管理的各项规章制度,无违纪违规现象,勇于制止不文明行为,能够尊重和配合管理人员的工作,同学之间能够做到团结友爱、和睦相处。

(2)全室人员学习认真刻苦,主修专业必修课合格,人均学分绩点不低于3.0分。

2.校标兵文明寝室从学年末五星级寝室中评选产生。

第三条评选程序

(一)星级寝室:

(1)学院初评:各学院组织辅导员、学生干部对本学院所有寝室进行初评,按学院寝室数的60%比例确定入选寝室,并将结果报送给学生工作部(处)。

(2)学校评比:学生工作部(处)组织相关部门、住楼辅导员、学生宿舍学生管理委员会成员和学院学生代表等对入选寝室进行评比,坚持公平、公正、公开原则,实事求是地组织检查与考核,根据检查与考核结果,在全校范围内进行公示。

(二)“校文明寝室”和“校标兵文明寝室”:学院根据本评选办法中校文明寝室和校标兵文明寝室的评选标准,在星级寝室的基础上,结合学生成绩,推荐备选寝室,并将推荐结果及相关学生成绩报送学生工作部(处)。学生工作部(处)根据学院推荐结果和文明寝室评选办法,评选出最终结果并公示。

第四条 奖励

(一)获得校级先进荣誉称号者,由学校表彰并颁发奖品;

(二)获得学年末校标兵文明寝室和校文明寝室成员,其学年综合测评按学校相应规定加分。

第五条 本办法由学生工作部(处)负责解释。

第六条 本办法自公布之日起执行。

\chapter{武汉理工大学 普通全日制本科学生优秀学生奖励办法}
第一章  总则

第一条  为了全面贯彻党和国家的教育方针,促进学生德、智、体、美等全面发展及相关工作的有效开展,根据国家有关文件精神,结合我校实际情况,特制定本办法。

第二条  本办法规定的评选项目包括:

(一)优秀学生评选,包括校三好学生标兵、校三好学生、校优秀学生干部、院三好学生、院优秀学生干部、优秀毕业生等。

(二)各类奖学金评定。

(三)学生精神文明建设奖励。

(四)学生学术、科研竞赛奖励。

第三条  学校设立学生评先、奖(助)学金评审委员会,负责学生评先及各类学生奖助学金的评审、管理、筹集和使用。

学生评先、奖(助)学金评审委员会下设办公室,挂靠学生工作部(处),负责处理全校学生评先、奖学金管理的日常工作。

各学院成立“大学生综合素质测评暨评先、奖(助)学金评定”工作领导小组,负责学生综合素质测评、评先及各类学生奖(助)学金的评审工作。



第二章  优秀学生评选

第四条  优秀学生的评选条件具体如下:

(一)校三好学生标兵。综合测评及学年学业水平排名均为班级第1名;思想道德素质为优,承担一定的班级或校院工作;学年学习平均学分绩点为4.0以上,主修专业必修课程单科成绩绩点不低于3.0,二年级(被测评学年,下同)及以上学生英语应达到学校规定的国家大学英语六级考试标准且须参加一定的科研活动;学生体质健康评定必须达良好及以上。

(二)校三好学生。综合测评排名为班级前12%, 学年学业水平排名为班级前17%;思想道德素质为优;学年学习平均学分绩点为3.0以上,主修专业必修课程单科成绩绩点不低于2.0,二年级及以上学生英语应达到学校规定的国家大学英语四级考试标准;学生体质健康评定必须达良好及以上。

(三)校优秀学生干部。担任学生干部一年以上且成绩突出,综合测评排名为班级前12%,学年学业水平排名为班级前17%;思想道德素质为优;学年学习平均学分绩点为3.0以上,主修专业必修课程单科成绩绩点不低于1.5,二年级及以上学生英语应达到学校规定的国家大学英语四级考试标准;学生体质健康评定必须达良好及以上。

(四)院三好学生。综合测评排名为班级前30%,学年学业水平排名为班级前40%;思想道德素质为优;学年学习平均学分绩点为3.0以上,主修专业必修课程单科成绩绩点不低于1.2,二年级及以上学生英语应达到学校规定的国家大学英语四级考试标准;学生体质健康评定必须达良好及以上。

(五)院优秀学生干部。担任学生干部一年以上且成绩突出,综合测评排名为班级前30%,学年学业水平排名为班级前40%;思想道德素质为优;学年学习平均学分绩点为3.0以上,主修专业必修课程单科成绩绩点不低于1.2,二年级及以上学生英语应达到学校规定的国家大学英语四级考试标准;学生体质健康评定必须达良好及以上。

(六)优秀毕业生的评选按照学校普通全日制本科学生优秀毕业生评选办法的相关规定执行。

第五条  优秀学生评选的相关程序及规定如下:

(一)学生评先工作按照学校统一部署,在各学院直接领导和辅导员、班主任的具体指导下,严格按照评选条件和要求进行,原则上综合测评排名不能并列。

(二)各班的评先工作小组根据评先条件和综合素质测评情况及学生本人的申请,确定本班的三好学生、优秀学生干部的推荐名单。

(三)各班的推荐名单确定后,将评选材料上报所在学院“大学生综合素质测评暨评先、奖(助)学金评定”工作领导小组,由学院组织有关人员进行审查,并征求有关方面的意见,报学院党政联席会初审,公示3个工作日后报学校审批。

(四)学生工作部(处)、团委、学生会等的学生干部参加所在学院的各类优秀学生的评选。

(五)评先工作以学生班级为单位,每学年评定1次,参评学生不能重复获得同类优秀学生荣誉称号。

(六)学生干部学年考核为优秀的,方可参评校优秀学生干部和院优秀学生干部。

(七)达到标兵班集体评定条件的班级评定各类优秀学生(校三好学生标兵除外)时,其学生综合测评及学年学业水平排名的要求可相应放宽10%的班级排名;达到优秀班集体评定条件的班级评定各类优秀学生(校三好学生标兵除外)时,其学生综合测评及学年学业水平排名的要求可相应放宽5%的班级排名。

第六条  凡被评为校三好学生标兵、校三好学生、校优秀学生干部均由学校授予称号,颁发证书。

凡被评为院三好学生、院优秀学生干部者,由所在学院予以表彰,并颁发证书。

第七条  优秀学生评选的其他相关规定如下:

(一)学生已修学年平均学分不低于45分,必须有1分以上拓展素质分方可参加年度评先。

(二)学生违反国家规定到网吧上网(8时-24时之外的时间)和经常旷课的学生的不得参与评先。

(三)学生干部范围按“社会工作奖励分”中的有关规定确定。

(四)凡受到党、团、行政警告以上处分者(含警告)不能参加当年评先。



第三章  奖学金评定

第八条  奖学金的名称、金额及比例如下:

(一)政府奖学金(含国家奖学金8000元/人;含国家励志奖学金5000元/人);

(二)学校奖学金(含卓越奖学金,20000元/人,约25名;学校一等奖学金,3000元/人,占参评学生2%;含学校二等奖学金,2000元/人,占参评学生4%;含学校三等奖学金,1000元/人,占参评学生12%);

(三)专项奖学金(含学术科研奖、国防奖学金、社会奖学金、专业奖学金、创业奖学金);

(四)单项奖学金。

第九条  奖学金的评选条件具体如下:

(一)国家奖学金。学年学业水平排名为班级前6%,综合测评排名为班级前6%;思想道德素质为优;主修专业必修课程单科成绩绩点不低于2.5;二年级及以上学生英语应达到学校规定的国家大学英语四级考试标准。

(二)国家励志奖学金。学年学业水平排名为班级前30%,综合测评排名为班级前30%;思想道德素质为优;主修专业必修课程单科成绩绩点不低于1.2;二年级及以上学生英语应达到学校规定的国家大学英语四级考试标准;家庭经济困难,生活俭朴。

(三)卓越奖学金。在满足学校一等奖学金评选条件下,在德、智、体、美等方面全面发展,学生认可度高,能在学生中起到榜样示范作用。

(四)学校一等奖学金。学年学业水平排名为班级(或专业)前9%,综合测评排名为班级前30%;思想道德素质为优;学年学习平均学分绩点为3.0以上,主修专业必修课程单科成绩绩点不低于2.0;二年级及以上学生英语应达到学校规定的国家大学英语四级考试标准。

(五)学校二等奖学金。学年学业水平排名为班级(或专业)前17%,综合测评排名为班级前30%;思想道德素质为优;学年学习平均学分绩点为3.0以上,主修专业必修课程单科成绩绩点不低于2.0;二年级及以上学生英语应达到学校规定的国家大学英语四级考试标准。

(六)学校三等奖学金。学年学业水平排名为班级(或专业)前30%,综合测评排名为班级前50%;思想道德素质为优;学年学习平均学分绩点为3.0以上,主修专业必修课程单科成绩绩点不低于1.2;二年级及以上学生英语应达到学校规定的国家大学英语四级考试标准。

(七)学术科研奖按学校学生学术、科研竞赛奖励办法相关规定评选。

(八)国防奖学金。为鼓励和资助志愿献身国防建设事业的学生,设立义务性奖学金。根据个人申请,经学校评审报部队审批。

(九)社会奖学金。社会奖学金是由社会出资方在我校设立的学生奖学金。该类奖学金的名称、评选等级、标准、比例及评选范围见各自的奖学金评选条例,该类奖学金的评选以每年学生评先、奖学金评定的文件为准。

(十)专业奖学金。该奖学金为我校航海类专业全日制普通本科生设立,金额为每年500元/人(其中100元用于优秀学生奖学金等其它费用)。

(十一)创业奖学金另行规定。

(十二)单项奖学金。单项奖学金由学院根据实际情况自行设定,总金额为学院参评学生总数的2%$\times$200元,个体奖项金额不超过学校三等奖学金金额。

第十条  奖学金的具体评定办法如下:

(一)学生奖学金评定工作按照学校统一部署,在各学院直接领导和辅导员、班主任的具体指导下,严格按照评定条件和要求进行。

(二)各班的奖(助)学金评定工作小组根据学生本人的申请,严格按照奖(助)学金评定条件和学年学习平均成绩情况(学年学习平均成绩优异者优先),确定本班的各类奖(助)学金获奖推荐名单。奖学金评定指标适当向校先进班集体和教学改革试验班倾斜。

(三)各班的推荐名单确定后,将评选材料上报所在学院“大学生综合素质测评暨评先、奖(助)学金评定”工作领导小组,由学院组织有关人员进行审查,并征求有关方面的意见,报学院党政联席会初审,公示3个工作日后报学校审批。

(四)奖学金评定工作每学年评定一次。国家奖学金、国家励志奖学金的评审工作,每学年根据教育部及学校有关文件执行,并将评审结果报教育部审核;优秀学生奖学金和单项奖的评审工作,每学年与三好学生、优秀学生干部的评选工作同步进行;专业奖学金奖励对象为我校指定专业的学生,经学院审核后直接报学生工作部(处)备案;社会奖学金的评审工作按各自合同或约定进行;国防奖学金经学校和部队审批后,学生本人与部队签订协议,毕业后到部队工作。

(五)国家奖学金、国家励志奖学金、学校奖学金均不能重复获得;获得专业奖学金的学生可同时获得其它各类奖学金;获得社会奖学金的学生可同时获得其它各类奖学金;获得国防奖学金的学生可同时获得其他非义务性奖学金。

第十一条  获国家或学校奖学金者,由国家或学校颁发证书,予以表彰,由学校发放奖学金。

第十二条  关于奖学金的其他相关规定如下:

(一)学生已修学年平均学分不低于45分,必须有1分以上拓展素质分方可参加奖学金评定。

(二)学生违反国家规定到网吧上网(8时-24时之外的时间)和经常旷课的学生不得参与评奖。

(三)凡受到党、团、行政警告以上处分者(含警告)不能参加当年奖学金评定;用奖学金请客酗酒等,一经发现,追回奖学金,并视情节予以处理。各学院应将取消的名单及取消原因及时送学生管理办公室备案。

(四)试点学院、卓越工程师班如班级整体学业水平较高,经批准后可适当放宽综合测评排名。

(五)卓越奖学金的评选由学生自行向学院提出申请,填写 “卓越奖学金”申请表、撰写申请书并准备相关支撑材料;经学院“大学生综合素质测评暨评先、奖(助)学金评定”工作领导小组进行材料审核后,在本单位进行公开答辩与展示后推荐到“大学生综合素质测评暨评先、奖(助)学金评定”办公室;学生工作部(处)经复审后组织由相关人员组成的评审组开展校内公开答辩与展示,学生网络投票等方式后推荐到学校学生评先、奖(助)学金评审委员会最终评定。

(六)校内一、二、三等奖学金按比例与奖学金标准的乘积和(即校内奖学金总额)下达到学院,学院在不改变各奖项金额的前提下,按学院实际向学生工作部(处)报备后分配各奖项指标,国际化示范学院奖学金指标按当年实际情况单独下达。



第四章  学生精神文明建设奖励

第十三条  学校对在精神文明建设中表现突出的学生个人坚持精神奖励为主的原则。

第十四条  学生精神文明建设奖励分为通报表扬和授予荣誉称号两类。

第十五条  学生精神文明建设奖励条件如下:

(一)积极参与校园精神文明建设并起表率作用,受到师生好评;

(二)热心公益事业,获得较好的社会声誉;

(三)当国家、集体和他人生命与财产安全受到威胁时,敢于维护正义,挺身而出;

(四)为有关部门调查案件提供有重大价值的线索或举报重大嫌疑人;

(五)拾金不昧,乐于助人;

(六)其他有益于社会的行为,为学校带来巨大荣誉。

第十六条  学生精神文明建设奖励的具体实施办法如下:

(一)校属有关单位、个人应及时发现生活中的好人好事,并向学校有关部门报告。

(二)对已经确认的好人好事,学生所在学院应及时整理材料并提出奖励意见。

(三)通报表扬分为学校通报表扬、学院通报表扬、年级通报表扬。学院通报表扬、年级通报表扬由学院奖励实施,并报学生工作部(处)备案。

(四)学校通报表扬由学生工作部(处)审批并发文。

(五)荣誉称号由校党委授予并发文。

(六)对获得通报表扬的学生以精神奖励为主,纳入综合素质测评体系中,并给予精神文明奖励加分。对授予荣誉称号者同时给予一次性奖励1000元。



第五章  学生学术、科研竞赛奖励

第十七条  电子设计、数学建模、挑战杯等竞赛获奖的奖励标准如下:

(一)获国家级特等奖:奖励10000元/项;

(二)获国家级一等奖:奖励5000元/项;

(三)获国家级二等奖:奖励3000元/项;

(四)获国家级三等奖:奖励2000元/项;

(五)获省级一等奖:奖励1000元/项;

(六)获省级二等奖:奖励600元/项;

(七)获省级三等奖:奖励300元/项。

获国家级特等奖的学生可作为免试研究生推荐,获国家级一、二、三等奖、省级一等奖的学生可以申请推荐免试研究生。

第十八条  全国大学生英语竞赛获奖的奖励标准如下:

(一)全国大学生英语竞赛特等奖:奖励500元/人;

(二)全国大学生英语演讲比赛、辩论赛一等奖:奖励300元/人;

(三)全国大学生英语演讲比赛、辩论赛二等奖:奖励200元/人;

(四)全国大学生英语演讲比赛、辩论赛三等奖:奖励100元/人。

第十九条  湖北省大学生素质教育活动及比赛(辩论赛、演讲比赛、知识竞赛、征文比赛等)获奖的奖励标准如下:

(一)获全国一等奖:奖励5000元/项;

(二)获全国二等奖:奖励2000元/项;

(三)获全国三等奖:奖励1000元/项;

(四)获省级一等奖:奖励400元/项;

(五)获省级二等奖:奖励300元/项;

(六)获省级三等奖:奖励200元/项。

第二十条  应邀参加国际学术会议,学校给予奖励500元/篇;应邀参加全国及省部级以上学术会议,奖励200元/篇。

第二十一条  撰写、编、译并出版书籍或在全国公开核心刊物上发表学术论文者,奖励500元/篇;在一般性公开刊物上发表学术论文者,奖励200元/篇。发表的学术论文均应为第一作者。

第二十二条  获授权发明专利者,奖励1000元/项;获授权实用新型专利、外观设计专利者,奖励500元/项。发明的专利必须以学生本人为第一作者。

第二十三条  省级竞赛名次第1名等同于一等奖,第2名等同于二等奖,第3名等同于三等奖,团体竞赛对每个成员均进行物质奖励,非主力队员奖金减半,邀请赛不列入奖励范围。

第二十四条  对于在学校、学院组织的学科竞赛中获奖者,由主办方进行表彰及奖励。



第六章  附则

第二十五条  奖励经费由组织实施评选的单位支出。

第二十六条  本办法由学生工作部(处)负责解释。

第二十七条  本办法自2017年9月1日起执行,原《武汉理工大学普通全日制本科学生优秀学生奖励办法》(校学字〔2015〕21号)在执行完2016-2017学年奖励后废止。
\chapter{武汉理工大学 普通全日制本科学生国家励志奖学金评审办法}
第一条 为激励家庭经济困难学生勤奋学习、努力进取,在德、智、体、美等方面得到全面发展,根据国家相关法律法规及文件精神,结合学校实际,特制定本办法。

第二条 国家励志奖学金奖励资助对象为我校普通全日制本科二年级以上(含二年级)在校生中品学兼优的家庭经济困难学生。

第三条 国家励志奖学金额度为每人每年5000元。

第四条 申请国家励志奖学金的基本条件

(一)热爱社会主义祖国,拥护中国共产党的领导;

(二)遵守宪法和法律,遵守学校规章制度;

(三)诚实守信,道德品质优良;

(四)在校期间学习成绩优秀,学年学业水平为班级前30%名,综合测评为班级前30%名,思想道德素质为优;主修专业必修课程单科成绩绩点不低于1.2,三、四年级学生英语应达到学校规定的国家大学英语四级考试标准;

(五)家庭经济困难,生活俭朴。

第五条 国家励志奖学金评审办法

(一)国家励志奖学金每学年评审一次,学校根据教育部下达的国家励志奖学金的名额和预算开展评审工作,坚持公开、公平、公正、择优的原则。

(二)国家励志奖学金的评审在学校学生综合测评的基础上进行。符合国家励志奖学金基本条件的学生可向所在学院申请国家励志奖学金。

(三)各学院大学生综合素质测评暨评先、奖(助)学金评定工作领导小组根据学生的申请,在学生班级、年级评议的基础上,统筹考虑学生受资助情况,按照同等条件下,学习成绩突出者优先的原则进行评议,确定初步名单,填写《国家励志奖学金申请表》,并张榜公示三天,无异议后报送学生工作部(处)。

(四)学生工作部(处)审核后报学校学生奖学金评审委员会审定,张榜公示五天,无异议后上报教育部。

第六条 国家励志奖学金发放及管理

(一)学校给获得国家励志奖学金的学生颁发荣誉证书,并将获奖情况记入学生学籍档案。

(二)同一学年内,获得国家励志奖学金的学生不能同时获得学校优秀学生奖学金和国家奖学金,可同时有条件申请并获得国家助学金、社会奖学金、专业奖学金、国防奖学金等奖助学金。

(三)中央主管部门的资金到位后,国家励志奖学金打入学生校园卡对应的工商银行联名卡账户。

(四)学生获国家励志奖学金后如有违法违纪行为或在评审中弄虚作假,一经核实,取消国家励志奖学获奖资格,收回已发奖励证书和奖学金,并按学校有关规定给予纪律处分。收回的奖学金进入学校奖学金专款账户,下学年使用。

第七条 本办法由学生工作(部)处负责解释。

第八条 本办法自发布之日起执行,原《武汉理工大学国家励志奖学金评审办法》(校学字〔2008〕15号)同时废止。
\chapter{武汉理工大学 普通全日制本科学生优秀毕业生评选办法}
第一条 为进一步促进校风、学风建设,鼓励学生健康发展、全面成才,结合学校实际情况,特制定本办法。

第二条 优秀毕业生评选对象为武汉理工大学普通全日制本科应届毕业生。

第三条 优秀毕业生评选条件

优秀毕业生评选对象必须符合以下必备条件和选择条件:

(一)必备条件

1.热爱祖国,拥护党的领导,遵纪守法,讲究社会公德,为人诚实守信,有良好的文明习惯和品行修养;

2.学习目的明确,勤奋刻苦,积极参加社会实践、科技文化等活动;

3.积极完成组织交办的各项任务,对所承担的社会工作认真负责,坚持原则;

4.积极参加体育锻炼,身体健康状况良好;

5.未受过学校处分。

(二)选择条件(应具备下列任意一条)

1.在校期间平均学分绩点与综合素质测评平均分值排名均为班级(专业)前15%;

2.获省部级及以上“三好学生”、“优秀学生干部”、“优秀共产党员”、“优秀共青团员”、“优秀共青团干部”且在校期间平均学分绩点3.0以上;

3.在大学生课外学术科技、文化活动中获得省部级及以上奖励且在校期间平均学分绩点2.5以上;

4.在国家核心刊物上以第一作者发表两篇以上学术论文且在校期间平均学分绩点2.5以上;

5.参加团中央“大学生志愿服务西部计划”、“中国青年志愿者支教扶贫接力计划”、“湖北省农村教师资助行动计划”等志愿服务项目且在校期间平均学分绩点2.5以上。

第四条优秀毕业生评选应先由各班初评推荐,学院综合评议,学生工作处审核后将候选名单在校园网公示一周,并报主管校领导批准。

第五条 优秀毕业生评选工作于每年4月份进行。

第六条 学校将对优秀毕业生进行表彰,授予“优秀毕业生”荣誉称号,给予一定的物质奖励。同时,组织学生填写《优秀毕业生登记表》,并将《优秀毕业生登记表》归入学生档案。

第七条 被评为“优秀毕业生”的学生如在毕业前发生违法违纪行为或出现不及格课程,将取消“优秀毕业生”称号。

第八条 本办法由学生工作部(处)负责解释。

第九条 本办法自发布之日起施行,原《武汉理工大学优秀毕业生评选办法》(校学字〔2008〕20号)同时废止。

\part{违纪处分}
\chapter{武汉理工大学普通全日制本科学生违纪处分办法}
第一章  总则

第一条  为维护学校教育教学秩序和生活秩序,建设良好的校风和学风,教育学生养成遵纪守法的优良品质,根据教育部《普通高等学校学生管理规定》(教育部令第41号),结合我校实际情况,制定本办法。

第二条  本办法适用于武汉理工大学普通全日制本科学生违法、违规、违纪行为的处理。

第三条  对有违法、违规、违纪行为的学生,必须依照规定给予批评教育直至纪律处分。

第四条  处理违反纪律的学生,坚持教育与处分相结合的原则;实施纪律处分应当按规定的程序进行,以事实为依据,与违纪行为的性质、情节和过错的严重程度相适应。



第二章  纪律处分的种类

第五条  纪律处分的种类分为:

(一)警告;

(二)严重警告;

(三)记过;

(四)留校察看;

(五)开除学籍。

第六条  除开除学籍处分以外,给予学生处分设置6-12个月期限。

纪律处分期限:

(一)警告,6个月;

(二)严重警告,8个月;

(三)记过,10个月;

(四)留校察看,12个月。

对在纪律处分期限内没有违纪行为并有悔改表现者,可以按期解除处分,解除处分由学生本人申请,经评议后由学校按规定程序予以解除。解除处分后,学生获得表彰、奖励及其他权益,不再受原处分的影响。

对有突出表现或先进事迹者,可以提前解除处分。

对处分期间无悔改表现者,由学校作出延长处分的决定,延长期限一般为6个月。

对留校察看期间有构成警告以上违纪处分行为者,给予开除学籍处分。



第三章  违纪与纪律处分

第七条  有危害国家安全的言论、行为者,视情形分别处理如下:

(一)未造成严重后果,经教育尚能改正者,给予记过或留校察看处分。经教育坚持不改者,给予开除学籍处分;

(二)造成严重后果者,给予开除学籍处分;

(三)违反国家法律、法规,破坏安定团结、扰乱社会秩序、危害国家安全者,给予开除学籍处分。

第八条  策划、组织、煽动闹事,破坏安定团结、扰乱社会秩序者,视不同情况分别给予下列纪律处分:

(一)唆使、煽动他人闹事,扰乱社会秩序者,给予记过直至开除学籍处分;

(二)组织、带头闹事,破坏安定团结者,给予记过直至开除学籍处分;

(三)散布违法言论或信息,煽动闹事或制造混乱者,给予记过直至开除学籍处分。

第九条  违反国家法律、法规,受到公安司法部门处罚者,视其处罚情况分别处理如下:

(一)违反《治安管理处罚法》被处以警告或罚款者,给予严重警告或记过处分;被处以行政拘留者,给予留校察看或开除学籍处分;

(二)构成刑事犯罪者,给予开除学籍处分。

第十条  有伤害他人、寻衅滋事、参与打架斗殴等行为者,视情节分别处理如下:

(一)虽未动手打人,但用语言或行为挑逗、侮辱、威胁他人,妨碍他人正常学习和生活,引起事端者,给予警告或严重警告处分;

(二)动手打人者,给予警告或严重警告处分;

(三)致他人轻微伤者,给予记过以上处分;

(四)致他人轻伤者,给予留校察看以上处分;

(五)致他人重伤者,给予开除学籍处分;

(六)结伙斗殴的一般参与者,给予严重警告或记过处分;为首者或动手殴打的主要责任者,给予留校察看以上处分,造成他人伤害者,给予开除学籍处分;

(七)怂恿、策划他人打架斗殴者,给予记过以下处分,后果严重者,给予留校察看以上处分;

(八)以“劝架”为名,偏袒一方,促使事态扩大或造成他人伤害者,给予严重警告或记过处分;

(九)为他人打架提供凶器者,视造成的后果,给予严重警告以上处分;

(十)先动手打人者,从重处分;持械打人者,从重处分;邀约校内、外人员寻衅滋事、打人、斗殴者,加重处分;对被打人、证人进行威胁、要挟、敲诈勒索、报复者,加重处分;

(十一)知情人故意为他人作伪证或给调查造成困难者,给予严重警告或记过处分;

(十二)凡打架斗殴,除按上述规定处理外,肇事者要赔偿受害者的经济损失并承担医疗及其它必要费用;拒绝或者不按时交纳上述费用者加重处分;肇事责任人为两人以上,由学校保卫部门根据具体情况裁定各人的赔偿份额;

(十三)本条所称“轻微伤”、“轻伤”、“重伤”均由法医鉴定部门作出结论,法医鉴定费用由肇事方承担。

第十一条  违反考试纪律者,按武汉理工大学普通全日制学生考试违规处理相关规定处理。

第十二条  未经批准擅自缺课或离校(擅自离校,连续天数中扣除规定的节假日,每天按4学时计算,实际学时超过此数时,按实际学时计算),一学期内旷课累计达20学时者,给予警告或严重警告处分;一学期内旷课累计达30学时者,给予记过处分;一学期内旷课累计达40学时者,给予留校察看以上处分。

第十三条  以偷窃、勒索、诈骗、冒领等不正当手段和途径侵占公私财物者,除追回赃款、赃物或令其赔偿损失外,视情节轻重分别处理如下:

(一)涉案价值400元以下者,给予严重警告以下处分;

(二)涉案价值400元至1000元者,给予记过处分;

(三)涉案价值1000元以上者,给予留校察看以上处分;

(四)有胁迫、威逼、诱骗、抢夺等情节者加重处分,直至开除学籍;

(五)在校期间多次作案者,按累计涉案价值适用以上条款,并根据情节加重处分,直至开除学籍; 

(六)明知赃物而购买或提供销赃窝赃条件者,给予留校察看以下处分,情节严重者,给予开除学籍处分;

(七)偷窃印章、重要公文、档案等物品者,视其情节给予记过以上处分。

第十四条  故意损坏公私财物者,除按规定赔偿外,给予下列纪律处分:

(一)损坏公私财物价值在400元以下者,给予警告处分;

(二)损坏公私财物价值在400元以上、1000元以下者,给予严重警告处分;

(三)损坏公私财物价值在1000元以上者,视情节轻重,给予记过直至开除学籍处分。

第十五条  在校园、学生生活园区内打麻将者,除收缴麻将外,给予警告处分;再犯者,视情节给予严重警告或记过处分。

第十六条  参与赌博或为赌博提供赌场、赌具、赌资者,视情节轻重分别处理如下:

(一)参与赌博者,视情节给予记过以上处分;

(二)为赌博提供赌具、赌场、赌资等条件者,视情节给予严重警告以上处分;

(三)屡次参与赌博者或赌博活动的主要组织者,视情节给予留校察看以上处分。

第十七条  有下列行为者,分别处理如下:

(一)传播、散布不健康或有害于团结的言论,或造谣、诬陷、侮辱、谩骂或威胁他人者,给予警告或严重警告处分;经批评教育不改者,给予记过处分;造成不良后果者,给予留校察看以上处分;

(二)涂写污秽语言,勾画污秽图像者,给予严重警告或记过处分;在校园、学生生活园区等公共场所观看淫秽书刊和音像制品者,给予记过或留校察看处分;传播、复制、贩卖淫秽书刊和音像制品者,给予留校察看以上处分; 

(三)在校园或学生生活园区内行为不文明,经劝阻无效者,视情节给予警告或严重警告处分;

(四)以低级下流语言、动作调戏、侮辱、挑逗异性者,或强行追逐异性谈恋爱者,视情节给予严重警告直至开除学籍处分;

(五)偷窥、偷拍或传播他人隐私者,视其情节给予留校察看以下处分,造成严重后果者,给予留校察看以上处分;

(六)有陪酒、陪舞等不良行为者给予记过以上处分;

(七)在学生宿舍男女同床者,或留宿异性或在异性宿舍留宿者,给予留校察看以上处分;

(八)发生非婚性行为者,给予留校察看以上处分;

(九)参与卖淫、嫖娼、吸毒、贩毒者,给予开除学籍处分;

(十)发送淫秽、侮辱、恐吓或其它信息,干扰他人正常生活的,视其情节给予留校察看以下处分,造成严重后果者,给予留校察看以上处分。

第十八条  有下列侵犯学校和他人正当权益行为,造成一定后果者,除赔偿损失外,视情节分别处理如下:

(一)在校园、学生生活园区内酗酒滋事,或有其他扰乱公共秩序行为者,视其情节给予留校察看以下处分;酗酒滋事造成严重后果者,给予开除学籍处分;

(二)违反学校用电管理规定,经批评教育不改者,视其情节给予留校察看以下处分;因违章用电造成严重后果者,加重处罚;

(三)在学生宿舍使用电炉、热得快、电饭煲、煤炉、酒精炉、液化气炉等燃器具者,视其情节给予留校察看以下处分;造成火警、火灾等事故者,加重处罚;

(四)在学生宿舍违章使用蜡烛等具有重大隐患的物品者,视其情节给予留校察看以下处分;造成严重后果者,加重处罚;

(五)在学生宿舍饲养宠物者,视其情节给予留校察看以下处分;屡教不改者加重处分;

(六)阻碍、干扰学校管理工作人员依校规执行公务者,视其情节给予留校察看以下处分;

(七)隐匿、毁弃或私拆他人邮件情节较轻者,视其情节给予留校察看以下处分;情节严重者,给予开除学籍处分;

(八)在学校内保存携带管制及危险物品,或向楼下乱扔东西,或乱烧杂物等妨碍公共安全者,视其情节给予留校察看以下处分;

(九)故意损毁学校党政部门发布的公告、通知标牌者,视其情节给予留校察看以下处分;

(十)学生在接受学校调查时知情不报或故意作伪证或有串供行为,妨碍学校调查者,视其情节给予留校察看以下处分;

(十一)学生在宿舍楼道治安值班不到位而造成治安责任事故者,视其情节给予留校察看以下处分;

(十二)学生未经学校同意擅自在外租房住宿或未经批准夜不归宿者,视其情节给予留校察看以下处分;

(十三)学生擅自出租、出借学生宿舍或床位,或未经批准在学生宿舍私自留宿外来人员者,视其情节给予留校察看以下处分;

(十四)寒、暑假期间不听从学校住宿安排和管理者,视其情节给予留校察看以下处分;

(十五)故意制作和传播计算机病毒,或进行网络攻击、非法入侵他人计算机或移动通讯网络系统、实施破坏性操作等危害网络安全者,视其情节给予留校察看以下处分;情节严重者,给予开除学籍处分;在网络上非法使用他人信息者,给予记过处分,情节严重者,给予留校察看以上处分;

(十六)登录非法网站和传播非法文字、音频、视频资料等,给予警告及以上处分;编造或传播虚假、有害信息给记过及以上处分。

第十九条  伪造、买卖或者使用伪造、变造的国家机关、人民团体、企业、事业单位或者其他组织的公文、证件、证明文件者,或有其他弄虚作假行为者,给予记过以上处分;造成不良影响或后果者,给予留校察看以上处分。

第二十条  剽窃、抄袭他人研究成果造成不良影响者,给予警告以上处分,情节严重、影响恶劣者,给予开除学籍处分;违反保密规定,给予记过以上处分,情节严重的,给予留校察看以上处分。

第二十一条  在校园内从事未经批准的经商活动,扰乱正常教学、生活秩序,经批评教育不改者,给予警告直至记过处分。

第二十二条  在校内或跨校建立、参加非法组织者,视情节给予留校察看以上处分。

第二十三条  在校内或跨校进行封建迷信活动不听从劝阻者,给予严重警告以上处分,情节严重或造成严重后果者,给予留校察看以上处分。

第二十四条  在校内或跨校进行宗教活动者,给予严重警告以上处分,情节严重或造成严重后果者,给予留校察看以上处分。

第二十五条  累计受到3次通报批评者,给予警告处分;如再有违纪现象者,给予严重警告以上处分。

第二十六条  已受纪律处分或在待处分期间再次违纪者,加重处分,直至给予开除学籍处分。同时有数种违纪行为者,按其数种违纪行为中应当受到的最高处分基础上加重处分。

第二十七条  其他有本办法中未具体列举的违反校园管理制度、扰乱校园正常秩序行为者,参照本办法有关条款或学校有关规定给予纪律处分,报校长办公会议审定执行。



第四章  纪律处分运用规则和程序

第二十八条  实施纪律处分必须有证据证明,以事实为根据,以本办法为准则,定性准确,处分适当,行文规范。以下各项均为有效证据:

(一)书证;

(二)物证;

(三)证人证言; 

(四)当事人的陈述;

(五)视听资料; 

(六)鉴定结论; 

(七)勘验笔录、现场笔录;

(八)其他有权部门依法作出的鉴定性结论、裁定书、判决书等。

第二十九条  违纪行为危害后果轻微,有下列情形之一者,可以从轻处分:

(一)主动承认错误,如实交待错误事实,检查认识深刻,有悔改表现的;

(二)积极主动协助调查,有重大立功表现的;

(三)由于他人胁迫或诱骗的。

第三十条  有下列情形之一者,从重处分:

(一)违纪后故意隐瞒重要情节,妨碍学校调查的;

(二)邀约校外人员来校参与违纪行为的;

(三)对检举人、证人、经办人威胁或打击报复的;

(四)情节严重,影响恶劣的。

第三十一条  受到纪律处分者,附加给予下列处理:

(一)从处分之日起,取消当学年一切评先、奖励及福利性补助和申请资格;

(二)担任学生干部者,从处分之日起取消其当学年学生干部任职资格;

(三)受处分者是党、团员的建议党、团组织给予相应纪律处分。

第三十二条  相关部门进行纪律处分的职权划分如下:

(一)除考试违规按《武汉理工大学普通全日制学生考试违规处理办法》处理外,全校在籍全日制普通本科学生违纪处分的主管部门为学生工作部(处);全校在籍全日制研究生违纪处分的主管部门为研究生院研究生管理处;其他各类学生违纪处分的主管部门为其学籍的管理部门。

(二)严重警告以下处分,学校授权违纪学生所在学院处理并作出处分决定,报相应的主管部门备案。

(三)记过处分,由违纪学生所在学院研究处理意见后报相应的主管部门审批并作出处分决定。

(四)留校察看处分由违纪学生所在学院提出处理意见,相应的主管部门审核。

(五)留校察看处分察看期满,由学生本人提出书面申请,违纪学生所在班级民主评议,学院研究处理意见后报相应的主管部门审查,并报主管校领导审批。

(六)开除学籍处分,由违纪学生所在学院提出处理意见,相应的主管部门审核,主管校领导审查,校长办公会议研究决定。

第三十三条  实施纪律处分的程序:

(一)开展调查取证。学生违纪事件发生后,学院应及时报告并进行调查或主动协助有关部门开展调查,及时收集学生违纪证据,并整理有关材料。

(二)形成拟处分意见。对事实清楚或已调查清楚的学生违纪事件,违纪学生所在学院应当在5个工作日内,根据本办法的规定作出拟给予纪律处分的意见。

(三)听取当事人的陈述和申辩。处分决定作出之前,应当告知当事人拟给予处分的有关事实、理由和依据,并听取学生的陈述和申辩。学生陈述和申辩之后,根据笔录整理成书面报告,该书面报告和笔录原件(拟受处分学生应在笔录上签字,如果拒绝签字,由主笔人写出文字说明)归入学生处分材料,作为学生处分报告的附件。

(四)作出处分决定书。对学生的处分,应当做到程序正当、证据充分、依据明确、定性准确、处分适当。给予学生的处分应按相应的公文管理办法正式行文,出具处分决定书。处分决定书应载明违纪行为的简单经过、处分的依据和学生享有的申诉权利。 

(五)处分决定书的送达。处分决定作出之后,由学生所在学院将处分决定书送达给学生本人,由学生本人签收。拒绝签收或因特殊情况不能签收的,采取留置送达、邮寄送达或公告送达,决定书生效日期以不同送达方式的法律规定为准。

(六)处分决定书的公布。处分决定视情况及时公布。

(七)处分材料的归档和管理。处分决定书及相关原始材料一律交学校主管部门,学校主管部门负责将有关材料归入学校文书档案和学生本人档案。开除学籍的处分决定书同时报湖北省教育厅备案。

第三十四条  学生对处分决定有异议的,在接到处分决定书之日起10日内,可以向学校学生申诉处理委员会提出书面申诉。申诉期间,不停止处分决定的执行。学生申诉处理委员会认为必要的,可以建议学校暂缓执行有关决定。

第三十五条  被开除学籍的学生,在处分生效后10个工作日内必须办完离校手续并离校,档案、户口退回其家庭户籍所在地,逾期不办者,其后果由学生本人承担。



第五章  附则

第三十六条  本办法所称“以上”、“以下”均包括本数。

第三十七条  本办法由学生工作部(处)负责解释。

第三十八条  本办法自2017年9月1日起施行,原《武汉理工大学普通全日制本科学生违纪处分办法》(校学字〔2015〕18号)同时废止。

\chapter{武汉理工大学 普通全日制学生考试违规处理办法}
第一章  总则

第一条  为了维护学校正常的教学秩序,建立良好的学习环境,严肃考风考纪,根据《普通高等学校学生管理规定》(教育部令第41号)等法律法规以及学校有关规章制度,制定本办法。

第二条  考试违规行为分为考试违纪、考试作弊两种。考试违纪是指不遵守考场规则和考试纪律,不服从考试工作人员安排和要求的行为;考试作弊是指违背考试公平、公正的原则,以不正当手段获得或者试图取得试题答案、考试成绩的行为。

第三条  凡武汉理工大学普通全日制在籍学生,无论参加校内或者校外考试,有违反考试管理规定和考场纪律,影响考试公平、公正的行为,其认定与处理均适用本办法。



第二章  考试违规行为的认定与处理

第一节  考试违纪行为的认定及处理

第四条  学生有下列行为之一的,由监考人员给予口头警告并予以纠正:

(一)携带考试规定以外的物品进入考场或未放在指定位置的;

(二)未按指定座位就坐的;

(三)未经允许自带空白答题纸或草稿纸的;

(四)考试开始信号发出前答题或者考试结束信号发出后继续答题的;

(五)未经允许使用计算器等具有计算功能设备的;

(六)考试中东张西望,企图偷看他人试卷、答卷(含答题卡、答题纸等,下同)的;

(七)在考试过程中交头接耳、互打暗号的;

(八)在考场或者教育考试机构禁止的范围内,喧哗、吸烟或者实施其他影响考场秩序的行为的;

(九)未经监考人员同意,在考试过程中擅自离开考场的。

第五条  学生有下列行为之一的,该门课程成绩无效,以零分记,并给予严重警告处分:

(一)因本办法第四条第(一)至(九)项中任何一种行为被口头警告后仍不改正的;

(二)未经允许带走本人试卷、答卷、草稿纸等考试材料的;

(三)用考试规定以外的笔或者纸答题或者在试卷规定以外的地方书写姓名、考号或者以其他方式在答卷上标记信息的; 

(四)有其他违反考场规则但尚未构成作弊的行为的。

第六条  学生有下列行为之一的,该门课程成绩无效,以零分记,并给予记过处分:

(一)故意损毁本人试卷、答卷等发放的考试材料的;

(二)故意扰乱考点、考场等考试工作场所秩序的;

(三)拒绝、妨碍考试工作人员履行管理职责的;

(四)威胁、侮辱、诽谤、诬陷考试工作人员与学生的;

(五)故意损坏考场设施设备的;

(六)有其他扰乱考试管理秩序行为的。

第七条  学生有下列行为之一,该门课程成绩无效,以零分记,并给予留校察看处分,情节特别恶劣的,给予开除学籍处分:

(一)带走或毁坏他人试卷、答卷等考试材料的;

(二)有本办法第六条第(一)至(六)项中任何一种行为,情节严重,致使考试无法正常进行的。

第二节  考试作弊行为的认定及处理

第八条  学生有下列行为之一的,该门课程成绩无效,以零分记,并给予记过处分:

(一)未经允许携带与考试内容相关的文字材料或者存储有与考试内容相关资料的电子设备(不具有网络通讯功能或虽然具有网络通讯功能,但未使用该功能)参加考试的(在自身或者周边物品上写、刻或印有与考试课程有关的内容,视为携带与考试内容相关的文字材料);

(二)考试中传、接与考试内容相关的纸条等物品的(学生未经监考人员允许自行借用他人物品且物品上写、刻或印有与考试课程相关的内容,开卷考试中未经监考人员允许自行借用他人的书籍、笔记、资料等,均视为传、接与考试内容有关的物品);

(三)抄袭或者协助他人抄袭试题答案、与考试内容相关的资料的(他人拿自己的试卷、答卷或草稿纸而未予拒绝或未向监考人员报告视为协助他人抄袭试题答案、与考试内容相关的资料);

(四)考试前后以送礼、请客、威胁等手段要求教师提供有关考试信息、加分或隐瞒违纪、作弊事实的;

(五)有其他与上述行为程度相当的作弊行为的。

第九条  学生有下列行为之一的,该门课程成绩无效,以零分记,并给予留校察看处分, 情节严重的,给予开除学籍处分:

(一) 携带具有发射功能或网络通讯功能(包括短信功能)的通信设备,且使用相应发射功能或网络通讯功能作弊的;

(二)考试过程中传、接或交换试卷、答卷的;

(三)抢夺、窃取他人试卷、答卷或者胁迫他人为自己抄袭提供方便的;

(四)有其他与上述行为程度相当的作弊行为的。

第十条  学生有下列行为之一的,该门课程成绩无效,以零分记,并给予开除学籍处分。

(一)组织作弊的;

(二)为组织作弊者提供作弊器材或者其他帮助的;

(三)偷盗试卷的;

(四)为实施考试作弊行为,向他人非法出售或提供考试试题或答案的;

(五)请他人代考的;

(六)代他人考试的;

(七)在答卷上填写他人姓名、考号等信息的。

第三节  多次考试违规的处理

第十一条  学生因考试违纪、作弊已经受到学校两次以上纪律处分(不论处分是否解除),第三次出现本办法第六至九条中任一考试违纪、作弊行为的,给予开除学籍处分。



第三章  学生考试违规的处理原则及程序

第十二条  对学生考试违规行为的认定与处理,应以事实为依据,做到证据充分、依据明确、定性准确、程序正当、处分适当。

第十三条  学生考试违规的确认程序如下:

(一)监考人员发现学生实施了考试违规行为,应收回试卷及物证,终止学生继续参加考试(属于口头警告范围的除外),告知学生考试违规的事实,同时在《考场记录》及《考试违规认定单》中如实记录当事人的姓名、学号及主要情节。

(二)《考试违规认定单》作为认定学生考试违规的依据应由学生本人签字。学生拒不签字的,由2名监考人员在《考试违规认定单》上说明相关情况并签名。

(三)巡考人员发现学生实施了考试违规行为,应当即向考场监考人员说明情况,由监考人员按上述办法处理,巡考人员在《考场记录》上签名。

(四)考试结束后,监考人员应将《考场记录》、《考试违规认定单》、试卷和物证一并在该门课程考试结束后的当天或次日(节假日顺延)交到相关学籍管理部门(教务处或研究生院),由相关学籍管理部门送达学生所在学院。

(五)学生所在学院收到学生考试违规的材料后,应安排专人对学生的考试违规行为进行调查、核实。

学生本人承认其考试违规事实的,应写出书面检讨,内容包括对考试违规行为的陈述以及学生本人对错误行为的认识。考试违规行为的陈述要清楚、客观。书面检讨由学生本人签名,并注明日期。

学生对监考人员认定的考试违规行为有异议的,学院可根据实际情况成立学院调查小组,或者报相关学籍管理部门成立学校调查小组进行调查。调查小组在调查过程中应做好调查记录。调查记录应当写明调查人、被调查人的姓名、性别、单位等基本情况。调查结束后交被调查人核对签名并注明日期。被调查人拒绝签名的,调查人员应当在记录上注明情况,并由2名以上(含2名)调查人员签名,注明日期。

第十四条  学生在考试结束后被发现或者被检举有考试违规行为的,按上述调查程序进行调查后,以调查记录及收集到的证据作为认定学生考试违规的依据。

第十五条  学生本人不配合调查的,不影响对其违规行为的处理。

第十六条  学院一般应在10个工作日内完成对学生违规事实的核实工作,并提出初步处理意见报相关学籍管理部门。因故无法在10个工作日内提出处理意见的,应及时向相关学籍管理部门报告。

第十七条  学院在提出处理意见之前,应当告知学生拟给予处分的有关事实、理由和依据,听取学生的陈述和申辩,认真做好记录,学生(或其代理人)应在记录上签字,拒绝签字的,由主笔人写出文字说明。

对拟被处分学生提出的陈述和申辩的事实、理由和证据,相关学籍管理部门应当进行复核,并书面回复处理意见。

学校不得因拟被处分学生提出陈述和申辩而加重处分。

第十八条  严重警告、记过处分,由学院党政联席会议提出处理意见,学校授权相关学籍管理部门审核处理。

留校察看处分,由学院党政联席会议提出初步处理意见后报相关学籍管理部门,相关学籍管理部门审核后报分管校领导审批。

开除学籍处分,由学院党政联席会议提出初步处理意见后报相关学籍管理部门,相关学籍管理部门转学校政策法规部门进行合法性审查后,报分管校领导审核,经校长办公会议研究决定。

第十九条  学院需报送相关学籍管理部门的材料包括:学生违纪处分登记表、学院党政联席会议纪要、学生本人的检讨或调查记录、与学生的谈话记录;如果学生提出申辩,还应报送学生申辩材料。

第二十条  学校对学生进行处分,应出具处分决定书。处分决定书包括学生基本信息,处分和处分事实、理由及依据,处分的种类、依据、期限,并告知学生可以提出申诉、申诉的途径及申诉的期限。

学生若对处分决定有异议,可以在接到学校处分决定之日起10日内,向学校学生申诉处理委员会提出书面申诉。申诉期间,不停止原处分决定的执行。

第二十一条  学生受处分的期限从处分决定之日起计算。各类处分的期限为:

(一)严重警告,8个月;

(二)记过,10 个月;

(三)留校察看,12 个月。

处分的解除条件及程序按学校学生违纪处分相关规定执行。

第二十二条  依据《中华人民共和国刑法修正案(九)》之规定,学生在法律规定的国家考试中,实施本办法第十条第(一)至(六)项等考试作弊行为的,移交国家司法机关处理。



第四章  处分决定书的送达、归档及学生离校

第二十三条  处分决定书由学院负责送达学生,并由学生本人签收送达回执(一式两份,一份交相关学籍管理部门,一份由学院留存),送达回执上的签收日期即为送达日期。

学生本人拒绝签收的,可以以留置方式送达;已离校的,可以采取邮寄方式送达;无法取得联系的,可以利用学院网站以公告方式送达。

第二十四条  学生受到处分后,其处分决定和有关材料存入学校档案和学生个人学籍档案。

第二十五条  被开除学籍的学生,由学校发给学习证明,在接到处分决定之日起一周内办理离校手续并离校,档案退回家庭所在地,户口迁回原户籍地或者家庭户籍所在地。逾期不办者,后果由学生本人承担。



第五章  附则

第二十六条  如学院对考试违规学生逾期未处理,学籍管理部门应敦促抓紧处理并通报批评。如学院在处理考试违规学生时未按规定处理,学籍管理部门应要求学院重新审议处理。

第二十七条  2016年12月1日后录取的非全日制研究生参照本办法执行。

第二十八条  本办法由教务处、研究生院负责解释。

第二十九条  本办法自2017年9月1日起施行,原《武汉理工大学普通全日制学生考试违规处理办法》(校教字〔2013〕10号)同时废止。

\chapter{武汉理工大学学生申诉处理办法}
第一章  总则

第一条  为保障学生的合法权益,规范学生申诉制度,保证学校处理行为的客观、公正,根据教育部《普通高等学校学生管理规定》(教育部令第41号)等有关法律法规和《武汉理工大学章程》,特制定本办法。

第二条  我校在籍的接受高等学历教育的各类学生对学校有关生效的学生违纪处理和学籍处理决定有异议或不服,有权依据本办法提起申诉。

第三条  学生申诉的提起及处理应当依照程序,合法、合理地进行,遵循公平、公正、实事求是和有错必纠的原则。

第四条  学校成立学生申诉处理委员会(以下简称申诉委员会),由分管校领导、职能部门和相关学院负责人、教师代表和学生代表、负责法律事务的相关机构负责人等组成。申诉委员会可以聘请校外法律、教育等方面专家参加,并根据学生申诉事由成立学生申诉处理小组。

学生申诉处理委员会下设学生申诉处理办公室,挂靠校团委。学生申诉处理办公室负责受理学生申诉等日常事务。



第二章  申诉的受理

第五条  学生对学校作出的涉及本人权益的下列处分、处理决定有异议或不服,须在收到决定或公告之日起10日内向学校提出申诉。

(一)对学生本人作出的警告、严重警告、记过、留校察看、开除学籍等纪律处分。

(二)对学生本人作出的退学等学籍处理决定。

申诉人因不可抗力或其它特殊情况逾期未能提起申诉的,应当向申诉受理机构说明理由,请求许可。

第六条  学生提出申诉时,应当以书面形式向学生申诉处理办公室递交申诉书,并附上学校的处理决定。申诉书应当载明下列内容:

(一)申诉人姓名、性别、学院、年级、班级、学号、通讯地址、联系方式等;

(二)事实、理由及证据;

(三)诉求事项。

第七条  存在以下情形的,申诉受理机构不予受理:

(一)重复申诉的(有新的证据除外);

(二)超出申诉范围的;

(三)超过规定期限的;

(四)其他不符合规定的。

第八条  在申诉期间,原处理决定不停止执行。学生申诉处理委员会认为必要的,可以建议学校暂缓执行有关决定。

第九条  学生可以在未作出申诉处理决定前撤回申诉。要求撤回申诉的,必须以书面形式提出。受理申诉的机构在接到关于撤回申诉的申请书后,申诉程序自行终止。同一事项申诉撤回后,不得以同一理由再次提出申诉。



第三章  申诉的处理程序

第十条  学生申诉处理办公室收到申诉书之日起15日内通知相关申诉处理工作小组进行评议并报学生申诉处理委员会形成申诉处理决定。必要时可延长,并通知书面申诉人。延长以一次为限,最长不得超过15日。

第十一条  学生申诉处理小组召开评议会应有三分之二以上成员出席,经过出席成员半数以上同意,方能形成决议。必要时,报申诉委员会研究决定。学生申诉处理小组根据实际情况可通知申诉人及相关人员到会说明或开展其他的查证。

第十二条  申诉委员会完成评议后,根据实际情况提出处理意见,区别不同情况,作出下列处理决定:

(一)原处理决定事实清楚、适用依据正确、程序规范的,维持原处理决定;

(二)原处理决定事实不清、依据或程序不当的,作出变更原处理决定的建议。

作出建议变更原处理决定的决定时,应当及时送达作出原处理决定的部门,由作出原处理决定部门重新处理。

第十三条  申诉处理决定书应当载明下列内容:

(一)申诉人的姓名等基本情况;

(二)原作出处理决定部门的名称和原处理决定的基本内容;

(三)申诉处理决定、事实及理由;

(四)申诉处理决定书形成日期。

不受理的申诉案件也应当形成申诉处理决定书,但其内容仅列申诉处理决定及理由。

对于退学处理或开除学籍处分的申诉处理决定书,应当在处理决定书中写明“如不服本申诉决定,可以在申诉处理决定书送达之日起15日内,向湖北省教育厅提出书面申诉”。

第十四条  申诉委员会委托申诉人所在的学院将申诉处理决定书送达申诉人。送达方式可采取下列任何一种:本人签收;按申诉书通讯地址邮寄。

第十五条  学生对申诉处理决定书有异议或不服的,在申诉处理决定书送达之日起15日内,可以向湖北省教育厅提出书面申诉。



第四章  附 则

第十六条  本办法由校团委负责解释。

第十七条  本办法自2017年9月1日起施行,原《武汉理工大学学生申诉处理办法》

\part{资助管理}
\chapter{武汉理工大学家庭经济困难学生认定办法}
第一章 总 则

第一条 为认真做好我校家庭经济困难学生认定工作,切实保证各项资助政策和措施真正落实到家庭经济困难学生身上,公平、公正、合理地分配资助资源,根据《教育部财政部关于认真做好高等学校家庭经济困难学生认定工作的指导意见》(教财[2007]8号)的有关精神,特制订本办法。

第二条 本办法适用于我校在籍全日制普通本科生。

第三条 本办法中家庭经济困难学生是指学生本人及其家庭所能筹集到的资金,难以支付其在校学习期间的学习和生活基本费用的学生。

第四条 家庭经济困难学生认定工作坚持实事求是,确定合理标准,由学生本人提出申请,实行民主评议和学校评定相结合的原则。

第二章 认定机构

第五条 学校学生资助工作领导小组全面领导本校家庭经济困难学生的认定工作。学生工作部(处)具体负责组织和管理全校的认定工作。

第六条 成立学院资助工作组。学院资助工作组以分管学生工作的党委副书记为组长、学生工作办公室主任、学生辅导员等为成员,负责本学院家庭经济困难学生认定的具体组织、审核及各种资助项目的评审工作。

第七条 成立班级资助评议小组。以班级为单位,成立以学生辅导员任组长,班主任、班长、团支部书记、党支部书记(党小组组长)、生活委员、寝室长及其他学生代表为成员的班级资助评议小组。班级资助评议小组成员中,学生人数视班级人数合理配置,应具有广泛的代表性,一般不少于班级总人数的20%,班级资助评议小组成员名单应在班级公示。班级资助评议小组负责家庭经济困难学生认定的民主评议工作及各种资助项目的评议工作。

第三章 认定标准、等级及条件

第八条 在校学习期间的学习和基本生活费用包括学费、住宿费和基本生活费用。学费和住宿费以计划财务处核定的数据为标准;基本生活费用以武汉市规定的城市最低生活保障线为标准。认定等级分为家庭经济特殊困难学生和家庭经济一般困难学生。

第九条 在家庭经济困难学生认定过程中应考虑如下因素:

1、必要条件:

(1)家庭人均年收入,家庭可提供的月均生活费;

(2)学生在校消费情况:生活俭朴,无高消费现象。

2、重要参考条件:

(3)孤儿、烈士子女、优抚对象子女;

(4)身体残疾或重大疾病的学生;

(5)学生家庭遭受严重自然灾害或重大变故;

(6)家庭享受城镇居民最低生活保障。

3、一般参考条件:

(7)贫困地区农村;

(8)父母无劳动能力或父母双下岗(失业)且无固定收入;

(9)家庭多子女上学(非义务教育);

(10)单亲家庭且经济困难;

(11)艰苦边远地区少数民族学生;

(12)其它情况导致家庭经济困难的。

第十条 综合考虑以上因素,按下列认定标准,将家庭经济困难学生分为特殊困难和一般困难两档。

1、家庭年收入无法负担学生学费和住宿费,家庭可提供的月均生活费低于武汉市公布的城市最低生活保障线,生活俭朴,没有高档消费品或高消费行为;且至少符合一项重要参考条件或至少符合两项一般参考条件的学生,原则上可认定为家庭经济特殊困难学生。

2、家庭年收入无法全额负担学生学费和住宿费,家庭可提供的月均生活费低于武汉市公布的城市最低生活保障线,生活俭朴,没有高档消费品或高消费行为;且至少符合一项一般参考条件的学生,原则上可认定为家庭经济一般困难学生。

第十一条 各学院资助工作组、班级资助评议小组应主动发现家庭经济特别困难但本人没有提出认定申请的学生,积极宣传国家资助政策。

第四章 认定程序

第十二条 家庭经济困难学生认定工作每学年进行一次。每年5-6月进行;新生中家庭经济困难学生认定工作在每年9-10月进行。

第十三条 填写《高等学校学生及家庭情况调查表》。学校每年在向新生寄送录取通知书时,同时寄送《高等学校学生及家庭情况调查表》;在校学生如果提出申请认定,可在每年开展认定工作时在学工广场下载或到学院学生工作办公室领取《高等学校学生及家庭情况调查表》,将填写完整的《高等学校学生及家庭情况调查表》提交班级资助评议小组。需要申请认定家庭经济困难的新生及提出申请认定的在校生要如实填写《高等学校学生及家庭情况调查表》,并持该表到家庭所在地乡、镇或街道民政部门加盖公章,以证明其家庭经济状况。

第十四条 班级资助评议小组负责收集《高等学校学生及家庭情况调查表》,并组织学生填写《高等学校家庭经济困难学生认定申请表》。上学年已被学校认定为家庭经济困难的学生再次申请认定时,如家庭经济状况无显著变化,可只提交《高等学校家庭经济困难学生认定申请表》,不再提交《高等学校学生及家庭情况调查表》。

第十五条 班级资助评议小组根据学生提交的《高等学校家庭经济困难学生认定申请表》和《高等学校学生及家庭情况调查表》,以学校确定的认定标准,并结合学生日常消费状况以及影响其家庭经济状况的有关情况,认真进行评议,确定本班级各档次的家庭经济困难学生资格,报学院资助工作组进行审核。

第十六条 学院认定工作组认真审核班级认定评议小组申报的初步评议结果。如有异议,应在征得班级资助评议小组意见后予以变更。学院认定工作组审核通过后,要将家庭经济困难学生名单及档次,以适当方式、在适当范围内公示5个工作日。如师生有异议,可通过学院网上信箱或书面形式向本学院资助工作组提出质疑。资助工作组应在接到书面质疑材料的3个工作日内予以答复。如对学院资助工作组的答复仍有异议,可提交书面材料向学生工作部(处)提请复议。学生工作部(处)在接到提请复议书面材料的3个工作日内予以答复。如质疑情况属实,应做出调整。

第十七条 学生工作部(处)负责汇总各学院审核通过的《高等学校家庭经济困难学生认定申请表》和《高等学校学生及家庭情况调查表》,报学校学生资助工作领导小组审批,审批通过后建立家庭经济困难学生信息档案。

第五章 监督与管理

第十八条 学生工作部(处)和各学院每学年定期对全部家庭经济困难学生进行一次资格复查,并不定期地随机抽选一定比例的家庭经济困难学生,通过信件、电话、实地走访等方式进行核实。如发现弄虚作假现象,一经核实,取消资助资格,收回资助资金。情节严重的,学校依据有关规定进行处理。

第十九条 有下列行为之一者,停止各项资助并退出家庭经济困难学生信息库。

1、拥有或使用高档消费品或有高消费行为的;

2、未经批准在校外租房或沉溺网络的;

3、有抽烟、酗酒等不良消费行为且经教育不改的;

4、有与其家庭经济困难状况不相符的其它不当消费行为的。

第二十条 各学院应加强学生的诚信教育,教育学生如实提供家庭情况,及时告知家庭经济显著变化情况。如学生家庭经济状况发生显著变化,应及时做出调整,对家庭突发重大变故导致家庭经济困难的,应临时调整进入家庭经济困难学生信息库;同时鼓励家庭经济状况好转的学生主动申请退出家庭经济困难学生信息库。临时调整进入或退出家庭经济困难学生信息库都按班级资助评议小组评议、学院资助工作组审核、学校资助工作领导小组审批的程序进行。

第六章 附 则

第二十一条 被认定为家庭经济困难且录入到家庭经济困难学生信息档案的学生可申请本学年提供的相关资助,保证基本的学习和生活。

第二十二条 本办法自颁布之日起执行。《武汉理工大学家庭经济困难学生认定办法(试行)》(校学字[2007]16号)同时废止。

第二十三条 本办法由学生工作部(处)负责解释。

\chapter{武汉理工大学国家助学贷款管理实施办法}
为了贯彻落实国家助学贷款政策,帮助我校家庭经济困难学生顺利完成学业,根据国家相关法律法规及文件精神,结合我校具体情况,特制定本办法。

第一章 组织机构及职责

第一条 武汉理工大学国家助学贷款工作领导小组为我校国家助学贷款工作的领导机构。学生工作部(处)为日常管理机构,各学院为国家助学贷款工作的具体组织实施单位。

第二条 相关部门的职责

(一)学生工作部(处):宣传贯彻执行国家有关助学贷款的方针、政策,代表学校作为学生的介绍人对外联系,拟定学生年度贷款需求计划;协助经办银行整理、审核借款学生材料以及其他相关手续的办理;负责借款学生相关信息档案管理及数据的统计和上报工作,建立助学贷款学生信息库,及时为贷款经办银行提供所需学生信息;配合当地资助中心做好生源地信用助学贷款相关工作;做好国家助学贷款和生源地信用助学贷款放款后划拨给借款学生的各项工作。

(二)财务处:负责学校与银行合作协议相关条款的落实工作以及风险补偿资金的预算及支付;根据学生工作部(处)拟定的学生年度借款需求计划与相关银行具体落实贷款额度;负责指定国家助学贷款的资金账户;接受经办银行直接划入的贷款资金,将国家助学贷款实放款额划拨到具体的学生并为其办理有关手续。

(三)教务处:负责向学生工作部(处)通报借款学生转学、休学、留级、退学、出国等学籍异动情况,待办妥国家助学贷款相关手续后,方可办理学籍异动手续。

(四)研究生院:负责对学院提供的申请贷款研究生家庭经济情况、学习成绩、平时表现提出初步审核意见及申请材料的整理;提供借款研究生转学、休学、留级、退学、出国及受处分的名单和毕业时的去向及联系地址、联系方式;协助做好贷款研究生毕业时的各项相关工作。

(五)各学院:对本单位申请借款学生(包括研究生)家庭经济情况以及在校的生活情况提出初步审核意见;负责本单位申请借款学生的材料整理、建档;负责本单位借款学生毕业时的贷款确认工作;协助做好借款学生的还贷工作。

第二章 贷款对象和条件

第三条 贷款对象为家庭经济困难的在校普通全日制本科学生和研究生。

第四条 申请贷款的学生需具备以下条件:

(一)具有中华人民共和国国籍,且持有中华人民共和国居民身份证;

(二)具有完全的民事行为能力(未成年人须有其法定监护人书面意见);

(三)遵守国家法律和学校规章制度,无违法违纪行为;勤俭节约;不挥霍浪费,无吸烟、酗酒等不良嗜好;

(四)品德优良、学习刻苦、能够正常完成学业;

(五)家庭经济收入不足以支付完成学业所需学费和住宿费;

(六)诚实守信,能严格执行贷款协议,无不良信用行为;

(七)符合贷款人规定的其他条件。

第三章 贷款额度和贷款申请

第五条 学校根据全国学生资助管理中心下达的贷款额度和经办银行的规定,核定每个学生的年度贷款金额,申请贷款的金额本科生每人每学年不超过8000元;研究生每人每学年不超过12000元。

第六条 学生在校期间原则上只能申请一次国家助学贷款,并须在每年学校规定的日期内向学院提出书面贷款申请。

第七条 申请国家助学贷款的学生必须如实提供以下材料:

(一)《国家助学贷款申请审批表》;

(二)国家助学贷款申请书(内容包含本人对家庭经济困难情况的说明,并由本人签名);

(三)有效家庭经济困难证明原件(学生家庭所在地乡/镇人民政府或县级民政部门、城市街道办事处出具的家庭经济困难证明);新生可提供有效家庭经济困难证明原件,或提供加盖上述部门公章的《高等学校学生及家庭经济情况调查表》;

(四)借款学生身份证复印件(未成年学生需提供法定监护人的身份证复印件和同意申请贷款的书面证明);

(五)银行或学校要求提供的其他证明材料。

第四章 贷款的期限、利率与贴息

第八条 国家助学贷款的贷款期限为学制加13年、最长不超过20年。借款学生毕业当年不再继续攻读学位或在还款期内继续攻读学位的借款学生再读学位毕业后,可享受36个月的还本宽限期,还本宽限期内借款学生只需偿还利息,无需偿还贷款本金。

第九条 国家助学贷款利率按照中国人民银行公布的法定贷款利率和国家有关利率政策执行。

第十条 为体现国家对经济困难学生的优惠政策,减轻学生的还款负担,对享受国家助学贷款的学生实行在读期间的贷款利息全部由财政补贴,毕业后全部自付的办法,借款学生毕业或终止学业后开始计付利息。

第五章 贷款的审批与发放

第十一条 学生申请国家助学贷款由学生所在学院初审,并组织符合贷款条件的学生如实填写银行规定的有关贷款申请材料、贷款合同及相关表格;报经学生工作部(处)复审后,由经办银行审批。

第十二条 国家助学贷款实行一次签订合同、贷款分年发放的办法。贷款直接划入学校指定的账户用于缴纳借款学生的学费和住宿费。

第六章 贷后管理

第十三条 借款学生要保证申请材料的真实和完整,认真履行与银行签订的还款协议,承担偿还贷款的全部责任。

第十四条 借款学生有下列行为之一,将停止发放贷款。

(一)借款学生未按合同规定的用途使用贷款的;

(二)借款学生有违法违纪行为,受到学校行政记过及以上处分或公安部门刑事处罚的。

第十五条 借款学生家庭经济好转,可以申请停发当年贷款。

第十六条 借款学生办理毕业手续前必须与经办银行签订还款确认协议,确认还款计划,还款期限由借贷双方协商确定。同时,必须向学生工作部(处)提供离校后的去向和有效联系方式,如借款学生不办理确认手续或未提交上述材料的,学校暂缓为其办理离校手续。借款学生毕业后工作单位和联系方式发生变更,须及时通过信函或其它方式告知贷款银行和学院备案。

学生结业后不再享受贷款利息补贴,须按正常程序与经办银行办理还款确认手续,确认还款计划。

第十七条 贷款学生毕业后可到经办银行申请一次性结清贷款;到异地工作的,可通过异地相关银行分支机构办理网上银行进行贷款结清操作,也可委托他人到贷款经办银行办理代为办理还贷手续。

第十八条 借款学生如发生转学、出国等情况,必须在学校和经办银行与待转入的学校和相应经办银行办理贷款的债务划转手续后,或者在还清贷款本息后方可办理相关手续;借款学生在校期间因患病等原因休学的,应及时向学校提供书面证明,由学校向经办银行提出申请,休学期间的贷款利息由财政全额贴息;借款学生如被退学或开除学籍,在办理离校手续前必须一次性结清贷款或签订还款协议并开始自主偿还贷款本息。

第十九条 借款学生毕业后,在还款期内继续攻读学位的,可申请继续贴息。借款学生应及时向学校提供书面证明,学校审核后报经办银行确认,继续攻读学位期间发生的贷款利息由财政部门继续全额贴息。

第二十条 借款学生毕业或终止学业后一年内,可以向银行提出一次调整还款计划的申请,经办银行根据实际情况和有关规定同意后,可以进行调整。因特殊情况不能按期还款的,经贷款银行同意,可按有关规定展期,并办理相关手续。

第二十一条 毕业后自愿到中西部地区或艰苦边远地区县以下基层单位就业,且服务期在三年以上(含三年)的应届毕业生,可申请国家助学贷款代偿资助。具体办法按教育部相关政策规定以及《武汉理工大学毕业生学费和国家助学贷款代偿实施办法》执行。在校生和毕业生应征入伍服义务兵役的,可申请应征入伍服义务兵役国家资助,具体办法按教育部相关政策规定以及《武汉理工大学学生应征入伍服义务兵役国家资助实施办法》执行。

第二十二条 对没有按照协议约定的期限、数额归还国家助学贷款的学生,经办银行将对违约贷款金额计收罚息,并将其违约行为载入金融机构征信系统,金融机构不再为其办理新的贷款和授信业务。

第七章 生源地信用助学贷款

第二十三条 生源地信用助学贷款是国家助学贷款的重要组成部分,与国家助学贷款享有同等优惠政策。生源地信用助学贷款是指国家开发银行向符合条件的家庭经济困难学生发放的、在学生入学前户籍所在县(市、区)办理的助学贷款。生源地信用助学贷款为信用贷款,学生和家长(或其他法定监护人)为共同借款人,学生在校期间的利息全部由国家财政负担,毕业后利息全部由借款人(或共同借款人)负担。

第二十四条 生源地信用助学贷款工作中学校的主要职责

(一)向生源地教育机构和经办银行证明学生在校期间的生活、学习等方面的表现;

(二)确认学生在校学习期间的学费和住宿费,并按合同要求确认贷款回执信息;

(三)提供学校财务账号和财务联系人,负责用学生贷款冲抵学费和住宿费,向借款学生通告到款信息。

第八章 贷款救助机制

第二十五条 学校利用国家助学贷款风险补偿金结余奖励资金、社会捐资助学资金或学生奖助基金,建立国家助学贷款还款救助机制,用于救助特别困难的毕业借款学生。对于因病丧失劳动能力、家庭遭遇重大自然灾害、家庭成员患有重大疾病以及经济收入特别低的毕业借款学生,如确实无法按期偿还贷款,可向经办机构(学校或组织办理生源地信用助学贷款的县级教育部门)提出救助申请并提供相关书面证明,经办机构核实后,可启动救助机制为其代偿应还本息。

第九章 附 则

第二十六条 本办法由学生工作部(处)处负责解释。

第二十七条 本办法自发布之日起实行。
\chapter{武汉理工大学国家助学贷款贷后管理实施办法}
第一章 总 则

第一条 为加强国家助学贷款实施过程中各环节的管理,保证国家助学贷款良性发展,协助贷款经办银行做好贷后管理的有关工作,特制订本办法。

第二条 本办法中所指的借款学生是指已经获得国家助学贷款资助的我校全日制本科学生和研究生。本办法中所指的借款毕业生是指获得国家助学贷款资助且已经毕业的学生。

第三条 本办法中所指的在校期间是指学生从获得国家助学贷款起至学生毕业离校的时间段。

第四条 本办法中所指的毕业离校后是指学生从毕业离校到全部还清贷款本息的时间段。

第五条 本办法中所指的合同是指借款学生与贷款经办银行之间签订的《国家助学贷款合同》。

第六条 本办法中所指的还款协议是指借款学生在毕业离校前和贷款经办银行签订的《国家助学贷款还款协议》。

第七条 本办法中所指的资料确认书是指贷款经办银行提供给借款毕业生填写的《国家助学贷款毕业生资料确认书》。

第二章 部门职责

第八条 学生工作部(处)

(一)负责与贷款经办银行联系开展新增国家助学贷款、续放款、还款确认、贷款展期等工作;

(二)负责将借款学生异动情况通知贷款经办银行;

(三)负责开展国家助学贷款政策宣传;

(四)负责汇总并审核借款学生的各项材料;

(五)负责为借款学生建立档案库,整理借款学生的相关材料并归档;

(六)负责将银行提供的国家助学贷款逾期未还款学生信息通知到相关学院,并将学院联系学生的情况集中反馈给经办银行。

第九条 财务处

(一)负责与国家助学贷款经办银行间的资金划拨;

(二)负责管理国家助学贷款风险补偿金;

(三)负责管理国家助学贷款代偿金;按照全国学生资助管理中心审批名单将借款学生相应代偿资金划拨经办银行。

第十条研究生院

建立健全贷款研究生信息档案,负责贷款研究生贷后管理工作的组织实施。

第十一条各学院

(一)建立健全借款学生信息档案;

(二)通过各种形式在学生中宣传国家助学贷款相关政策,组织学生开展诚信教育活动,提高学生诚信还款意识;

(三)组织并指导借款学生填写贷款相关材料,并审核材料是否完整、准确;

(四)及时报送借款学生转学、休学、出国、退学、开除学籍等情况,督促并指导学生办理相关手续;

(五)负责联系逾期未还款学生,并向学生工作部(处)提供逾期未还款学生的有效联系方式。

第三章 学生在校期间的贷后管理

第十二条 学生获得国家助学贷款后,各学院要及时建立国家助学贷款学生档案库,实行动态管理。

第十三条 借款学生转专业时,转出、转入学院都应向学生工作部(处)上报转专业借款学生的相关信息,同时转出学院应及时将转专业借款学生相关信息移交转入学院,转入学院要及时为借款学生建立贷款档案。

第十四条 借款学生办理了留级、休学、复学手续后,学院需在一周内将借款学生留级、休学、复学通知单复印件报送学生工作部(处)。

第十五条 借款学生转学、退学时,学院要及时通知其到贷款经办银行偿还贷款本息。借款学生凭银行出具的贷款结清凭证到学院办理转学手续,学院在一周内将学生贷款结清凭证复印件报送学生工作部(处)。

第十六条 借款学生毕业离校时,由学院组织并指导借款学生填写《国家助学贷款还款协议》(一式三份)、《国家助学贷款毕业生资料确认书》(一式两份)及《国家助学贷款毕业生信息采集表》(电子版),学院负责对上述材料进行审核,确保材料完整、准确,并在规定时间内报送学生工作部(处)。对于未填写上述材料的借款学生,学院暂缓为其办理离校手续。每年7月10日前,学院将未办理还款手续的借款学生基本情况以书面形式报学生工作部(处)。

学院将银行审核后的还款协议在毕业生离校前发放至学生本人一份,学生工作部(处)存档保存一份。

第十七条 借款学生在校期间因患病等原因休学的,应及时向学校提供书面证明,由学校向经办银行提出申请,休学期间的贷款利息由财政全额贴息。

第十八条 借款学生毕业后,在还款期内继续攻读学位的,可申请继续贴息。借款学生应及时向学校提供书面证明,学校审核后报经办银行确认,继续攻读学位期间发生的贷款利息由财政部门继续全额贴息。符合申请条件的学生必须在学校和经办银行要求的时间内办理相关手续。未按要求提供书面证明或未及时办理相关手续的借款学生,继续攻读学位期间,银行将执行原还款协议相关内容。

第十九条 毕业后自愿到中西部地区或艰苦边远地区县以下基层单位就业,且服务期在三年以上(含三年)的应届毕业生,可申请国家助学贷款代偿资助。具体办法按教育部相关政策规定以及武汉理工大学毕业生学费和国家助学贷款代偿实施相关规定执行。在校生和毕业生应征入伍服义务兵役的,可申请应征入伍服义务兵役国家资助,具体办法按教育部相关政策规定以及武汉理工大学毕业生学费和国家助学贷款代偿实施相关规定执行。学院应根据相关要求组织并指导学生准备材料,汇总后按规定时间报送学生工作部(处)。经全国学生资助管理中心审批获得代偿资格的学生,其代偿资金下拨学校后,由学生工作部(处)联系贷款经办银行逐步结清学生贷款,财务处负责具体的资金划拨操作。

第二十条 借款学生在校期间意外死亡的,学院将法定的死亡证明和火化证明复印件报送学生工作部(处),学生工作部(处)按规定程序报送贷款经办银行,核销其助学贷款。

第二十一条 办理生源地信用助学贷款的学生,其在校期间的贷后管理工作根据学生生源所在地的资助机构相关要求办理。

第二十二条 诚信教育是贷后管理的重要内容,学院需开展经常性的诚信教育活动,提高学生诚信意识。

第四章 学生毕业离校后的贷后管理

第二十三条 学生毕业离校前,学院根据借款学生毕业去向建立借款毕业生信息档案,补充完善《国家助学贷款毕业生信息采集表》。学生工作部(处)根据毕业生信息填写发放给“贷款毕业生的三封信”(即《致国家助学贷款毕业生的一封信》、《致国贷毕业生家长一封信》、《致国贷毕业生单位一封信》),并统一安排邮寄。对借款学生或学生家长的信件回执学院应按照回执内容在《国家助学贷款毕业生信息采集表》中修改、补充相关信息。对被退回或未反馈回执的信件,学院要将相应的借款学生重点记录,联系学生本人重新核对通讯信息,并告知学生工作部(处)重新邮寄相关信件。

第二十四条 银行向学校提供的逾期未还款学生名单,学生工作部(处)核实学生信息后发送至相关学院。学院及时与逾期未还款学生联系,督促其尽快履行还款义务,并将联系情况按要求报送学生工作部(处)。学生工作部(处)汇总整理后反馈贷款经办银行。助学贷款逾期信息各学院留存备查,并进一步跟踪督促学生及时偿还国家助学贷款。

第二十五条 各省级学生资助管理部门、学校利用国家助学贷款风险补偿金结余奖励资金、社会捐资助学资金或学生奖助基金,建立国家助学贷款还款救助机制,用于救助特别困难的毕业借款学生。对于因病丧失劳动能力、家庭遭遇重大自然灾害、家庭成员患有重大疾病以及经济收入特别低的毕业借款学生,如确实无法按期偿还贷款,可向经办机构(学校或组织办理生源地信用助学贷款的县级教育部门)提出救助申请并提供相关书面证明,经办机构核实后,可启动救助机制为其代偿应还本息。

第五章 附 则

第二十六条 本办法由学生工作部(处)负责解释。

第二十七条 本办法自发布之日起施行。

\chapter{武汉理工大学国家助学金管理实施办法}
第一章 总 则

第一条 为体现党、政府和学校对普通本科高校家庭经济困难学生的关怀,帮助他们顺利完成学业,根据财政部、教育部《普通本科高校、高等职业学校国家助学金管理暂行办法》(财教[2007]92号)有关精神,结合学校实际,特制定本办法。

第二条 国家助学金每学年评审一次,坚持公开、公平、公正和集体决定的原则。

第三条 国家助学金的资助对象是全日制普通本科在校生中的家庭经济困难学生。

第二章 资助标准与基本条件

第四条 国家助学金主要资助家庭经济困难学生的生活费用开支,资助标准分为甲、乙、丙三等,其中:甲等每年3000元,乙等每年2000元,丙等每年1000元。按10个月发放。

第五条 申请国家助学金的基本条件:

1、热爱社会主义祖国,拥护中国共产党的领导;

2、自觉遵守宪法和法律,遵守学校各项规章制度;

3、诚实守信,道德品质优良;

4、学校认定的家庭经济困难学生,生活俭朴;

5、学习态度端正,勤奋上进;

6、上学年所学主修专业必修课合格或综合测评排名在本专业或班级前70%或学习成绩较上学年有进步。

第三章 评审程序

第六条 国家助学金评审与学生评先评奖同步进行,指标数根据教育部分配的名额划拨到学院。

第七条 学生向所在学院提出书面申请,详细说明申请理由,填写《武汉理工大学国家助学金申请表》。

第八条 班级资助评议小组根据申请条件,综合考虑学生家庭经济情况及本学年受资助情况,对申请国家助学金的学生进行评议,提出享受国家助学金的建议名单及资助等级,并报学院资助工作组审核。

第九条 学院资助工作组对班级资助评议小组提出的建议名单及资助等级进行资格审查和评选,初步确定享受国家助学金的学生名单及资助等级,在全院范围内以适当形式公示三天,无异议后在《武汉理工大学国家助学金申请表》上签署推荐意见,在规定的时间内报送学生工作部(处)。

第十条 学生工作部(处)审核后报学校资助工作领导小组审定,公示三天,无异议后上报教育部。

第四章 发放及管理

第十一条 学校接到国家助学金批复和专款后,由学生工作部(处)和计划财务处统一组织按月直接打入学生校园卡对应的银行联名卡账户。

第十二条 有下列情形之一者,停发国家助学金:

1、购买高档消费品或有高消费行为的;

2、有违法违纪行为,受到警告及以上处分的;

3、弄虚作假,谎报家庭经济情况获取资助的;

4、因各种原因休学、退学、保留学籍的,或以其它方式离开学校的;

5、临时退出家庭经济困难学生信息库的;

6、其它不符合国家助学金相关管理规定的。

第十三条 加强学生诚信教育,对弄虚作假者,停发国家助学金并追回已发国家助学金。情节严重者,按《学生手册》有关规定给予纪律处分。

第十四条 停发或追回的国家助学金进入学校助学金专款账户,下学年使用。

第十五条 鼓励获得国家助学金的学生积极参加各种社会公益活动,感恩他人,回报社会。学生参加社会公益活动情况作为下一次资助评定的参考条件。

第五章 附 则

第十六条 本办法自颁布之日起执行。《武汉理工大学国家助学金评审办法(试行)》(校学字[2007]17号)同时废止。

第十七条 本办法由学生工作部(处)负责解释。

\chapter{武汉理工大学全日制普通本科学生无息借款管理办法}
第一条 为了进一步规范全日制普通本科学生无息借款管理工作,根据国家相关法律法规及文件精神,结合学校实际,制定本办法。

第二条 申请借款条件

凡我校在籍全日制普通本科学生中的家庭经济困难学生,遵守国家法律和学校的各项规章制度,学习刻苦,态度端正,艰苦朴素,勤俭节约,无吸烟、酗酒等不良嗜好者均可申请无息借款。

第三条 借款等级和金额

甲等借款:每人每学年2000元;

乙等借款:每人每学年1000元。

第四条 借款程序

(一)学生本人提出申请,填写《武汉理工大学学生无息借款申请审批表》;

(二)班级评议,学院审核并公示无异议后,报学生工作部(处)审批;

(三)财务处发放。

第五条 借款期限与偿还

(一)无息借款期限以学生毕业时间为最后期限,毕业前必须全部还清借款;

(二)在校期间,中途退学、出国者,必须全部还清借款。

第六条 借款减免条件

(一)学生借款后,符合下列条件之一者,可减免50%的借款:

1.获得“校三好学生”或“校优秀学生干部”或“五四青年奖章”称号一次者;

2.两次获得“校优秀团干”或“校优秀共青团员”或“校优秀学生会干部”称号者;

3.三次获得“院三好学生”或“院优秀学生干部”称号者;

4.毕业时被录取为研究生者;

5.在省以上大学生课外学术活动或竞赛活动中,其科研成果或竞赛获得三等奖及以上者;

6.在省以上大型文艺演出比赛中获单项比赛三等奖及以上者;集体项目三等奖及以上的主要表演者,或由省教育厅、团省委、高校体协组织的体育正规联赛(不包括邀请赛和选拔赛)冠军队的主力队员。

(二)学生借款后,符合下列条件之一者,可减免100%的借款:

1.获省及以上“三好学生”或“校三好学生标兵”或“校优秀共产党员”称号者;

2.两次获得“校三好学生”或“校优秀学生干部”或“五四青年奖章”称号者;

3.在校期间获国家级大学生课外学术科研成果或竞赛三等奖及以上者;

4.在国家级大型文艺演出或比赛中获单项比赛三等奖及以上者;集体项目三等奖及以上的主要表演者或由国家组织的体育正规联赛(不包括邀请赛和选拔赛)冠军队的主力队员;

5.毕业志愿到国家规定的边远贫困地区工作且办理了手续者或志愿服务西部计划人员;

6.志愿到农村支教或到村任职且办理了相关手续的毕业生;

7.在校期间,因故严重病残或死亡者。

以上各项减免条件不累加、不重复使用。

第七条 获得学校无息借款的学生应认真履行协议内容,按时还款,并积极参加学校组织的志愿服务和义工活动。

第八条 本办法由学生工作部(处)负责解释。

第九条 本办法自公布之日起执行。

\chapter{武汉理工大学全日制普通本科学生减免学费管理办法}
第一条 根据国家有关文件精神,为切实做好家庭经济困难学生资助工作,进一步完善家庭经济特别困难学生的资助和保障机制,使他们能安心学习,顺利完成学业,结合我校实际,制订本办法。

第二条 减免条件

凡我校在籍全日制普通本科学生,家庭经济特别困难,无力支付学费且当年所获资助不足以帮助其完成学业,同时符合以下条件者,可申请减免学费。

(一)遵守宪法和国家的各项法律、规定;

(二)自觉遵守大学生行为准则和学校规章制度;

(三)学习勤奋刻苦、学习目的明确、学习态度端正;

(四)道德品质好,生活艰苦朴素、勤俭节约,无高档消费品和高消费行为。

第三条 对符合上述条件的孤儿、残疾学生、烈士子女、优抚对象子女、贫困地区的少数民族学生、家庭遭受重大自然灾害的学生予以优先考虑。

第四条 减免程序

(一)本人提出书面申请,同时提交由学生家庭所在地乡镇(街)人民政府或县级以上民政部门出具的家庭经济困难状况证明材料;

(二)经班级民主评议,班主任或辅导员签署意见,学院审查并公示无异议后,填写《武汉理工大学学生减免学费申请表》;

(三)学生工作部(处)审核,学校奖助工作领导小组审批,统一办理减免当学年学费手续;

(四)财务处根据减免学费名单及金额,一次性减免学费。

第五条 减免学费金额根据学生的家庭经济状况和所获资助状况确定。

第六条按照国家有关法律法规和政策规定应当减免学费的(如“西部开发助学工程”受助学生的学费减免)按国家相关政策执行。

第七条 取消减免

(一)违犯国家法律、法规和学校规章制度,受到党、团、行政警告及以上处分者,补交所减免的学费;

(二)弄虚作假、谎报家庭情况而获得减免者,除补交所减免的学费外,在校期间不得再申请享受其它资助,并根据情节轻重给予一定的纪律处分;

(三)获得减免后购买高档消费品或有高消费行为者,不再享受学校资助。

第八条 获得学费减免的学生应认真履行一定的义务,并积极参加学校的志愿服务和义工活动。

第九条 本办法由学生工作部(处)负责解释。

第十条 本办法自公布之日起执行。

\chapter{武汉理工大学全日制普通本科学生困难补助管理办法}
第一条 根据国家有关文件精神,为切实做好家庭经济困难学生资助工作,进一步完善家庭经济特别困难学生的资助和保障机制,使他们能安心学习,顺利完成学业,结合我校实际,制订本办法。

第二条 凡我校在籍全日制普通本科学生,家庭经济特别困难者,可申请困难补助。

第三条 学生困难补助分为临时困难补助和专项特困补助两类。

第四条 申请条件

(一)家庭经济特别困难,或因意外事件或疾病造成暂时生活困难;

(二)学习成绩良好,无违纪行为。

第五条 补助标准和申请程序

(一)学生临时困难补助和特殊临时困难补助标准由学工部(处)根据实际情况确定。

学生申请临时困难补助须本人书面申请,经班主任或辅导员核实,签署意见后,呈报学院审批,由财务处通过酬金发放系统将补助金额划入学生银行卡中。

学生因突发事件造成家庭重大经济损失,严重影响学生正常学习和生活,可提出特殊临时困难补助申请。特殊临时困难补助须本人书面申请,经班主任或辅导员核实,签署意见后,呈报学院审核,经学生工作部(处)审批后,凭《学生临时困难补助通知单》到财务处领取。

(二)学生专项特困补助分为甲、乙、丙三等,其金额为:甲等:1000元/人;乙等:500元/人;丙等:300元/人。

专项特困补助由学生工作(部)处根据实际情况确定是否开展及开展时间。专项特困补助须学生本人书面申请,班级评议,班主任或辅导员签署意见,学院审核并公示无异议后,报学生工作部(处)审批,由财务处发放。

第六条 取消补助

凡有下列行为之一者,不再享受困难补助:

(一)学习不刻苦,经常旷课、迟到,一学年有3门及以上主修专业必修课不合格;

(二)有抽烟、酗酒等铺张浪费行为和高消费行为;

(三)对所提供的勤工助学岗位不积极参与。

第七条 获得临时困难补助的学生应积极参加学校组织的志愿服务和义工活动。

第八条 本办法由学生工作部(处)负责解释。

第九条 本办法自公布之日起执行。

\chapter{武汉理工大学全日制普通本科学生勤工助学实施办法}
第一条 为加强对校内勤工助学活动的统一管理,特制定本办法。

第二条 参加勤工助学的对象为家庭经济困难的在籍全日制普通本科学生。

第三条 参加勤工助学的条件

(一)遵守国家法律,遵守校纪校规;

(二)被认定为家庭经济困难学生;

(三)学有余力;

(四)诚实守信,吃苦耐劳,有责任心,能正确处理有酬劳动和公益劳动的关系。

第四条 设岗定编基本原则

(一)设岗原则

校内勤工助学固定岗位是持续一个学期以上的长期性岗位和寒暑假期间的连续性岗位,由学校出资或由校内用工单位出资(全部或部分),为解决家庭经济困难学生的生活困难而设立的岗位。岗位设置需符合以下要求:

1.有利于培养学生自强、自立、自尊精神;

2.有利于建设和谐校园,营造育人氛围;

3.安全、无毒、无害,学生力所能及;

4.不能与学生的学习时间产生冲突;

5.不能替代校内教职员工的本职工作。

(二)定编原则

1.学院办公室助理,主要指各学院学工、行政、教学等办公室,每学院可设12-20个勤工助学岗位;

2.学院资料室助理,主要指各学院教工资料室和学生资料室、图书馆分馆、阅览室等,每学院可设1-3个勤工助学岗位;

3.机关及直属单位如要求申请勤工助学岗位,须提供详细的岗位职责、岗位工作内容和工作时间,根据部门实际工作情况与岗位工作量确定岗位数。

第五条 岗位申报

(一)用工单位根据本部门的工作特点和实际情况向学生工作部(处)申报用工计划,提出用工申请,按照要求填写《勤工助学岗位计划申请表》,说明所需岗位名称、岗位数、岗位职责、工作时间及用工基本要求,同时将《勤工助学岗位计划申请表》纸质文档经用工单位负责人签署意见后交学生工作部(处),经学生工作部(处)审核,报学生工作部(处)审批后方可确定勤工助学岗位数;

(二)固定岗位申报、核定工作原则上一年进行一次,在每年6月份(假期用工在寒、暑假学生正式放假30天前)进行。临时岗位须事先书面申请,经学生工作部(处)同意后方可用工。

第六条 岗位招聘

(一)每年9月初由学生工作部(处)组织用工单位、学生举行双向招聘面试;在10月和4月底分别举行新生上岗和毕业生离岗后招聘面试;

(二)学生工作部(处)将招聘岗位名称、岗位数量、职责范围、劳动时间、酬金标准、招聘条件、招聘时间等向全校公开发布;

(三)申请参加勤工助学的学生,填写《学生勤工助学申请表》,向用工单位提出上岗申请;

(四)用工单位根据学生工作部(处)的招聘安排组织面试,申请岗位的学生持本人学生证参加面试;

(五)用工单位将面试录用名单提交学生工作部(处),学生工作部(处)经审核并网上公布后,由用工单位与学生签订《勤工助学用工协议》并安排上岗;

(六)学生工作部(处)将被录用的学生名单反馈给各学院备份,作为实施联动资助的参考条件之一;

(七)用工单位在录用时应根据用工需要、学生个人自愿、学生家庭经济困难的原则录用,不得一人多岗;对某些技术含量高,要求有一定管理才能和特殊专业能力,家庭经济困难学生中无人胜任的岗位,用工单位应与学生工作部(处)书面协商,适当考虑其他学生参加。

第七条 岗位管理

(一)用工单位勤工助学工作应有领导专门负责,并指派思想素质好、业务能力和责任心强的同志具体负责岗位的申报、招聘、录用、指导、考核等工作,切实加强勤工助学管理;

(二)学生上岗前,用工单位必须对学生进行短期培训,进行安全技术、岗位要求和职业道德的教育;

(三)学生上岗后,必须遵守用工单位的劳动纪律,按要求完成工作任务,不得擅自离岗、退岗。如有特殊情况,须向用工单位提出书面申请,经用工单位负责老师签字盖章后,到学生工作部(处)办理退岗手续;

(四)为保证学生的学习时间,学生参加勤工助学时间原则上每周不超过8小时,每月不超过35小时;

(五)用工期间,学生如无重大失职行为或明显不适应工作情况,中途不得随意解聘学生,每学年末自行解除聘任。各用工单位用工期内需要调整的上岗学生,需在当月20日前报学生工作部(处),经批准后,由各用工单位按计划公开招聘符合应聘条件的学生;

(六)进入期末考试阶段,用工单位应暂停学生的勤工助学工作。如有特殊情况留用勤工助学学生,须向学生工作部(处)提出书面申请,批准后方可在岗并核发相应的酬金,否则由用工单位承担学生酬金;

(七)毕业生原则上在离校之前2个月内(即5月1日以后)不得再从事校内勤工助学,各用工单位应在4月中旬将毕业生离岗人数及时报学生工作部(处),由学生工作部(处)负责公开发布招聘信息,招聘人员;

(八)对因参加勤工助学活动而影响专业学习和违反勤工助学协议的学生,可按照协议停止其勤工助学活动。对在勤工助学活动中违反校纪校规的,按照学校相关管理规定进行教育和处理;

(九)各用工单位的勤工助学岗位必须接受勤工助学检查组的检查,并积极配合有关的工作协调等;检查中如出现学生缺岗、空岗等情况,将视情况扣发酬金;如出现不配合检查等情况,将由用人单位自筹经费负责酬金的发放。

第八条 酬金考核及酬金发放

(一)固定岗位酬金标准由学生工作部(处)根据岗位的工作时间、劳动性质、劳动强度等条件核定,并根据具体情况适当调整,酬金标准为260-300元/月;

(二)临时岗位酬金标准原则上在6小时内按照8-10元/小时计算,工作时间超过6小时,则按照48-60元/天计算,每人每月临时用工酬金不得超过300元;

(三)岗位酬金每月发放一次,每月最后一个工作日前,各用工单位根据审定的岗位酬金标准和学生劳动考勤情况,填写《勤工助学酬金发放明细表》,并将打印表经用工单位领导和经办人签字盖章,报校学生工作部(处)审核,如逾期视同未报;

(四)用工单位将核定后的酬金和账号以纸质文稿形式告知学生,并经学生签名确认后由用工单位留存备查;

(五)勤工助学专项经费的使用遵守国家财经法规和学校财务制度,用工单位在填写酬金考核明细表时,应按用工的实际情况如实填写,如发现虚报、假报、克扣及其他违纪行为,学校将根据有关规定严肃处理;

(六)学生工作部(处)每月15日前,将学生勤工助学固定岗位酬金核定表报校财务处,通过银行将学生酬金直接划入学生工行校园联名卡账户;

(七)学生工作部(处)每月从学校卡务中心读取学生账号等信息,用工单位每月应及时关注“学工广场助学方舟勤工助学”公布的酬金发放未成功的学生名单,并通知学生必须到学校卡务中心将工行灵通卡与校园卡关联,学生工作部(处)不受理用工单位和学生个人更新卡号;

(八)设在学校机关、学院办公室的勤工助学岗位和直属单位部分勤工助学岗位酬金由学生工作部(处)从勤工助学专项资金全额支付;其它校内勤工助学岗位酬金的50%由用工单位自筹,通过财务处转账划拨到学校勤工助学专项资金,另50%由学生资助管理中心从勤工助学专项资金支付。

第九条 用工单位考核

(一)用工单位负责老师应按要求到现场组织面试招聘学生,不得缺席,否则视为放弃本年度用工计划;

(二)用工单位因工作不到位导致学年内出现两次及以上学生酬金不能正常发放,核减下一学年该用工单位勤工助学岗位计划数的10%。

第十条 本办法由学生工作部(处)负责解释。

第十一条 本办法自颁布之日起执行。

\chapter{武汉理工大学全日制普通本科学生社会助学金评选办法}
第一章 总 则

第一条 为规范和完善我校社会助学金的评选与管理,充分发挥社会资助在育人方面的积极作用,帮助家庭经济困难学生顺利完成学业,培养学生全面成才,制定本办法。

第二条 社会助学金是指社会团体、企事业单位和个人在我校捐赠设立的,专门用于资助家庭经济困难学生的学习和生活,以帮助他们顺利完成学业的专项助学金。

第三条社会助学金的资助对象是我校全日制普通本科生中的家庭经济困难学生。

第四条 受武汉理工大学教育发展基金会委托,由学生工作部(处)统一负责社会助学金的评选。

第二章 申请条件与资助标准

第五条 申请助学金的基本条件

(一)热爱社会主义祖国,拥护中国共产党的领导;

(二)遵守宪法和法律,遵守学校规章制度;

(三)诚实守信,道德品质优良;

(四)勤奋学习,积极上进,成绩合格;

(五)家庭经济困难,生活俭朴,并已建立家庭经济困难学生档案;

(六)积极参加集体活动和公益活动,认真履行协议规定的相关义务;

(七)符合资助方的其他要求。

第六条 社会助学金的资助标准由资助方和学校共同商定。

第七条 同一名家庭经济困难学生一学年内接受各种资助的总额原则上不得超过8000元,资助项目原则上不得超过两项。

第三章 申请和评审

第八条 社会助学金的评审工作坚持公平、公正、公开的原则。

第九条 社会助学金的评选严格按助学金捐赠方要求进行,资助名额与金额不得拆分。

第十条 社会助学金按学年评审,每年9—11月集中进行申请和评审。

第十一条 学生工作部(处)和武汉理工大学教育发展基金会拟定相应的评选通知在校内公布,各学院做好宣传和申报工作,保证资助信息的公开和透明。

第十二条 符合社会助学金申请条件的学生可向所在学院提出申请,并填写助学金申请表,如实说明申请理由及家庭经济状况,提供上学年成绩单(经学院认定并加盖公章)。

第十三条 辅导员或班主任组织助学金班级评议小组,对申请材料进行核实,经班级民主评议后,上报学院,学院资助工作组根据申请学生的家庭经济情况及平时表现进行评审,确定初选名单,在学院内公示无异议后,报学生工作部(处)。

第十四条 学生工作部(处)对通过初审的学生进行审核,报学校学生奖助工作领导小组审定后,将最终受助学生名单发送给资助方。

第四章 发放与终止

第十五条 社会助学金由教育发展基金会直接划拨入学生的银行卡中。

第十六条 被资助对象填写一式三份《社会助学金申请表》,分别送资助单位、交教育发展基金会存档、交学生工作部(处)存档,便于跟踪管理。

第十七条 被资助对象如受到警告及以上行政处分或学年学习成绩有两门主修专业必修课不合格者,取消资助资格。

第五章 附 则

第十八条 本办法由学生工作部(处)负责解释。

第十九条 本办法自公布之日起执行。

\part{日常管理与课外活动}
\chapter{武汉理工大学学生社团管理办法}
第一章  总则

第一条  为进一步加强学校学生社团管理,深化学校育人功能,积极促进学生社团健康发展,依照国家有关规定,特制定本办法。

第二条  本办法所称学生社团,是指由学校学生依据兴趣爱好自愿组成,由学校登记注册,按照章程自主开展活动的群众性学生组织。

第三条  学生社团须遵守宪法、法律、法规、党的路线方针政策和国家各级相关规定以及校纪校规,积极践行和弘扬社会主义核心价值观,不得损害国家、社会、集体的利益和其他公民合法权益。

第四条  学生社团的基本任务:遵循和贯彻党的教育方针,坚持立德树人的基本导向,团结和带领广大同学,按照自愿、自主、自发原则,开展主题鲜明、健康有益、丰富多彩的活动,繁荣校园文化,培养同学的社会责任感、创新精神和实践能力,提升同学的综合素质,促进同学成长成才。



第二章  学生社团的管理机构

第五条  校党委统一领导学校学生社团工作;校团委履行学校学生社团工作的管理职能,承担学生社团的成立、年审、变更、注销、活动管理、经费管理等工作。校级社团业务指导单位负主体责任,单位主要负责人为第一责任人;院级社团业务指导学院党组织负主体责任,学院党组织书记为第一责任人。

第六条  校学生会要在校内学生组织中发挥枢纽性作用,配合校团委加强对学生社团的引导、服务和联系。校学生会须明确1名主席团成员负责学校学生社团工作。



第三章  学生社团的成立、年审、变更和注销

第七条  学生社团一般按照思想政治、学术科技、创新创业、文化体育、志愿公益、自律互助等进行分类,由发起者向校团委申请登记审批。提交申请材料的时间为每年5月至6月,校团委统一审批并于9月进行公示。

(一)成立学生社团,应具备下列条件:

1、有5名及以上的学生联合发起,发起人须是具有正式学籍的学生,应具有开展该社团活动必备的基本素质、无不及格课程、无违反法律法规或校纪校规的行为;

2、有规范的社团名称和相应的组织机构,学生社团的名称应符合法律、法规和相关规定,不得违背校园文明风尚;学生社团名称应与其性质相符,准确反映其特征;学生社团名称中应冠有“武汉理工大学”字样;

3、有业务指导单位,有至少1名社团指导教师。社团指导教师必须是本校在职教职工,具备较强的思想政治素质、组织管理能力和指导学生社团开展活动的专业能力;

4、有规范的社团章程。

(二)申请材料应包括:

1、社团成立申请书;

2、社团章程,包括:①社团名称,②宗旨、活动内容和形式,③社团会员资格及其权利、义务,④日常活动、经费等管理制度,⑤组织机构产生程序及权限,⑥社团干部当选资格、权限和任免程序,⑦章程修改程序,⑧社团终止程序,⑨应当由章程规定的其他事项;

3、社团成员基本情况;

4、业务指导单位确认书 、社团指导教师基本情况。

(三)批准成立的学生社团,应自批准成立之日起30个工作日内召开全体会员大会,产生执行机构和主要负责人。

(四)有以下情况者不予成立社团:

1、社团宗旨、活动内容及范围不符合本办法相关规定;

2、校内已有活动内容和形式相同或相近的学生社团;

3、申请材料中有弄虚作假情况;

4、开展宗教类活动;

5、以企业及校外机构或个人冠名;

6、参与商业性营销活动;

7、具有“老乡会”、“同窗会”等性质。

第八条  校团委每年9月对学生社团进行年审,并集中公示公告社团年审情况:

(一)年审内容应包括社团规模、社团成员构成、年度活动清单、社团指导教师情况、业务指导单位意见、财务状况、有无违纪违规情况等内容;

(二)年审过程要公开、公平、公正;

(三)对年审不合格的社团,要提出整改,整改期间,社团不得开展除整改以外的其他活动。

第九条  学生社团的登记事项、备案事项需要变更的,应向校团委提交书面申请,由校团委审核批准;学生社团修改章程,应向校团委提交书面申请,由校团委审核批准。

第十条  学生社团有下列情况应予以注销:

(一)违反学生社团管理办法和社团章程;

(二)会员大会决议解散;

(三)未进行年审或年审不合格且整改无效;

(四)1年之内未开展活动;

(五)无社团指导教师、业务指导单位;

(六)由于其他原因终止。



第四章  学生社团的活动管理

第十一条  学生社团举办活动须遵守学校相关规章制度,并按照相应的审批程序进行,不得在学生中散布违背宪法、法律、法规和党的路线方针政策的错误观点和言论,不得开展与其宗旨不符的活动,不得开展纯商业性活动。

第十二条  校团委每年10月统一组织学生社团招新。

第十三条  学生社团会员大会每学期应至少召开1次,社团指导教师应现场指导。

第十四条  会员大会行使下列职权:

(一)修改社团章程;

(二)审议社团工作报告;

(三)审议社团经费使用情况;

(四)对社团变更、注销等事项做出决定;

(五)选举学生社团干部。

第十五条  学生社团的印章和旗帜,由校团委统一制作和管理。

第十六条  指导教师每月至少指导学生社团开展1次活动。

第十七条  学生社团举办活动,须提前向社团指导教师、业务指导单位递交活动申请,经批准同意后提前1周向校团委报备。

第十八条  学生社团邀请校外人士出席活动,须经社团指导教师、业务指导单位和校团委同意,必要时须报校党委宣传部门或相关职能部门批准。

第十九条  学生社团组织、参与校外活动,须提前向社团指导教师、业务指导单位递交活动方案,经批准同意后提前3周填写《武汉理工大学学生社团校外活动申请表》,提交校团委,审批同意后方可开展。活动全程须由社团指导教师现场指导。

第二十条  社团创办刊物应向校团委提交书面申请,详细说明刊物名称、主要内容、出版周期、发行对象、字数及发行份数等内容,经批准后方可出版发行。校团委有权对违反规定的社团刊物进行整改和停刊。

第二十一条  学生社团新媒体平台的用户名(账号)应向校团委登记备案。



第五章  学生社团的经费管理

第二十二条  学生社团的活动经费坚持学校支持与自筹相结合、自筹为主的原则。学生社团接受捐赠和资助,应向校团委报告接受、使用捐赠和资助的有关情况,并向全体社团会员公开。

第二十三条  学生社团如收取会费,须根据实际情况明确收费标准,经社团内部民主决策,报校团委审核后进行公示,并应写入社团章程。

第二十四条  学生社团应制定严格的经费管理制度,社团经费应做到专人保管,钱账分离,每项费用支出至少3人签字,原则上为正(副)会长、财务负责人、经办人。学生社团每学期应向全体会员公布经费使用情况,校团委在换届前对学生社团进行财务审核。

第二十五条  已注销的学生社团,应由校团委进行经费使用情况清查,剩余经费返还社团会员。



第六章  学生社团会员的权利义务

第二十六条  社团会员有权了解所在社团的章程,对学生社团的管理和活动提出建议。

第二十七条  社团会员享有选举权、被选举权。

第二十八条  社团会员有权向校团委反映所在社团运行中出现的问题。

第二十九条  社团会员应积极参加学生社团的各项活动,为学生社团的发展献计献策,促进学生社团的健康发展。

第三十条  社团会员应定期在社团注册。



第七章  学生社团的奖惩

第三十一条  校团委每年对学生社团开展星级评定、优秀社团个人评选工作。

第三十二条  学校设立学生社团活动经费,每年对学校精品社团进行资助。

第三十三条  学生社团有下列情况之一者,校团委有权责令其停止活动,情节严重的,责令解散:

(一)社团活动违反法律法规或校纪校规;

(二)发展宗教组织,开展宗教活动;

(三)不接受校团委管理;

(四)活动期间发生重大事故;

(五)未经批准擅自以学生社团名义开展活动;

(六)社团会员信息丢失或泄露;

(七)经费使用情况不明或记录丢失;

(八)私自刻制社团印章或制作旗帜;

(九)其他应当进行整顿的行为。

第三十四条  有下列情况之一者,不得担任学生社团干部:

(一)违反法律法规或校纪校规;

(二)有不及格课程;

(三)被责令解散社团的干部;

(四)其他不宜担任社团干部的情况。



第八章  附则

第三十五条  本办法由校团委负责解释。

第三十六条  本办法自发布之日起执行。  

\chapter{武汉理工大学青年志愿服务管理办法}
第一章  总则

第一条 为了弘扬奉献、友爱、互助、进步的志愿精神,推进建设和谐社会的进程,鼓励、引导和规范我校青年志愿者服务活动,维护青年志愿者、志愿服务组织及志愿服务对象的合法权益,推动我校志愿服务事业的健康发展,根据相关法律、法规,结合我校实际,特制定本办法。

第二条 武汉理工大学的青年志愿者、青年志愿者组织及其相关志愿服务活动适用本办法。

第三条 本办法所称志愿服务,是指不以物质报酬为目的,自愿以自己的时间、知识、劳动和技能等为他人和社会提供帮助和服务的公益性活动。

本办法所称青年志愿者,是指经个人申请、我校各级青年志愿者组织登记注册,参加志愿服务活动的青年。

本办法所称青年志愿者组织,是指从事青年志愿服务的非营利性的学校公益组织,包括校院两级青年志愿者协会及其下设的青年志愿服务队。

第四条 志愿服务应当遵循自愿、无偿、平等、诚信、合法的原则,不得违背社会公德,青年志愿者应将志愿者誓词铭记在心,践行到底。

志愿者誓词:我愿意成为一名光荣的志愿者。我承诺:尽己所能,不计报酬,帮助他人,服务社会,践行志愿精神,传播先进文化,为建设团结互助、平等友爱、共同前进的美好社会贡献力量。

第五条 学校共青团组织负责对青年志愿者组织及其志愿服务活动进行规划组织、指导协调以及监督检查等。

第六条 学校共青团组织应当制定和完善青年志愿服务的支持和保障政策,合理安排志愿服务所需物资,引导和促进志愿服务事业的发展。

第七条 学校共青团组织以及青年志愿者组织应当建立志愿服务信息平台,为志愿者注册、培训以及志愿服务信息发布等活动提供便利条件,实现志愿者、志愿服务组织、志愿服务项目、志愿服务需求的对接。

 

第二章  志愿者

第八条 青年志愿者应当具备相应民事行为能力以及与其从事的志愿服务相适应的知识、技能和身体条件等。

第九条 青年志愿者享有下列权利:

(一)自愿加入或者退出志愿服务组织;

(二)自愿参加志愿服务活动,有权选择与自己行为能力相适应的活动进行志愿服务;

(三)接受志愿服务活动所需的知识和技能培训;

(四)接受志愿服务活动所提供的相应补助;

(五)请求青年志愿者组织和有关单位、组织帮助解决在志愿服务中遇到的困难和问题;

(六)对青年志愿服务工作提出建议、意见,并进行监督;

(七)自身有特殊困难时优先获得志愿服务;

(八)法律、法规以及志愿服务组织章程规定的其他权利。

第十条 青年志愿者履行下列义务:

(一)遵守志愿服务组织的章程和管理制度;

(二)履行志愿服务承诺,完成志愿服务活动;

(三)尊重志愿服务对象的合法权益,保护志愿服务对象的隐私和商业机密;

(四)维护青年志愿者组织和青年志愿者的声誉和形象,不得从事违反法律或者违背社会公德的活动;

(五)法律、法规规定的其他义务。

第十一条 青年学生可以在志愿服务组织的指导下成为注册志愿者,从而可以参与相关的志愿服务活动。

 

第三章  志愿服务组织

第十二条 我校二级学院可以在学院共青团组织指导下建立院级青年志愿者协会、特色志愿服务队,协会、服务队成立需学院党委同意并经校团委批准。

青年志愿者协会按照协会章程负责本协会青年志愿服务活动、院级志愿服务队的组织、协调、管理和指导,维护青年志愿者的合法权益,建立和完善青年志愿服务绩效评价和激励机制。

第十三条 青年志愿服务组织在招募志愿者时,应当公告志愿服务项目、服务内容、志愿者需要达到的条件和所需志愿者的数量等信息,告知志愿服务过程中可能出现的情况或遇到的问题。

第十四条 青年志愿者组织应当为志愿者的青年志愿服务活动提供必要的经费支持和服务保障。青年志愿者组织在组织志愿服务活动时应该对青年志愿者进行相关的安全教育,为服务过程中发生意外的青年志愿者及时提供援助,对可能出现安全风险的志愿服务活动,应当根据实际需要,为志愿者办理相应的人身保险。

 

第四章  志愿服务活动

第十五条 志愿服务的范围包括扶贫济困、支教助学、科技推广、文化体育、医疗卫生、法律服务、环境保护、心理咨询、社区事务、文明劝导、应急救援、大型社会活动等。

青年志愿服务的重点对象应当是残疾人、老年人、未成年人、优抚对象、城镇下岗失业人员、农村特困人口和其他有特殊困难需要帮助的社会成员以及大型社会公益活动。

第十六条 青年志愿者、青年志愿者组织与服务对象之间是自愿、平等的服务与被服务关系,应当相互尊重、平等对待。

第十七条 青年志愿者在开展志愿服务活动时应当佩戴统一的青年志愿者标志,遵循志愿服务的原则,注重服务质量,向志愿服务对象提供优质的志愿服务。

第十八条 任何组织和个人不得强行指派青年志愿者或者青年志愿组织提供志愿服务,不得利用志愿服务的名义从事违法活动或者营利性活动。

志愿服务组织和志愿者不得向志愿服务对象收取或者变相收取费用。

 

第五章  保障与激励

第十九条 组织开展青年志愿服务的经费来源:

(一)学校划拨经费;

(二)依法获得的社会资助和捐赠;

(三)基金会资助;

(四)其他合法收益。

第二十条 公民、法人和其他组织对我校各级青年志愿者组织资助和捐赠的资金、物资,由校团委接收、登记,并划拨具体组织,专项用于青年志愿服务活动。

资助和捐赠的资金、物资的使用应当尊重资助和捐赠者的意愿,坚持公开的原则并加强对志愿服务活动经费的管理和监督,任何组织和个人不得私分、挪用或者侵占。

第二十一条 全校师生应当倡导开展志愿服务活动,将志愿服务纳入学生思想政治教育体系,不断提高志愿服务在个人成才中的重要性。校院两级党团组织应对表现突出的青年志愿者、青年志愿者组织,以及在帮助、支持青年志愿服务中有突出贡献的组织和个人,给予表彰和奖励。

第二十二条 校团委及校青年志愿者协会会每年定期开展志愿者表彰活动,根据志愿者服务时数、服务质量、工作业绩等评选优秀志愿者服务集体和优秀志愿者;并推荐特别突出的志愿者组织和志愿者参加上级志愿者组织的评选表彰活动。

 

第六章  法律责任

第二十三条 对以我校青年志愿者组织名义、标志从事以营利为目的的经营性活动和其他违法活动的,我校将依法追究其相应的法律责任。

 

第七章  附则

第二十四条 本办法自发布之日起实行。
\chapter{武汉理工大学学生军事训练工作实施意见}
第一章 总 则

第一条 为了加强国防教育,科学规范学生军事训练(以下简称“学生军训”)工作,依据《中华人民共和国兵役法》、《中华人民共和国国防法》、《中华人民共和国国防教育法》的有关规定,以及教育部、总政治部、总参谋部联合颁发的《普通高等学校军事课教学大纲》、《学生军事训练工作规定》的精神和要求,贯彻落实教育部等部门《关于进一步加强高校实践育人工作的若干意见》(教思政〔2012〕1号)精神,结合我校实际,制定本实施意见。

第二条 开展学生军训工作是国家人才培养和国防后备力量建设的重要措施,是大学生接受国防教育的基本形式,是学校教育和教学的一项重要内容。

第三条 学生军训严格执行“三个结合、一个确保”的工作原则,即“把严格要求与关心爱护相结合,把艰苦紧张的军事训练与耐心细致的思想政治教育相结合,把拥军与爱民相结合,确保军训工作的绝对安全”。

第四条 学生军训所需教官和枪支弹药由学校武装部向湖北省军区提出申请计划,由湖北省军区学生军训办公室协调保障。

学生军训所需经费纳入学校年度经费预算管理予以保障。

学生军训期间,学校应满足军训场地要求,优先为学生军训提供场地,为实施规范化的军事技能训练提供条件。

第五条 学生军训是在校学生的必修课程,每名学生应按要求完成相应的学分。

第二章 组织领导与实施

第六条 学生军训工作由学校党委和行政领导,学校国防教育领导小组统一指挥,学校武装部组织协调,各军训团具体实施。

学校国防教育领导小组由分管武装工作的校领导担任组长,分管教学工作和分管后勤工作的校领导任担副组长。成员包括武装部、校区管委会等有关部门负责人。领导小组办公室设在武装部。

校区军训工作小组由校区管委会和学校相关部门负责人及承训部队带队教官组成。

军训团在军训开始前由学校国防教育领导小组负责组建,并在学生军训动员大会上以开训命令予以公布。

第七条 武装部作为学生军训工作职能部门,负责学生军训的前期准备,实施方案与计划制定,日常训练、科目考核、总结阅兵组织等,会同学校有关部门共同完成军训任务。

第八条 学生军训是国防教育的重要组成部分,学校专职军事理论课教师必须参与学生军训全过程。其教学工作量按照教务处相关规定进行核算。

第九条 学生军训实行军地协同,各司其职。团长、连长、排长和团参谋长由承训部队教官担任,负责组织军事科目训练;政委、副团长、副政委、政治部主任、指导员、副指导员等由学校人员担任,负责军训学生思想政治工作和后勤保障工作。

第十条 积极组织形式多样、丰富多彩的宣传教育和养成教育活动,开展扎实有效的思想政治工作,为军训学生保持高昂战斗士气营造良好氛围。

第十一条 建立军训例会制度和讲评制度。军训团每天召开各连副指导员例会,了解并讲评训练情况,通报问题及注意事项,布置工作任务;军训团团长在每单元训练后,集合连、排长讲评训练情况,落实训练进度,改进训练方法。

第十二条 健全军训各项规章制度,强化教育,科学施训,确保军训工作的绝对安全。

开训前,召开新生辅导员(连副指导员)培训会,学习军训工作条例条令及各项安全制度。

训练中,在参训学生中设立连、排、寝室三级安全员,根据学生身体状况及天气情况调整训练科目和训练场地,加强学生训练场地安全管制。

实弹射击前,军训团制定实弹射击安排和交通运输方案,落实交通指挥与靶场安全措施,强化学生进入靶场和靶位、击发过程中的安全程序。

第十三条 狠抓作风养成和一日生活制度教育,开展内务卫生、宣传报道、日常综合管理评比和队列、分列式、军体拳、轻武器射击等军事科目的考核。

第十四条 认真抓好军训总结表彰和阅兵式汇报表演,检验和展示学生军训成果。对表现优秀的连队及个人进行表彰,激发全体参训人员的集体观与荣誉感。

第三章 时间、内容、学分及成绩管理

第十五条 军训一般在新生入学第一学年进行,集中训练时间为2-3周,考核合格者记1.5个学分。

第十六条 军训以《普通高等学校军事课教学大纲》为依据,主要进行中国人民解放军条令条例教育与训练、军体拳训练、轻武器射击训练、中国人民解放军优良作风和优良传统教育、安全知识等教育。

第十七条 军训考核按照平时考核与集中考核相结合的原则,主要对思想素质、队列训练、实弹射击、日常管理四个科目进行考核。

考核办法:队列训练、射击训练的考核以平时考核和集中训练考核各占50%计算,由承训部队主考,各连副指导员配合;思想素质、日常管理的考核以平时检查为主,主要由各连副指导员组织考核,连长配合。

军训总成绩计算方法:总成绩=队列成绩$\times$40%+射击成绩$\times$20%+日常管理$\times$40%。

第十八条 按照学生军训计划和要求完成学生军训各项科目训练,成绩评定在合格以上的学生即可获得军训学分。成绩不合格的学生不能取得军训学分,随翌年学生补训。

第十九条 做好学生军训成绩记载。参训学生应认真填写《武汉理工大学军训成绩鉴定表》,由各连队签字,学院及武装部审核盖章。学生成绩由武装部集中上载到教务处。《武汉理工大学军训成绩鉴定表》转交档案馆载入参训学生档案。

第四章 组织纪律

第二十条 凡参加军训的学生,必须按照计划规定的时间进行训练,无特殊原因不准请假。

因病请假者,须由我校医院出具证明;其他原因请假者,须本人写出申请,由连队副指导员核实后签署意见。请假1天以内,由连批准;请假1天以上,由团批准。

凡未经请假批准或超过假期而不参加训练的学生,除按旷课处理(每天按8学时计算)外,军训成绩以零分计;虽经批准,但病(事)假累计超过5天(含5天),视为军训不合格,随翌年学生补训。

第二十一条 参训学生应遵守纪律,听从教官指挥,严格要求自己。对在军训中有抵触情绪并散布消极言论,不服从组训人员管理,严重影响军训正常进行的学生,其军训成绩以零分计。

第五章 免训、缓训、补训

第二十二条 入学前已经服义务兵役的学生,凭《退伍证》由本人提出书面申请,经审核、批准后可以免训,直接获得军训学分;因患有严重疾病,确实不能参加集中军训的可申请免训,免训者须提出免训申请并由我校医院出具证明,经军训团、所在学院领导签署意见后,报武装部备案,由武装部对其组织军事技能知识测试。

第二十三条 因病或其它原因不能在规定时间参加集中军训的学生,可申请缓训。因病缓训的学生需由我校医院出具证明,因其它原因缓训的学生由本人提出申请并经所在学院领导签署意见后,报武装部备案。

第二十四条 缓训原因消除或当年参加军训未合格者必须补训。如病愈需持我校医院证明,由本人提出申请并经所在学院领导签署意见后,报武装部审核批准,随翌年学生补训。

第六章 附 则

第二十五条 本规定由学校武装部负责解释。

第二十六条 本规定自发布之日起执行。

附件:武汉理工大学学生军事训练免训、缓训、补训申请表

(注:请将申请证明材料附后)

\chapter{武汉理工大学学生应征入伍工作管理办法}
征集地方大学生应征入伍,对于推进国防和军队现代化建设,促进青年学生成长成才具有重要意义。根据上级组织相关文件精神,结合我校实际,制定本办法。

一、组织与领导

在学校国防教育领导小组的统一领导下,由武装部具体负责组织实施,学生工作部(处)、教务处、计划财务处、保卫处、宣传部等有关单位协助,共同完成学生应征入伍工作。

二、征集对象

征集对象为我校普通本科生和研究生。征集年龄为18至24周岁。

三、征集程序

(一)应届毕业生入伍预征

1.应届毕业生入伍预征工作,通常在每年5月-7月毕业生离校之前进行。

2.有应征意愿的应届毕业生在预征工作启动后,登陆“大学生网上预征报名系统”(http://zbbm.chsi.com.cn或http://zbbm.chsi.cn)报名预征,持学生证和身份证到学校武装部登记备案。符合基本征集条件的学生确定为预征对象,并填写《应届毕业生预征对象登记表》。

3.确定为预征对象的应届毕业生,离校时将户口迁回入学前户籍所在地,冬季征兵工作开始前持《应届毕业生预征对象登记表》到入学前户籍所在地县级兵役机关报名应征。

4.离校前未参加预征报名的应届毕业生,如直接报名应征入伍,可返回学校武装部补办预征手续。

5.通过正式体检、政审并符合其他征集条件的,由县(市、区)人民政府兵役机关批准入伍。入伍学生应及时将批准入伍的信息反馈到学校武装部。

(二)在校学生应征入伍

1.在校学生应征入伍工作,通常在每年10月-12月进行,有应征意愿的学生在征兵工作启动后,持学生证、身份证和6张1寸蓝底免冠近照到学校武装部报名。

2.学校武装部协助地方人民政府兵役机关,对应征者进行体检和政审。

3.通过体检、政审并符合其他征集条件的学生,经地方人民政府兵役机关批准入伍。

4.批准入伍的学生,持学校武装部出具的应征入伍证明到所在学院办理保留学籍手续。

四、优扶政策

(一)学籍管理

1.被批准入伍的在校学生,入伍时已经修完规定课程或已修满规定学分,符合毕业条件的,学校可准予毕业发给其毕业证书,并予以电子注册。对未达到修业年限或尚不具备毕业条件的,可保留学籍至退出现役后2年内。

2.在校学生入伍前,学校应尽可能安排其参加本学期所学课程的考试,也可以根据其平时的学习情况,由学生本人提出书面申请,对本学期所学课程进行面试或免试,确定成绩和学分。

3.服役期满退出现役的学生,应在保留学籍有效期内到原所在学院办理复学手续,原则上回原专业复学,逾期不予复学。

4.在部队服役期间荣立三等功及以上奖励的学生,复学后可选择其它专业学习。

5.退役后3年内参加硕士研究生入学考试的学生,其入学考试初试成绩总分加10分。在部队服役期间荣立二等功及以上奖励的,本科毕业后,可免试推荐攻读硕士研究生。

6.复学期间经学生本人申请可免修公共体育、军事技能和军事理论课程,直接获得相应学分。

(二)补偿(资助)学费和代偿国家助学贷款

1.对应届毕业生在校期间和在校生入伍前缴纳的学费实行补偿,对入伍的在校生复学后实行学费资助。每学年补偿(资助)学费或代偿国家助学贷款本息的金额,最高不超过6000元,对于每学年实际缴纳的学费或获得的国家助学贷款本息低于6000元的,按两者就高的原则,实行补偿或代偿。

2.在校期间获得国家助学贷款的学生,其学费补偿款必须首先用于偿还国家助学贷款本金及全部偿还之前产生的利息。

3.在校期间已享受免除全部学费的学生,定向生、委培生、国防生、部队招收的大学毕业生干部,以及从高校毕业生中直接招收的士官等其他不属于服义务兵役到部队参军的学生不包括在内。

4.被确定为预征对象的应届毕业生需向学生工作部(处)提交《应征入伍高校毕业生补偿学费代偿国家助学贷款申请表》(一式两份),如获得国家助学贷款的,还需提供与国家助学贷款经办银行签订的毕业后还款计划书复印件。经学校学生工作部(处)审核确认无误后,于当年6月30日前,将上述申请表及《应届毕业生预征对象登记表》一并发给学生本人。预征学生于当年10月31日前到户籍所在地报名应征时,将上述材料交户籍所在地县级兵役机关。

5.应征入伍的在校学生应向学生工作部(处)提交《应征入伍高校在校生补偿学费代偿国家助学贷款申请表》(一式两份)。在校期间获得国家助学贷款的学生,还需提供与国家助学贷款经办银行签订的一次性还款计划书,其中应注明已申请国家助学贷款代偿;在校期间获得生源地信用助学贷款的高校在校学生,还需向学校提交本人签字的一次性还款计划书,其中应注明代偿经费将直接由学校学生工作部(处)向银行一次性代为偿还贷款。经学生工作部(处)审核确认无误后,将《应征入伍高校在校生补偿学费代偿国家助学贷款申请表》交给学校武装部。由武装部统一转交至地方政府兵役机关办理相关手续。

6.应征应届毕业生或应征在校生被批准入伍后,于次年1月15日前,将《应征入伍通知书》复印件及加盖当地兵役机关公章的《应征入伍高校毕业生(在校生)补偿学费和代偿国家助学贷款申请表》(一式两份)原件,寄送至学校学生工作部(处),由学生工作部(处)将上述材料报全国学生资助管理中心审核。

7.应征入伍的在校学生退役复学到学校报到后,向学校学生工作部(处)提出学费资助申请,并填写提交《应征入伍高校复学学生资助学费申请表》和退出现役证书复印件。学生工作部(处)会同武装部对学生申请资格进行认定,确认无误后,由学生工作部(处)将上述材料报全国学生资助管理中心审核。

8.学校在收到补偿学费和代偿国家助学贷款本息资金后,由学生工作部(处)会同计划财务处向毕业生(复学学生)补偿学费或代替毕业生(复学学生)按照还款协议向银行偿还国家助学贷款本息。

(三)就业

1.学校为毕业生退役士兵提供就业信息,重点推荐就业岗位,提供就业指导服务。

2.通过预征入伍的应届毕业生,在退出现役当年12月1日到次年12月31日期间就业的,学校按应届毕业生办理就业手续。

五、管理与监督

1.因本人思想原因、故意隐瞒病史或违法犯罪等行为造成退兵的毕业生或原在校生,取消补偿学费或学费资助和代偿国家助学贷款资格。

2.被部队退回的毕业生,其已补偿的学费或代偿的国家助学贷款本息资金由毕业生户籍所在地县(市、区)教育行政部门会同同级人民政府征兵办公室收回。

3.被部队退回并取消补偿代偿资格的原在校生,如退回学生返回其原户籍所在地,其已补偿的学费或代偿的国家助学贷款由学生户籍所在地县(市、区)教育行政部门会同同级人民政府征兵办公室收回;如退回学生返回学校,其已补偿的学费或代偿的国家助学贷款由学校学生工作部(处)收回,并上缴全国学生资助管理中心。

4.未经学校批准私自通过社会或其他途径应征入伍的学生,不享受学费补偿(资助)和代偿国家助学贷款优待政策。

5.服役期间,受除名、开除军籍及以上处分的不享受任何优待政策。

\chapter{武汉理工大学学生安全管理规定}
第一章 总 则

第一条 为了加强学校学生安全管理,维护正常的教学和生活秩序,保障学生人身财产安全,促进学生身心健康发展,根据教育部《普通高等学校学生管理规定》、《学生伤害事故处理办法》和《湖北省高校学生非正常死亡事故(案件)应急处置办法(暂行)》,结合学校实际,特制定本规定。

第二条 学生安全管理的主要任务是:宣传、贯彻国家有关安全管理工作的方针、政策、法律法规及我校安全管理工作相关规定,对学生实施安全教育及管理,预防并妥善处置各类安全事故,引导学生健康成长。

第三条 坚持预防为主、保障安全的方针,本着保护学生、教育先行、明确责任、教管结合、实事求是、妥善处理的原则,做好学生安全教育和管理。

第四条 学校实行安全工作目标责任制和责任追究制。建立和健全安全管理规章制度,加强防范,严格管理。

第五条 学校全体教职工要树立“以人为本,服务学生”的思想,关心、爱护学生,努力做好本职工作,保护学生人身和财产安全。

第六条 本规定所称学生为我校在册就读的本科生、研究生。

第二章 安全教育

第七条 加强学生安全教育的组织领导,把安全教育作为经常性工作,常抓不懈。学校各部门要相互配合,积极开展法制教育和安全教育,普及安全知识,增强学生的安全意识和法制观念,提高防范能力。

第八条 组织学生学习大学生安全知识,根据环境、季节及有关规律进行防火、防盗、防骗、防灾害、防意外事故等方面的教育,使之经常化、制度化。

第九条 学生安全教育应根据不同专业、不同年级和不同校区环境等特点,在各种教学活动和日常生活中特别是在节假日前,采取多种形式、有重点地对学生进行安全教育,防患于未然。

第十条 学院(部)要加强学生的心理健康教育,注重心理疏导,教育学生保持健康的心理状态,帮助学生克服各种心理障碍。

第三章 安全管理

第十一条 各学院(部)要做好学生日常安全管理工作,加强安全防范,建立和健全安全管理规章制度,严格管理。

第十二条 各学院(部)应组织学生志愿者参与校园安全公益活动,充分发挥学生在安全管理中自我教育、自我管理、自我约束、自我服务的作用。

第十三条 学生应自觉遵守公民道德规范,严格遵守国家法律法规及学校规章制度,维护学校公共秩序和校园稳定。

(一)学生不得在校园内张贴、散发危害学校正常秩序的标语、传单和大、小字报;不得利用网络传播、散布非法信息,不得登录非法网站浏览色情、暴力网页;不得组织、参与未经批准的游行、示威、集会等活动。

(二)学生不得参与非法传销活动;不得进行邪教、封建迷信;不得在校园内进行宗教活动;不得有酗酒、打架斗殴、赌博、吸毒和传播、复制、贩卖非法书刊和音像制品等违反治安管理法律法规的行为。

(三)学生应自觉遵守教学楼、图书馆、运动场(游泳池)、礼堂、食堂等公共场所管理制度,服从管理。

(四)学生在校内组织大型活动,应按《武汉理工大学大型活动安全管理规定》申报、审批,保证活动安全。

第十四条 学生应严格遵守学生宿舍管理规定,自觉维护学生宿舍的正常秩序,爱护公共设施,保护人身、财产安全,防止各种事故发生。

(一)学生应遵守门卫管理制度,服从宿舍安保人员管理,配合相关部门进行安全检查。

(二)学生应妥善保管好个人贵重物品,离开宿舍或休息时要锁好房门。学生离校时,应将贵重物品带走或委托他人保管,防止被盗。

(三)学生要遵守安全用电制度,严禁私拉电线、使用大功率电器,严禁在宿舍内使用明火,焚烧物品,防止发生火灾事故。

(四)严禁将易燃易爆、有毒等危险品带入学生宿舍。

(五)宿舍区严禁饲养宠物和从事商业活动。

第十五条 学生应自觉遵守《武汉理工大学校园道路交通安全管理规定》,防止发生交通事故。

(一)学生不得在校园道路上使用滑板、旱冰鞋等滑行器具,不得从事有碍交通安全的活动;不得移动、损毁交通标志。

(二)学生在校园内驾驶机动车辆须遵守交通法规,按交通标志缓速行驶;严禁无证驾驶;严禁驾驶无牌无证机动车;使用非机动车辆要依规行驶;所有车辆都要在指定地点停放。

(三)学生在校外要自觉遵守道路交通安全规规,保障出行交通安全。

第十六条 学生应遵守《武汉理工大学消防安全管理规定》,爱护学校消防设施、器材,积极履行预防火灾、报告火警及协助扑灭初起火灾等义务。

学生应积极参加消防安全知识教育和应急疏散演练活动,提高安全防范意识和自救逃生能力。

第十七条 学生应自觉遵守实验室、实践基地(车间)的安全管理制度,增强实验、实践活动的安全意识,所有实验、实践活动应在教师的指导下进行,确保安全。

学生校外社会实践活动应落实各项安全防范措施,防止发生安全事故。

第十八条 学生发现刑事、治安案件和各类安全事故,应及时向学校保卫部门或公安机关报告,并协助保护现场,救助受伤人员,配合对案件或事故的调查处理。

第四章 安全事故处理

第十九条 学生发生安全事故,依照教育部《学生伤害事故处理办法》、《湖北省高校学生非正常死亡事故(案件)应急处置办法(暂行)》,坚持依法依规、客观公正、合理适当的原则,及时、妥善地处理。

第二十条 学校成立学生安全事故处理领导小组,制定学生安全事故处理应急预案,加强领导,明确责任,积极稳妥地处理学生安全事故。

第二十一条 学生发生安全事故,其事故和责任的认定,依照教育部《学生伤害事故处理办法》的相关规定执行。

第二十二条 学生发生安全事故后,事故处理程序、事故损害的赔偿和事故责任者的处理,按《学生伤害事故处理办法》和《湖北省高校学生非正常死亡事故(案件)应急处置办法(暂行)》处理。

第二十三条 对事故处理不服或持有异议者,可向学校或学校上一级主管部门申诉,由上级部门作出复议决定;当事人对复议决定不服的,可依法向人民法院提出诉讼。

第五章 奖励与处分

第二十四条 学生勇于同各种违法犯罪行为和治安灾害事故作斗争,成绩显著的,由学校给予表彰和奖励;因保护国家财产和他人人身安全,作出突出贡献的报请人民政府授予荣誉称号,并给予相应的待遇。

第二十五条 学生违反国家法律法规、不遵守学校安全管理规定,造成他人人身伤害和公私财产损失,情节轻微的,应赔偿相关损失;构成违法犯罪的,由公安、司法机关依法追究法律责任。同时,学校根据《学生违纪处分条例》给予纪律处分。

第六章 附 则

第二十六条 本规定由校保卫处负责解释。

第二十七条 本规定自发布之日起施行。

\chapter{武汉理工大学校园网管理规定}
第一章 总 则

第一条 为了进一步规范学校校园计算机网络(以下简称校园网)管理,为学校各项工作的开展和师生员工的学习生活提供网络服务与保障,确保校园网安全、稳定运行和健康有序发展,特制定本规定。

第二条 本规定所称校园网指由学校组织建设或管理的计算机网络,是学校信息化建设的基础性平台。凡接入校园网,通过校园网提供和接受网络信息服务的单位和个人均为校园网用户。

第三条 校园网分为教学办公区网络和宿舍区网络。教学办公区网络指在各教学楼、办公楼、实验室、图书馆等公共教学办公区域接入的有线和无线校园网络;宿舍区网络指在学生公寓和教工家属区等区域接入的有线和无线校园网络。

第二章 用户管理

第四条 学校所有在编在册的教职工、学生均可直接成为校园网正式用户。在学校从事短期教学、管理工作或学习的个人凭相关单位证明和个人有效身份证件向网络信息中心申请成为校园网临时用户。

第五条 为合理利用有限的网络带宽资源,保证用户正常的工作和学习需要,杜绝浪费,校园网实行包周期、限流量管理。以用户首次登录校园网为周期的起始,30天为一个使用周期。根据用户身份类别和工作性质分别给予周期限额网络流量。正式用户在规定流量范围内可直接在教学办公区使用校园网;宿舍区网络和临时用户须按相关规定缴纳网络使用费后,方可使用校园网。

第六条 校园网正式用户因毕业、离职等原因离开学校后,账户自动注销。临时用户使用期限到期后,账户自动注销。

第七条 校园网用户的个人隐私权和通信自由受法律保护。除法律另有规定外,任何组织和个人不得侵犯他人的隐私和通信秘密。

第三章 信息服务管理

第八条 学校各二级单位、团体和教职工个人如需通过接入校园网对外提供信息服务,须经党委宣传部审批同意后到网络信息中心办理相关手续。

第九条 通过校园网建立网站并对外提供信息服务的,其网站内容必须与学校的教学、科研、管理、生活服务等工作相关,同时遵守中华人民共和国《互联网信息服务管理办法》。未经允许,任何组织和个人不得利用校园网建站从事与学校师生员工无关的、以盈利为目的的经营性活动。

第四章 运行管理

第十条 校园网线路设备属于国有资产,由网络信息中心负责管理维护。任何单位和个人不得改变校园网网络设备的设置和用途,不得拆卸和损坏网络设施。造成国有资产损失的单位和个人,需照价赔偿。

第十一条 校园网用户有自觉维护校园网安全、正常运行的义务,发现网络运行异常或非法行为和信息,应保护现场痕迹,并及时向网络信息中心和学校有关部门举报和反映,接受并配合国家和学校有关部门依法进行的网络安全监督和检查。

第五章 违规处理

第十二条 校园网用户不得擅自转让用户帐号,不得盗用他人账户,私自设立代理服务器,非法入侵他人计算机系统等。

第十三条 任何用户不得进行下列危害校园网安全的行为:

(一)非法接入校园网或者使用计算机网络资源;

(二)未经允许对校园网的功能进行删除、修改、增加;

(三)未经允许对校园网中存储、处理、传输的数据和应用程序进行删除、修改、增加;

(四)故意制作和传播计算机病毒;

(五)恶意消耗网络通信系统资源,滥用网络资源;

(六)利用校园网进行网络攻击、破坏性操作、窃取他人研究成果和受法律保护的资源;

(七)恶意破坏校园网设施、设备的行为;

(八)其他危害校园网安全的行为。

第十四条 校园网用户在使用校园网时,必须自觉遵守有关国家机密的各项法律规定和有关保护知识产权的各项法律规定,不得进行下列行为:

(一)擅自复制和使用网络上未公开和未授权的文件;

(二)在网络上擅自传播、拷贝享有版权的软件或销售免费共享的软件;

(三)在网络上从事违法犯罪活动;

(四)登录非法网站;

(五)制作、复制、发布、传播含有以下内容的信息:

1.违反宪法所确定的基本原则;

2.危害国家安全,泄漏国家秘密,颠覆国家政权,破坏国家统一;

3.损害国家和学校荣誉和利益;

4.煽动民族仇恨、民族歧视,破坏民族团结;

5.破坏国家宗教政策,宣扬邪教和封建迷信;

6.散布谣言,扰乱社会秩序和正常的教学秩序,破坏社会稳定;

7.散布淫秽、色情、赌博、暴力、凶杀、恐怖、教唆犯罪;

8.侮辱或者诽谤他人,侵害他人合法权益;

9.含有法律、行政法规禁止的其他内容。

第十五条 违反第十三、十四条规定的用户,构成犯罪的,将移交司法机关依法追究刑事责任;尚不构成犯罪的,学校将依照相关规定予以处罚,同时网络信息中心将视情况分别采取暂停用户帐号、闭站整改、关闭网站等技术处罚措施;相关责任人应承担因违规而造成的经济损失,并依法承担相关责任。

第十六条 对本规定中未涉及的其它违规行为,学校有权根据情节轻重作出相应处理决定。

第六章 附 则

第十七条 本规定由网络信息中心负责解释。

第十八条 本规定自发布之日起执行。
\chapter{武汉理工大学学生住宿管理规定}
第一章 总 则

第一条 学生宿舍是学生学习、生活的重要场所,是培养学生文明行为养成的重要阵地。为了加强学生住宿管理,营造文明、和谐、安全的学习与生活环境,特制定本规定。

第二条 学生宿舍的调整和分配,由学生工作部(处)统一安排、调配,财务处收取住宿费。其它单位和部门未经许可不得私自占用学生宿舍和收取住宿费。

第三条 本规定所称学生为我校在校普通本科生、研究生。学生宿舍为学校自有宿舍、社会化公寓和租赁宿舍等。

第二章 入住与退宿

第四条 普通本科生原则由学校统一安排住宿,并按规定缴纳住宿费。研究生报到时可根据实际情况自愿申请住宿学生宿舍或不住宿。

第五条 学生应按入住时指定的房间住宿,入住后不得擅自调换房间,不得出借、出租房间或床位。寒、暑假期间,学校将视学生留校情况统一安排住宿和管理。

第六条 普通本科生因故停止住宿,其住宿费按实际住宿月份收取(不满一月按一月收取,每学年度按十个月计算)。对违反学校规定擅自校外住宿的学生,学校不退住宿费。

第七条 普通本科生如因特殊原因需要调换房间,须由本人提出申请,经所在学院审核,报学生工作部(处)批准后方可调换,调换后的住宿费据实核算。

第八条 普通本科生因下列特殊原因,可申请校外住宿。

(一)患传染性疾病的;

(二)患皮肤病需单独住宿治疗的;

(三)身患残疾生活不能自理需家长陪读的;

(四)其他特殊情况,需单独住宿的。

第九条 符合上述特殊情况需在学生宿舍外住宿的学生需按规定办理相关手续,其处理程序为:

(一)学生个人申请,说明校外住宿的原因、校外住宿详细地址、联系方式、住宿期限等;

(二)患病学生提交县市一级医院证明和学校医院确认证明;

(三)学生提交学习成绩情况说明,以学院教学办出具的学生成绩登记表为准;

(四)其他特殊情况,须提交学校有关部门书面确认;

(五)将由学生本人和学生家长共同签字的申报材料报学院审核;

(六)学院登记并审核研究后,明确签署意见,报学生工作部(处)审批。

(七)学生工作部(处)审核并进行备案登记。审核通过后,学生方可搬离学生宿舍,办理退宿手续之前应结清水电费并由后勤保障部门、物业管理部门安排楼栋管理员检查验收家具等公用设施,若有损坏,应照价赔偿。

(八)办理完退宿手续后,到财务处办理学生住宿费注销手续。

第十条 因休学、调整专业和年级等发生异动需调整住宿,或者在学生宿舍外住宿期满等原因要求回到学生宿舍住宿的普通本科生,须持教务处相关文件、所属学院相关证明(在学生宿舍外住宿期满的同学只需要学院证明)到学生工作部(处)办理相关入住手续。

第十一条 入住研究生要求退宿时,须在每年6、7月份(研究生新生在入住二周内)提出申请,导师签字并经学院研究生工作办公室同意后,方可退宿。办理退宿手续之前应结清水电费并由后勤保障部门、物业管理部门安排楼栋管理员检查验收家具等公用设施,若有损坏,应照价赔偿。退宿研究生原则上不再安排住宿。

第十二条 学生入住时,每人配备房门钥匙一把,办理入住手续时领取。钥匙必须妥善保管使用,不得自行更换门锁或加挂门锁。值班室备用钥匙仅供应急使用,学生须凭学生证等有效证件方可借用。

第十三条 学生毕业办理退宿手续之前应结清水电费并由后勤保障部门、物业管理部门安排楼栋管理员检查验收家具等公用设施,若有损坏,应照价赔偿。办理完退宿手续的学生应该搬离学生宿舍,搬离时应将房间钥匙交还到本楼栋值班室值班人员处。

第三章 日常管理

第十四条 学生宿舍日常供电时间为6:00~23:00,周五、周六及每年5月1日至10月6日全天供电。法定节日和遇重大活动等情况时的供电,由学校统一安排。学生在宿舍的用水用电标准按国家及学校有关管理规定执行(根据鄂价费〔2006〕183号文件精神,免费额度为用水3吨/月生,用电8度/月生),超出部分由学生自己承担。

第十五条 学生宿舍楼每天23:00至次日6:00关闭大门,晚归者需出示学生证或相关证件登记后,经值班人员许可方可进入。

第十六条 学生应爱护公物,按规定使用宿舍楼和宿舍内家具等各种公共设施,并应妥善保管。发生自然损坏,应及时到本楼栋值班室登记报修;如属人为损坏,须照价赔偿。

第十七条 学生宿舍内务、卫生、美化属于学生自理范围。学生应根据本宿舍实际情况,制定寝室公约,张贴在寝室醒目处,并监督落实;建立轮流值班制度,保持宿舍整洁卫生,做到墙面无球印、脚印,无乱写字迹和乱贴字句。宿舍垃圾须主动清扫到垃圾桶内,任何时间不得清扫到走道,以免影响他人正常生活。

第十八条 学生宿舍楼出入口、走廊、楼梯等公共场所或消防通道,应保持通畅。禁止在以上地方停放自行车和堆放其它物品,自行车、电动车等交通工具应按指定位置摆放整齐。

第十九条 在学生宿舍使用计算机等安全电器,必须注意用电安全,必须遵守学校作息时间,按时熄灯就寝。携带大件行李及贵重物品进出学生楼时,必须到本楼栋值班室登记,征得值班人员同意后方可进出。

第二十条 未经学校同意,任何单位或个人不允许在学生宿舍区内进行各种商业宣传、经商、服务等活动,一经发现,由保卫处处理。

第二十一条 学生宿舍里是学生集体学习生活的场所,禁止以下行为:

(一)严禁使用大功率电器,不得使用煤油炉、酒精炉、液化气炉、电饭煲、电炉、热得快、功率转换器等用具;

(二)严禁私拉电线、私自换保险,不得使用蜡烛等其他可能危害公共设施及人身安全的用火、用电,不得使用劣质充电器、路由器等“三无”电器;

(三)严禁在宿舍内焚烧物品,防止发生火灾事故;

(四)严禁将易燃易爆、有毒等危险品带入学生宿舍;

(五)不得损坏宿舍楼和宿舍内各种公共设施,杜绝聚众起哄、打、扔、摔、砸等不文明行为;

(六)不得私自更换门锁,确因需要更换门锁的须交付本楼栋值班室备用钥匙,离开宿舍或休息时要锁好房门;

(七)不得私自留宿亲友、异性及其他外来人员;

(八)不得在宿舍内饲养宠物;

(九)不得在学生宿舍内、楼道或走廊从事打球、溜冰、大声喧哗等影响和妨碍他人生活学习的活动;

(十)不得将剩饭、剩菜以及容易堵塞水管之物倒入池内,不得从窗户向外泼水、抛果皮、纸屑及其他垃圾杂物;

(十一)不得打麻将或赌博行为,不得酗酒滋事,不得观看、传播反动、淫秽出版物,不得盗窃或抢劫公私财物,不得敲诈勒索钱财,不得寻衅滋事,不得攻击侮辱他人,不得危害他人人身安全,不得侮辱他人人格,不得传播封建迷信思想及非法组织活动。

第二十二条 建立学生宿舍学生管理委员会,协助参与学生宿舍的管理与服务等工作,积极参与星级寝室评比活动,开展文明宿舍建设,反映学生需求,引导学生实现“自我教育、自我管理、自我服务”。

第四章 会客制度

第二十三条 学生宿舍采取会客登记制度,非本楼人员进楼必须先到该楼栋值班室登记,经值班人员同意后方可入内。

第二十四条 女生宿舍楼实行封闭管理,男性不得进入。男生宿舍楼晚上21:00之后,女性不得进入;晚21:30之前,女性必须离开男生寝室。

第五章 附 则

第二十五条 凡违反本规定的,将视情节轻重根据武汉理工大学学生违纪处分相关规定给予纪律处分。

第二十六条 本规定由学生工作部(处)负责解释。

第二十七条 本规定自公布之日起执行。

\chapter{武汉理工大学学生宿舍空调设备使用管理暂行办法}
第一章 总 则

第一条 为进一步改善学生的学习与生活环境,加强学校学生宿舍空调设备的使用与管理,提高资源使用效率,确保使用安全,特制定本办法。

第二条 本办法适用于由学校在所有学生宿舍统一安装的空调设备。

第三条 本办法规定的空调设备,包括学生宿舍空调室内机、室外机、线控器及相关附件等。

第二章 空调设备的日常管理与安全使用

第四条 学生宿舍安装的空调设备是学校的固定资产,由学校后勤保障处委托物业公司进行日常管理。空调设备的开启、关闭由学生自主负责。

第五条 夏季最低温度在摄氏28度以上、冬季最高温度在摄氏10度以下时,学校对学生宿舍空调设备实行24小时供电。

第六条 学生宿舍空调设备使用者应按照《空调使用说明书》操作。使用者在使用过程中因操作不当、违章操作或故意损坏等原因造成设备损坏或人身伤害的,自行承担相应的经济与法律责任。使用者应注意以下事项:

(一)开机前检查线控器模式设置是否正确,设置正确后方可开机,重新开机需间隔3-5分钟。

(二)为保证室内空气质量及空调效果,应定期拆下空调室内机滤网进行冲洗,严禁直接用水冲洗空调室内机。

(三)空调设备电源专用,不得插接其他电器或拉接临时电源。

(四)禁止在空调室内机和管道上覆盖物品、悬挂衣物等。

(五)禁止在空调室外机上摆放任何物件。

(六)不宜长时间停留在空调室内,房间需经常通风换气,保持室内空气良好。

(七)如遇雷雨天气,应关闭空调电源,避免遭受雷击。

第七条 加强节能环保意识,提倡节约用电,使用空调时应关闭门窗,夏季室内空调温度设置不得低于26摄氏度,冬季室内空调温度设置不得高于20摄氏度。室内温度适宜时,建议减少空调的使用。

第三章 空调设备的.维护和维修

第八条 空调设备的维护、保养由学校指定的设备供应商负责。其他任何部门和个人不得私自拆卸室内机和室外机(机滤网除外)。学生如发现空调出现异常情况应立即关闭空调,并及时向楼栋管理员报修。

第九条 空调设备出现故障由学校指定的设备供应商负责维修。严禁学生擅自排除故障。

第四章 收费管理

第十条 安装空调的学生宿舍,每生每年住宿费按政府物价部门核定的标准收取,按学年缴纳。

第十一条 学生宿舍空调使用的电费,按规定的居民生活用电价格由本宿舍的同学共同承担。空调使用实行学生自愿消费和协商使用原则,如在使用过程中发生电费分摊争议等问题,由本宿舍成员协商解决。

第十二条 学生搬离宿舍时,应确保空调设备状态完好,如发现人为损坏,使用者应对损坏设备进行相应赔偿。

第十三条 学生在使用空调的过程中应随时关注本寝室的剩余电量,在剩余电量小于10度时请及时进行电费充值。

第五章 附 则

第十四条 本办法由学生工作部(处)负责解释。

第十五条 本办法自公布之日起实施。

\chapter{武汉理工大学全日制本科生图书馆图书资料借阅管理办法}
为了充分发挥学校图书馆图书资料的功能,更好地为广大本科生的学习、科研服务,加强图书馆图书资料的管理,特制订本管理办法。

一、读者校园卡管理

(一)读者校园卡资格认证

1、具有学校学籍的全日制本科生名单,通过自主入馆教育和考试后,自动获得校园卡读者资格认证。只有获得图书馆资格认证的校园卡,才能在图书馆借阅图书,在图书馆网站利用各项资源与服务。

2、图书馆根据学校规定的本科生学制年限设定校园卡的图书借阅期限,过期即行失效。

(二)读者校园卡遗失处理

1、已获得图书馆读者资格认证的校园卡丢失,应立即登录图书馆网站个人信息系统挂失;或到就近图书馆借书处办理挂失。因校园卡挂失前被他人使用而造成的图书资料损失,由丢失者本人负责。

2、挂失后在校园卡卡务中心补办的校园卡,需到图书馆借书处办理解挂手续,才能恢复图书借阅功能。

(三)离校手续办理

1、毕业生离校,应统一按学校安排的日程就近在各校区图书馆办理离校手续。在办理离校手续时,应还清所借图书及有关欠款。

2、特殊情况需办理离校手续的学生,可随时到所在校区图书馆借书处办理。

(四)借阅权限

1、校园卡(借书证)仅限本人使用,不得转借他人。

2、本科生借书册数权限为18册,借期为30天。

二、图书借阅办法

(一)外借图书

1、图书馆图书实行开架借阅,读者凭本人校园卡进入书库选书。为保证架上图书的有序性,在书架上选书时请使用代书板。

2、读者应注意检查所选图书有无污损(涂写、裁割、水渍、破损等),若有污损,应请工作人员处理后再借。

3、选好书后,通过自助借还设备或到借书处办理借书手续。

(二)馆内阅览图书

1、读者凭本人校园卡进入借阅区(书库或阅览室,下同)阅览,在书架上选书时请使用代书板。

2、每位读者每次限取期刊或图书3册,特殊需要向工作人员说明,阅毕应归还到原位或借阅区指定位置。

(三)图书预约、续借及催还

1、图书预约

(1)读者要借的书若处于借出状态,可通过电话、Email预约或在借书处登记预约。

(2)预约范围为重要的、学术价值高的或新版的图书资料。不能预约自己所借的图书和文艺小说类图书。

(3)预约图书到馆后,图书馆通过电话或Email通知读者,通知三天后读者不来办理手续,视为自动放弃预约权。

2、图书续借

读者所借图书到期仍需继续使用,在无读者预约该书的前提下,可到借书处续借或登录图书馆网站个人信息系统续借一次,延长借期30天。

3、图书催还

(1)读者所借图书在到期前3天,图书管理系统自动发送电子邮件到读者提供的电子邮箱内,提醒读者尽快还书;在图书到期当天,系统会再次发送电子邮件提醒读者及时还书。

(2)此项服务仅限于向图书馆提供了电子邮箱的读者,读者可登录图书馆网站个人信息系统中提交电子邮箱。

(四)外借、阅览注意事项

1、保持书刊架位整齐,不乱取乱放。

2、应爱惜图书,不污损图书。

3、所借图书应按时归还,若图书丢失,应及时到馆处理,以免超期。

4、规定不外借的书刊,可在馆内复印,但需办理复印手续。

5、读者离开借阅区时,应带走所有个人物品。

6、读者通过出口处,若遇监测仪报警,要主动配合工作人员查明原因,方可离开。

三、读者登录图书馆网站个人信息系统的途径

1、图书馆网站地址:http://www.lib.whut.edu.cn/,网站主页读者登录框登录

2、手机移动图书馆:http://m.lib.whut.edu.cn/,“个人中心”栏目登录

3、图书馆微信平台登录

四、违章行为的处理

1、冒用他人校园卡、借书逾期、污损书刊、遗失书刊等纸本文献借阅违章行为的处理,详见图书馆网站规章制度栏目《图书馆图书借阅违章行为的处理办法》。

2、恶意下载等电子资源违规行为的处理,详见图书馆网站规章制度栏目《关于合理使用图书馆电子资源的暂行规定》。

五、本办法由图书馆负责解释。

\chapter{武汉理工大学大学生医疗管理办法}
第一条 根据国务院办公厅[2008]119号、湖北省政府(鄂政办发[2009]21号)、武汉市政府(武政规[2009]10号)和(武人社发[2009]4号)等文件精神,自2009年9月1日起将全日制大学生纳入城镇居民基本医疗保险范围,结合学校有关规定,制定本办法。

第二条 大学生通过参加城镇居民基本医疗保险、补充商业医疗保险和学校日常医疗三条途径来保障基本医疗需求。

第三条 大学生城镇居民基本医疗保险年度参保时间为每年的9月1日至10月31日。当年9月1日至次年8月31日为一个保险年度,年度保险费由学校随学费一起代收代缴。鼓励大学生自愿参加补充商业医疗保险。

第四条 新生入学,必须经校医院健康体格复查。经过复查合格,取得学籍后,由校医院发放病历,即开始享受学校医疗费待遇。

第五条 因体检不合格而保留入学资格回家治疗者,所发生的一切费用自理。

第六条 学生入学前患有的一切慢性病、先天性、免疫性等疾病,且不符合退学标准者,其在校期间因该疾病就诊的医疗费用学校不予报销,住院的学生可通过参加城镇居民基本医疗保险报销、符合补充商业医疗保险赔付范围的疾病由补充商业医疗保险赔付。

第七条 学生凭校医院病历及校园卡在校医院挂号就诊,在校医院门诊和住院的医疗费自付比例按《武汉理工大学大学生医疗管理办法实施细则》执行。

第八条 学生因病情需要转校外医院就诊,必须经校医院相关科室主任同意并开具转诊单后,方可转外院就诊;转外住院者到对口医保医院住院治疗。门诊自行外出就诊或取药,按自费处理。

第九条 学生因急症不能到校医院就诊时,可就近诊治,应于2个工作日内到校医院补办有关手续,由校医院决定是否在校医院或转院治疗。

第十条 经校医院同意转外医院就诊者,门诊医疗费报销时需提供以下资料:1)本人校内病历;2)转诊单;3)校外医院病历;4)转诊医生审核签字的有效发票,急诊要出具盖有“急诊”字样的发票和急诊病历;5)双处方或明细清单。

住院和门诊重症治疗由大学生城镇居民基本医疗保险报销和补充商业医疗保险赔付。

第十一条 外出实习的学生在外地急诊的门诊医疗费用,凭学生所在学院证明、急诊病历及有效发票按学校规定比例报销;住院治疗按城镇居民基本医疗保险、补充商业医疗保险相关规定办理报销、赔付手续。

第十二条 寒、暑假回家探亲患病治疗费用,门诊按《武汉理工大学大学生医疗管理办法实施细则》报销,住院的学生可通过城镇居民基本医疗保险报销、补充商业医疗保险赔付。

第十三条 在籍学生因病休学者,第一年可在当地选一所县级或以上医保医院就诊,门诊医疗费按《武汉理工大学大学生医疗管理办法实施细则》报销。休学期间住院学生的医疗费通过城镇居民基本医疗保险报销、补充商业医疗保险赔付。

第十四条 医疗政策规定的不予报销的诊疗项目和医疗服务设施,如:挂号、病历工本、保险、体检、矫形美容、交通肇事、打架、斗殴、酗酒、自杀等发生的医疗费用一律自理。

第十五条 学校病历遗失者,按《学校医院病历管理制度》补办。

第十六条 凡将病历和校园卡转让他人在校内医院就诊、假冒医生笔迹开处方或病假条者,按“学生违纪处分条例”有关条款处罚,其发生的医疗费用自理。

第十七条 若发现新生体检中有冒名顶替等作弊行为,其发生的医疗费用自理。

第十八条毕业生离校时,应到校医院财务室办理离校手续并结清医疗费用。

第十九条 本办法由学校医疗费管理办公室(大学生医保办)负责解释。

\chapter{武汉理工大学学分制收费管理实施办法}
校财字〔2011〕10号

第一章 总 则

第一条 根据《省物价局省财政厅省教育厅关于印发<湖北省普通高等学校学分制收费管理办法>的通知》(鄂价费规〔2011〕67号)精神及我校实际,制定本办法。

第二条 学校全日制本科生(不含国际合作办学)学费适用本办法。

第三条 学分制学费由专业注册学费(以下简称专业学费)和课程学分学费(以下简称学分学费)两部分组成。学生按学分制正常完成学业所缴纳的学分学费与专业学费之和,不超过物价部门规定的学年制学费总额。

第二章 收费标准

第四条 学分制学费总额=专业学费标准$\times$学制+每学分学费标准$\times$总学分数。

专业学费标准=(专业学年制学费总额-每学分学费标准$\times$总学分数)$\div$学制。

学分学费按学生实际所修学分计收,学分学费标准不分专业,为每生每学分80元。

第五条 学年制学费的具体收费标准将根据物价部门核准的标准执行。

第三章 缴费规定

第六条 学生须在新学年开学初缴清本学年专业学费后,方能注册,并取得选课资格。对于确因家庭经济困难不能按时缴纳者,按照国家和学校有关规定,采取奖、贷、补、减和“绿色通道”等多种方式进行资助,及时办理有关注册和选课手续。

第七条 学分学费实行“先选课后缴费”的原则。学生应根据所在专业的培养方案及个人学习发展规划,在导师指导下,按规定的时间登陆教务管理信息系统选择拟修读的课程,计算出所选课程的总学分后,按每学分80元的标准缴费。每位学生每学期所选课程总学分原则上不得低于15学分,不得超过30学分。

第八条 学生可按学籍管理规定,申请免听课程、免修课程。免听课程,须按要求完成课程作业、实践环节和课程考试,缴纳该课程25%的学分学费;免修课程,不缴纳学分学费。

第九条 按合作协议培养的学生在外校学习期间,专业学费由学校收取,学分学费按合作协议规定执行。

第四章  学费结算

第十条 学生发生学籍异动的,按如下情况分别结算费用:

(一)对弄虚作假经学校复查不符合国家招生规定入学的,以及按学校有关管理规定作退学处理、开除学籍的,不退还专业学费和已修课程的学分学费。

(二)在校期间需转专业的学生,转学当年按转入、转出专业学费标准各一半交纳。转入前按规定已合格修完的课程,不再收取学分学费,学生已修课程学分学费不予退还。

(三)因故休学(保留学籍)、中断学业或因出国、退学、转学等原因终止学习的学生,按月计收当年的专业学费。休学期满复学,当年按月计收专业学费。专业学费按每学年10个月计算,不足一个月的按一个月结算。

(四)提前毕业或延长学习年限的学生,按照实际学习年限交纳专业学费。

第十一条 学生在专业培养方案规定的毕业最低学分以外加修、辅修、跨专业选修,或经一次免费补考后仍不及格需重修课程学分的,免收专业学费,按所修课程的学分收取学分学费。

第十二条 学生在毕业(结业、肄业)前,必须缴清专业注册学费和所修课程学分学费后,方能取得毕业(结业、肄业)资格,办理离校手续。

第五章 收费管理

第十三条 每学期开学前,学校按照国家和省规定的时限和程序,到省物价部门办理《收费许可证》。收费时使用财政部门统一印制的非税收入票据,所收资金全额上交中央财政,实行“收支两条线”管理,接受物价、财政、教育部门的监督检查。

第十四条 严格执行收费公示制度。在招生简章、新生录取通知书中注明学分制收费方式、收费项目、收费标准,在校内通过校园网、公示栏等方式将收费项目、收费标准、收费资金的使用情况和投诉电话等进行公示,接受社会监督,增加收费透明度。

第十五条 学分制收费按“新生新政策、老生老办法”的原则执行。

第六章 附 则

第十六条 本办法由计划财务处和教务处共同解释,自2011年秋季开学起实施。

\chapter{武汉理工大学普通全日制本科学生学生证、校徽管理办法}
第一条 学生证、校徽是证明学生身份的证件和标志,只限于本人使用,严禁涂改和转借。

第二条 新生办理入学注册手续后,发学生证一本,校徽一枚。

第三条 学生证和校徽须妥善保管。如有遗失或损坏,应由本人写出书面申请,学院辅导员签字并盖章,到学生工作处学生管理办公室或余家头校区学生工作办公室申请补办。

第四条 因家庭地址变动,需要更改乘车区间的学生,应持原学生证和所在学院证明以及变动后的家庭所在当地派出所证明到学生工作处学生管理办公室或余家头校区学生工作办公室换证。

第五条 学生在校期间,一般限补办一次学生证。

第六条 补办学生证、补领假期乘火车优惠卡、校徽需交纳成本费。

第七条 补发学生证、校徽以后,若原件寻获,应将原件交回学生工作处注销,不得继续使用;学生一人不得持有两个学生证。

第八条 凡擅自涂改学生证、利用学生证弄虚作假以及将学生证转借他人使用者,或一人持有两个学生证者,根据情节予以批评教育直至纪律处分。

第九条 凡因毕业、修业期满、退学或开除学籍的学生,在办理离校手续时应将学生证交由学院学生工作办公室注销,不注销者不予办理离校手续。校徽由本人保管。

第十条 本办法适用于我校普通全日制本科学生的学生证、校徽的管理。

第十一条 本办法由学生工作部(处)负责解释。

第十二条 本办法自发布之日起执行。



武汉理工大学航海类专业半军事化管理办法

第一章 一日生活制度

第一条 起床

1.各级值班员应提前起床,以便督促学生按时起床及检查学生起床情况。

2.起床号音响后,全体学生应立即起床。

3.集合号音响后,全体学生应迅速按规定到指定地点集合。

4.各级值班员在指定地点集合整理队伍,清点人数,作好出操准备。

5.无论什么天气,学生均必须按时起床,遇雨不能集合由值班员在宿舍检查起床情况。

第二条 早操

1.除因天气原因和节、假日不出操外,其它均按规定出操。因病或有特殊情况不能出操者须持医生证明请假或经中队长批准。

2.全体学生应认真参加早操,切实使早操能够达到锻炼身体,锻炼意志的目的。

3.早操不得迟到早退。

第三条 内务卫生

1.早操后进行洗漱、整理内务和打扫清洁卫生。洗漱应遵守秩序,注意节约用水。每个学生在早读前应整理好自己的床铺桌椅及书架。

2.内务卫生由值日生在洗漱完后进行,任务包括清扫地板,抹桌擦窗,监督其他学生整理内务,保证室内整洁和安静。

3.每个同学都应保持室内整洁和安静。

第四条 进餐

1.学生进餐时要遵守秩序,自觉排队购买饭菜,不得在食堂大声喧哗和敲打碗筷。

2.积极维护食堂卫生,不准随地乱倒饭菜和洗碗水。

3.学生对食堂工作及伙食方面的意见,应找管理人员及有关领导反映,不得同炊事、服务人员争吵、打骂。

第五条 上课

1.上课预备铃响后,学生应带好课堂书籍及用具到指定教室,作好上课准备,不准迟到早退。

2.每门课上第一节课时,区队长(大班课时由值班区队长)下达“起立”口令,学生全体起立并向老师行注目礼,待老师还礼后指示坐下来方可坐下。下课时,区队长下达“起立”口令,向老师行注目礼,待老师还礼后走出教室或老师同意下课,学生才可离开教室。

3.上课应集中精力听讲,不准睡觉,不准做与课程无关的事情。学生提问须先举手,等老师许可后方能起立提问。老师提问时,学生应起立回答,答完经老师许可后方能坐下。

4.上课不准会客,如有特殊情况必须离开教室,须经老师批准。因病不能上课须持医生证明请病假,因事不能上课须经有关领导批准。

5.上课时着装整齐,不准穿背心、内短裤和拖鞋进教室。

6.学生应爱护教室内的一切设备,保持教室和走廊清洁。

第六条 午休

1.午休时间应保持宿舍安静。

2.午休时间未经批准不得组织其它活动。

3.午休后应按时起床并整理好内务。

第七条 自习

1.自习时间为学校规定的学习时间,主要用于消化当天的学习内容,完成作业和预习第二天的课程,不得做与学习无关的事情,自习时间未经请假不得离开学校。

2.自习在寝室、教室或图书馆、阅览室进行。自觉遵守自习纪律,保持自习场所安静,不得在自习场所大声喧哗、弹奏乐器、放收音、收录机。

第八条 自由活动

1.下午第二节课后到晚自习前这段时间是自由活动时间,自由活动时间主要用于开展文体活动和处理个人的事情。

2.学生在自由活动时间应积极参加各项有益活动,不准利用此时间外出聚餐、聚赌和做无益身心健康的事情。

3.各大队、中队、区队要充分利用自由活动时间开展各项有益活动和进行思想、学习交流,推进精神文明的建设。

第九条 就寝与熄灯

1.全体学生依照规定按时熄灯就寝。

2.熄灯后应保持安静,不准讲话,放收音、收录机,弹乐器及搞其它活动。

3.中队值班员应督促学生按时就寝,负责检查熄灯情况。

4.学生未经批准一律不得外出住宿,学生在节假日外出必须在晚自习前返校,未经批准一律不准在宿舍留人住宿。

第二章 晚点名制度

第十条

1.晚点名每周进行一次,定于星期天晚上自习开始进行。

2.晚点名以中队为单位进行,内容为清点人数,进行讲、评活动,布置工作、学习任务。

3.每个学生必须自觉参加晚点名,不准无故缺席。

4.晚点名由中队长主持,点名前必须作好充分准备,分析本中队有关情况,严格进行点名和讲评活动。

5.学生管理部门、各学生大队,平时还应进行不定期的检查晚点名工作,抽查学生的基本情况。

第三章 请、销假制度

第十一条

1.学生外出时间较长(超过半天)应向区队长说明去向,并办理请假手续。

2.学生非节假日外出或不能坚持参加正常学生活动,必须按级请假并按时销假,未按时销假者作旷课论处。

3.学生病假凭医院医生证明方可休假。

4.学生请假一天以内由中队长批准,一天以上三天以内由大队批准,三天以上一周以内报上级批准。

第四章 着装礼貌

第十二条

1.凡参加集合、会操、阅兵及其它大型活动要求着装时,全体学生应按规定统一着装。

2.学生穿校服时,必须按学校规定戴正帽子,扣好衣裤钮扣,扎学校统一定制的领带。

3.学生平时着装必须整洁大方。

4.经常修理须发,禁止染发,保持仪容端庄。

5.学生应以《高等学校学生行为准则》严格要求自己。

6.学生应尊敬师长,团结同学,说话和气,讲究礼貌。

7.凡参加学校、大队的集会活动,学生必须准时整队进场,会议结束后,按规定顺序退场。

第五章 值班制度

第十三条

1.各大队每天都应安排专人值班,负责检查当天本大队的有关情况,值班计划由大队负责制订。

2.值班检查,包括早操、清洁卫生、内务管理、上课、午休、自习、就寝、熄灯等。值班人员应作好值班记录,并将检查结果整理上报。

3.值班人员对当天发生的各种问题应采取及时有效的措施进行处理并及时上报有关领导及部门。

4.值班人员必须认真履行职责,坚持原则,认真负责。

5.学校对各大队、中队的值班情况实行定期和不定期的抽查。

第六章 内务卫生制度

第十四条

1.建立宿舍值日员制度,值日员由本室的学生轮流担任。对内务卫生负全面责任。

2.值日员早操后,负责打扫本室卫生,打开水,并督促本室学生整理好自己的内务。

3.室内所有物品在使用方便的情况下都应定位放置,做到整齐、清洁。值日员应督促全室学生维护保持室内整洁。

4.床上用品按规定将被子叠好并将双叠口向外放正,枕头枕巾铺于被子下面,夏季不戴帽子置于被子中央,帽徽向外,上衣不穿时应叠好平放于帽子下。

5.学生严禁向走廊、楼梯、阳台泼水和饭菜等脏物,严禁由窗内向外泼水,严禁随地乱丢果皮纸屑及其它杂物,严禁在宿舍打球、溜冰等。

6.学生宿舍内不准乱挂衣物,衣服应在规定地方晾晒。

7.每星期四下午进行大扫除,并进行检查评比。

8.节约用水用电,不准在宿舍私接电线,不准私用电热器具。

第七章 检查评比制度

第十五条

1.各大队每天对本大队学生的早操或队列训练情况进行检查,学校对早操或队列训练进行抽查。

2.每周以大队为单位进行一次全面内务卫生检查,检查结果以大队为单位公布。

3.学校不定期进行全校内务卫生大检查,对检查结果进行讲评,优秀者给予奖励,差者进行批评,限期改正。

4.每天的午休、自习等时间由大队组织进行抽查,并作好检查记录。

5.学校对上课、午休、自习、熄灯就寝等情况进行抽查,对抽查结果实行公开通报和内部通报。

6.平时上课、集体活动、劳动、实验、实习等,以分队为单位进行考勤,考勤每月以大队为单位上报。

7.执勤队长和学生纠察队每天对当日生活制度进行检查,学生必须服从执勤队长和学生纠察队的检查、督促和管理,根据检查情况,及时通报,并作为对学生奖惩依据之一。对违纪严重,不服从纠察管理者,从严从重处理。

第八章 考勤

第十六条 学生上课、学习、军训、劳动、实习等均要考勤。考勤由学生班长负责,然后上报中队,由中队上报大队。

第九章 执勤队长

第十七条 为了保证半军事化管理的有效进行,将实行执勤队长及学生纠察队制度。执勤队长由各大队副大队长和中队长轮流担任。

第十八条 执勤队长的职责和要求:

1.执勤队长代表学校负责带领学生纠察队,对水上专业学生午睡按时起床,按时熄灯就寝等情况进行检查和督促,对不服从管理的学生,执勤队长有权处理紧急情况,并及时上报学校各有关部门或直接报告校领导。

2.执勤队长每人值勤一周,从周一早上开始接班到周日晚熄灯就寝后交班。

3.执勤队长在执勤周时,侧重执勤工作同时作好所属中队的学生管理教育工作。

4.执勤队长在执勤时,必须着装,佩戴标志,以身作则,严格执行学校的规章制度,认真负责地履行职责。

第十章 违纪处分

第十九条 凡实行半军事化管理的年级或专业的学生,遵照此办法办理。不遵守半军事化管理的办法及规定,给予相应的纪律处分。

第二十条 本办法由学生工作部(处)负责解释。

\chapter{武汉理工大学课堂纪律规范}
一、课堂是进行教学活动的重要场所,必须保持严肃、安静、清洁、整齐。

二、教师应按时上、下课;上课须衣容整洁大方;课堂、课间均不得在教室内吸烟。

三、学生必须在上课铃响前进入教室并就座,教师走向讲台后,全体学生须起立,待教师答礼后方可坐下。迟到学生应在门口报告取得教师同意后方准就座听课。教师未宣布下课,学生不得擅自离开教室。

四、学生进入教学楼要衣着整齐大方,不得穿拖鞋或赤脚、穿背心、短衬裤进入教室,夏季穿着不得过于暴露,不得携带食物进入教学楼。

五、上课期间,学生应认真听讲,不得谈笑喧哗,不得影响他人听课,不得随意进出教室。前堂未下课,后堂课学生不可闯进教室抢占座位。

六、保持教室整洁、卫生。严禁吸烟、随地吐痰、乱丢杂物、纸屑,不得在桌上涂写、刻划或随意张贴。

七、上课时间无关人员不准随意进出教室或在课堂内逗留,重修生和进修生必须按规定办理听课手续,对无听课证者,教师和班干部有权予以清退。

八、教师应严格要求学生遵守课堂纪律,对违反课堂纪律的学生,教师应予以制止并给予适当的批评,严重者可令其退出课堂,课后报教务处和学生所在学院学工办予以教育和处理。

九、上课时间教师和学生都必须关闭所有通讯工具。

\chapter{武汉理工大学校园卡使用管理办法}
第一条 为规范武汉理工大学校园卡(以下简称校园卡)的使用和管理,充分利用校园卡为我校师生员工的学习、工作、生活服务,根据学校实际情况,特制定本管理办法。

第二条 校园卡是学校为方便师生员工的学习、工作、生活和加强学校管理而制作发行的一种电子卡片。它具有校内电子钱包和身份识别两大功能,凭此卡可在校园范围内进行各类消费、缴纳费用、借阅图书、上机操作、看病治疗、出入特殊场合等活动。校园卡是我校各类人员验证身份的有效证件。

第三条 凡取得我校学籍的学生和我校在册各类工作人员、离退休人员,均有权申请办理正式校园卡一张。首次申领校园卡及其使用手册实行免费。

第四条 在使用校园卡之前,持卡人必须认真阅读《校园卡使用手册》,学习操作方法,以利正确使用校园卡。

第五条 校园卡服务中心是校园卡的管理和服务部门,有责任对校园卡实行依法监管,并提供与校园卡相关的各类服务业务。

第六条 持卡人对本人的各类消费信息有知情权,可通过各类校园卡系统查询设备和校园网进行查询,必要时可向校园卡服务中心工作人员寻求帮助。

第七条 校园卡服务中心负有为持卡人的各类用卡活动保密的义务,除法律、法规允许外,不得擅自泄漏持卡人的任何隐私信息。

第八条 校园卡系统设备是校园卡正常运行的基础,属国家财产,持卡人负有保护义务。发现异常现象,应及时与校园卡服务中心联系。

第九条 持卡人不得以任何形式伪造、篡改、删除校园卡上的数据信息。

第十条 不得转借、盗用他人校园卡,不得利用他人校园卡从事任何非法活动和不道德行为。

第十一条 遗失校园卡时应及时挂失。因持卡人的原因所造成的财产和名誉损失由持卡人负全责。

第十二条 持卡人离校时,学校将冻结校园卡的身份识别功能,保留校园卡的电子钱包功能。

第十三条 对违反本管理办法的相关责任人,必须全额赔偿因违规而造成的经济损失,同时将按有关法律和学校《学生违纪处分办法》予以处理。

第十四条 对违反本管理办法的行为,学校有权视情况采取相应的技术措施,以利最大限度减少损失。

第十五条 对本管理办法中未涉及的其它不良行为,学校有权作出裁决和处理决定。

第十六条 本办法由网络信息中心负责解释。

第十七条 本办法自2008年9月1日起执行。

\part{附录}
\chapter{武汉理工大学校内常用联系方式}
一、常用邮箱:

分管学生工作校领导信箱:xueshenggz@whut.edu.cn

分管教学工作校领导信箱:jiaoxuegz@whut.edu.cn

学生工作部(处)信箱:xuegongbu@whut.edu.cn

教务处信箱:jiaowuch@whut.edu.cn



二、学生工作部(处)、团委、武装部联系电话

综合办公室:87850156(兼传真)

招生办公室:87859017 87858399 87785198

学生心理健康与发展教育中心:

87850042(马区心理教育咨询预约电话)

86565551(余区心理咨询预约电话)

学生教育办公室:87658102

学生管理办公室:87216505(日常管理)

87850976(就业管理)

学生资助与服务中心:87658189 87213483(资助)

学生工作部(处)余区办公室:86551149(综合)

86554452(就业管理)

学生就业创业指导中心:87160203

87858199(就业创业指导咨询)

87851106

87851306(就业信息与市场)

校团委办公室:87850840

校学生会:87857932

武装部办公室:87651416、87651419(传真)



三、教务处联系电话

教务处办公室:87859014

教学研究办公室:87850017

教务管理办公室:87850027

学籍管理办公室:87850013

实践教学办公室:87850020

考务中心:87858410

评建办:87156719

教务处余区办公室:86534486



四、学院学工办、教学办联系电话

材料学院学工办:87652332 教学办:87651776

交通学院学工办:86565841 教学办:86565711

管理学院学工办:87859153 教学办:87658240

机电学院学工办:87651796 教学办:87859133

能动学院学工办:86582033 教学办:86582031

土建学院学工办:87213949 教学办:87651785

汽车学院学工办:87859135 教学办:87859136-825

资环学院学工办:87215931 教学办:87211752

信息学院学工办:87651802 教学办:87399249

计算机学院学工办:87215602 教学办:86551167

自动化学院学工办:87859131 教学办:87859069

航运学院学工办:86581996 教学办:86582958

文法学院学工办:87859799 教学办:87850679

理学院学工办:87651970 教学办:87162636

经济学院学工办:87651812 教学办:87651809

艺术学院学工办:87299305 教学办:87651139

外国语学院学工办:87297047 教学办:87651823

物流学院学工办:86567321 教学办:86562947

政治学院学工办:86561105 教学办:86544129

化工学院学工办:87850939 教学办:87859019

国际教育学院学工办:87643040教学办:87658017

网络(继续)教育学院学工办:87858271 教学办:87651794

职业技术学院学工办:87156839 教学办:87156990



五、学生服务电话

1.保卫处电话

马房山校区报警电话:87651110

余家头校区报警电话:86534110

升升公寓治安室电话:50454127

南湖报警点电话:87757110

2.校医院电话

急诊室(马区西院):87860274

急诊室(马区东院):87850931

急诊室(余区):86557354 3563(内线)

升升医务室:87387122

南湖教学区医务室:87755963

3.现代教育中心电话

马区:教四楼多媒体教室:87156669

鉴主楼多媒体教室:87297283

鉴三楼多媒体教室:87392243

鉴四楼多媒体教室:87391692

南湖新2多媒体教室:87757310

南湖新1、3、4多媒体教室:87756562

余区:南楼多媒体教室:86582996

主楼多媒体教室:86566615

4.后勤服务电话

后勤服务热线:87651111

饮食服务中心:马区:87382546(西院)87858151(东院)

余区:86540930

宿舍管理中心:马区:87855199(西院)

余区:86534529

5.校园卡服务中心电话:87669108(鉴湖)

87859108(东院)

86539108(余区)

校园卡服务自助热线电话:87651807

6.网络信息中心热线电话:87297533(马区)

86557984(余区)
\end{document}